\subsubsection{体捌き・Tai sabaki}
\begin{table}[H]
\begin{center}
\begin{tabular}{rl}
    ? & (\ruby{}{})\\
    ? & (tanuzi no randori)\\
    ontwijkingen kanaliseren & (vrijgevecht)
    %%esquives canalisation? & (?)
\end{tabular}
\end{center}
\label{kyuu_1_taisabaki}
\end{table}
\begin{center}
    1 $\leftrightarrow$ 2
\end{center}

\subsubsection{受身技・Ukemi waza}
\begin{table}[H]
\begin{center}
\begin{tabular}{c}
    Alle valtechnieken.
\end{tabular}
\end{center}
\label{kyuu_1_ukemi_waza}
\end{table}

\subsubsection{突き技・Tsuki waza}
\begin{table}[H]
\begin{center}
\begin{tabular}{c}
    Alle stoottechnieken.
\end{tabular}
\end{center}
\label{kyuu_1_tsuki_waza}
\end{table}

\subsubsection{蹴り技・Keri waza}
\begin{table}[H]
\begin{center}
\begin{tabular}{c}
    \ruby{蹴}{け}り\ruby{}{}の\ruby{型}{かた}\\
    keri goho no kata\\
    5-staps trap kata
\end{tabular}
\end{center}
\label{kyuu_1_keri_waza}
\end{table}

\subsubsection{補助運動・Hojo und\={o}}
\begin{table}[H]
\begin{center}
\begin{tabular}{c}
    Alle voorgaande technieken.
\end{tabular}
\end{center}
\label{kyuu_1_hojo_undou}
\end{table}

\subsubsection{掴み型と手解き・Tsukami kata \& te hodoki}
\begin{table}[H]
\begin{center}
\begin{tabular}{c}
    \ruby{前}{まえ}\ruby{組}{くみ}\ruby{突}{つ}き\\
    mae kumi tsuki\\
    voorwaartse set van stoten\\
    \hline
    \ruby{腕}{うで}\ruby{締}{し}め\\
    ude jime\\
    arm vastklemmen
\end{tabular}
\end{center}
\label{kyuu_1_te_hodoki}
\end{table}

\subsubsection{追加技・Aanvullende technieken}
\begin{table}[H]
\begin{center}
\begin{tabular}{c}
    \ruby{}{}\\
    ?\\
    op elke vorm van aanval en inkomen
\end{tabular}
\end{center}
\label{kyuu_1_additional}
\end{table}

\subsubsection{基本投げ技・Kihon nage waza}
\begin{table}[H]
\begin{center}
\begin{tabular}{rl}
    \ruby{}{}\ruby{}{} & \\
    koshi nage & (ry\={o} te dori)\\
    ? & (?)
\end{tabular}
\end{center}
\label{kyuu_1_kihon_nage_waza}
\end{table}

\subsubsection{基本押え技・Kihon osae waza}
\begin{table}[H]
\begin{center}
\begin{tabular}{rl}
    \ruby{}{}\ruby{}{} & \\
    mukae daoshi & (ushiro uwate)\\
    ? & (?)
\end{tabular}
\end{center}
\label{kyuu_1_kihon_osae_waza}
\end{table}

\subsubsection{歴史的技・Historische technieken}
\begin{table}[H]
\begin{center}
\begin{tabular}{c}
    \ruby{}{}\ruby{}{}\\
    shime gaeshi\\
    ?
\end{tabular}
\end{center}
\label{kyuu_1_historic}
\end{table}

\subsubsection{型・Kata}
\begin{table}[H]
\begin{center}
\begin{tabular}{c}
    \ruby{}{}\ruby{}{}\\
    ken jutsu (itsutsu no tachi)\\
    ?
\end{tabular}
\end{center}
\label{kyuu_1_kata}
\end{table}

\subsubsection{和の精神・Wa no seishin}
\begin{table}[H]
\begin{center}
\begin{tabular}{c}
    \ruby{}{}\ruby{}{}\\
    ?\\
    onder de vorm van een vrij gevecht
\end{tabular}
\end{center}
\label{kyuu_1_wa_no_seishin}
\end{table}

\subsubsection{?・Verdedigingsstok}
\begin{table}[H]
\begin{center}
\begin{tabular}{c}
    \ruby{}{}\\
    tanto waza\\
    ?
\end{tabular}
\end{center}
\label{kyuu_1_defense_stick}
\end{table}

\subsubsection{乱取り・Randori}
\begin{table}[H]
\begin{center}
\begin{tabular}{c}
    \ruby{多人}{たにん}\ruby{数}{ずう}の\ruby{乱取}{らんど}り\\
    taninz\={u} no randori\\
    vrij gevecht tegen een groot aantal personen
\end{tabular}
\end{center}
\label{kyuu_1_randori}
\end{table}
\begin{center}
    1 $\leftrightarrow$ 1..$\infty$
\end{center}

%%%%%%%%%%%%%%%%%%%%%%%%%%%%%%%%%%%%%%
% Imports
%%%%%%%%%%%%%%%%%%%%%%%%%%%%%%%%%%%%%%
\documentclass[a4paper, 12pt]{article}
\usepackage[dutch]{babel}
\usepackage{CJK}
\usepackage[CJK, overlap]{ruby} %furigana support
\usepackage{float}
\usepackage{graphicx}
\usepackage{multirow}
\usepackage{pdflscape}
\usepackage[left=2cm,top=1cm,right=2cm,bottom=2cm, nohead,nofoot]{geometry}
\usepackage[usenames,dvipsnames]{color}
\usepackage{array}


%%%%%%%%%%%%%%%%%%%%%%%%%%%%%%%%%%%%%%
% Types
%%%%%%%%%%%%%%%%%%%%%%%%%%%%%%%%%%%%%%
\newcolumntype{s}{>{\centering\arraybackslash}m{1cm}}
\newcolumntype{B}{>{\centering\arraybackslash}m{4.0cm}}
\newcolumntype{L}{>{\centering\arraybackslash}m{7.0cm}}


%%%%%%%%%%%%%%%%%%%%%%%%%%%%%%%%%%%%%%
% Commands
%%%%%%%%%%%%%%%%%%%%%%%%%%%%%%%%%%%%%%
\renewcommand{\rubysize}{0.5}
\renewcommand{\rubysep}{-0.1ex}
\newcommand{\tran}[1]{{\itshape{\color{Gray}{#1}}}}


%%%%%%%%%%%%%%%%%%%%%%%%%%%%%%%%%%%%%%
% Start of document + settings
%%%%%%%%%%%%%%%%%%%%%%%%%%%%%%%%%%%%%%
\begin{document}
% General settings, which can be used over multiple documents
\setlength{\footskip}{20pt} % Make sure the page number doesn't touch the text.
\setlength{\parindent}{0pt} % Don't indent paragraphs. Then you don't neet to use \noindent everywhere.

\begin{CJK*}{UTF8}{min}
\CJKtilde

%%%%%%%%%%%%%%%%%%%%%%%%%%%%%%%%%%%%%%
% Title page
%%%%%%%%%%%%%%%%%%%%%%%%%%%%%%%%%%%%%%
\title{Aikibudo Terminologie}
\author{Andy Nagels}
\maketitle
\thispagestyle{empty} %should remove the page number
%\pagestyle{empty} %should remove the page number in the whole document
\begin{figure}[H]
\centering
\includegraphics[width=2.5cm]{../img/schild_aikibudo.eps}
\end{figure}

\begin{center}
合気武道
\end{center}

\begin{figure}[H]
\centering
\includegraphics[width=1.0cm]{../img/aikibudo-kanji.eps}
\end{figure}


%%%%%%%%%%%%%%%%%%%%%%%%%%%%%%%%%%%%%%
% Disclaimer
%%%%%%%%%%%%%%%%%%%%%%%%%%%%%%%%%%%%%%
\newpage
\begin{center}
\textbf{Disclaimer}\\
De vertalingen in dit document zijn gemaakt door mezelf, m.b.v. offici\"{e}le documenten, moderne technologie en mijn eigen kennis van het Japans. Daar ik echter geen expert ben in de Japanse taal, is het mogelijk dat er fouten staan in dit document. Naarmate mijn kennis van Aikibudo en Japans stijgt, zal ik de nieuwe kennis die ik heb opgedaan dan ook gebruiken om dit document na te kijken en te corrigeren.\\
Dit document is dus niet als finaal te beschouwen. Het wordt reeds vrijgegeven omdat het kan helpen bij het onthouden van de technieken.
\end{center}
\begin{center}
\textit{{\em Opmerking:} De laatste versie kan gratis gedownload worden van de
volgende url: http://c.dommel.be/aikibudo.html.}
\end{center}


%%%%%%%%%%%%%%%%%%%%%%%%%%%%%%%%%%%%%%
% Table of contents
%%%%%%%%%%%%%%%%%%%%%%%%%%%%%%%%%%%%%%
\newpage
\setcounter{page}{1}
\pagenumbering{Roman}
\tableofcontents

%%%%%%%%%%%%%%%%%%%%%%%%%%%%%%%%%%%%%%
% General info
%%%%%%%%%%%%%%%%%%%%%%%%%%%%%%%%%%%%%%
\newpage
\setcounter{page}{1}
\pagenumbering{arabic}

\section{Inleiding}
\noindent Dit document bevat termen die gebruikt worden in aikibudo.

\section{De Japanse taal}
\noindent Alvorens we de berippen bekijken, dient een korte uitleg gegeven te worden over de Japanse taal.\\

\noindent Japans is opgebouwd uit 3 grote onderdelen: \textit{hiragana, katakana} en \textit{kanji}.\\

\noindent Hiragana en katakana zijn allebei een \textit{phonetisch} alfabet. Dat wil zeggen dat het symbolen voorstelt, die een klank uitbeelden.
Japans heeft dus geen aparte letters, zoals de meeste Westerse alfabetten.\\

\noindent \textbf{Hiragana} wordt gebruikt om zinnen te vormen en grammatica toe te passen op de Japanse taal. Het toont hoe Japans wordt uitgesproken.\\

\noindent \textbf{Katakana} bevat dezelfde klanken en enkele extra klanken. Het is ontstaan om de verschillende buitenlandse termen te kunnen uitspreken. Zo worden bvb. namen van niet Japanners in katakana geschreven.\\

\noindent \textbf{Kanji} zijn de meer complexe Japanse symbolen, die woorden en begrippen voorstellen. Ook Japanse namen en plaatsnamen worden meestal met kanji geschreven. Om te weten hoe deze symbolen worden uitgesproken, wordt gebruik gemaakt van hiragana, omdat hiragana een phonetisch alfabet is dat klanken in beeld brengt.\\

\noindent \textbf{Romaji} is een westerse schrijfwijze van Japanse klanken. Het wordt in Japan ook gebruikt om op een computer te typen.
Het is deze schrijfwijze, die het mogelijk maakt voor mensen die geen Japans kennen, om Japanse termen correct uit te spreken.

\noindent \textbf{Furigana} zijn kleine symbolen boven de kanji die in hiragana weergeven hoe de kanji moet worden uitgesproken.

\noindent In de tabel hieronder, kan je een overzicht vinden van de verschillende alfabetten, ter verduidelijking van bovenstaande uitleg.\\

\begin{table}[H]
\begin{center}
\begin{tabular}{c|c|c|c}
Nederlands alfabet & Hiragana & Katakana & Romaji\\
\hline
a &  あ & ア & a\\
k & bestaat niet & bestaat niet & bestaat niet\\
bestaat niet & か & カ & ka
\end{tabular}
\end{center}
\caption{Een kleine vergelijking ter verduidelijking}
\label{vergelijking_alfabetten}
\end{table}

\noindent De begrippen in dit document, zijn op volgende manier weergegeven:\\

\textit{Nederlandse omschrijving} $|$ \textit{Japanse term [met hiragana uitspraak boven de kanji]} $|$ \textit{romaji uitspraak}\\

\noindent De technieken zijn echter te lang om op die manier weer te geven. Zij staan dus onder elkaar:\\

\begin{center}
\textit{Nederlandse term}\\
\textit{Japanse term [met hiragana uitspraak boven de kanji]}\\
\textit{Romaji uitspraak}
\end{center}


\section{Uitspraak}
\noindent Een aantal letters worden iets anders uitgesproken in de romaji uitspraak, t.o.v. het Nederlands.
Het is interessant om dit even te lezen, om zo geen verkeerde uitspraak te leren voor de techniek.\\

\noindent j: $|zj|$ zoals in strand\textbf{j}anet, \textbf{J}efke, ... (en NIET zoals in \textbf{J}ommeke)\\
ch: $|tsj|$ zoals in \textbf{tsj}oeke \textbf{tsj}oeke tuut tuut\\
sh: $|sj|$ zoals in \textbf{ch}oco, \textbf{sh}ampoo, ...\\
u: korte $|oe|$ zoals in nen t\textbf{oe}k \textbf{oe}p aa bakkes, vake en m\textbf{oe}ke, ...\\
\={o} of ou: lange $|o|$ zoals in b\textbf{oo}t\\
\={u} of uu: lange $|u|$ zoals in v\textbf{uu}r


%%%%%%%%%%%%%%%%%%%%%%%%%%%%%%%%%%%%%%
% General terminology
%%%%%%%%%%%%%%%%%%%%%%%%%%%%%%%%%%%%%%
\section{Algemene terminologie}
\subsection{Groeten}
\begin{table}[H]
\begin{center}
\begin{tabular}{c}
    \ruby{神}{しん}\ruby{前}{ぜん}に\ruby{礼}{れい}\\
    shinzen ni rei\\
    \tran{aan het altaar, een bedanking}\\
    \hline
    \ruby{先}{せん}\ruby{生}{せい}に\ruby{礼}{れい}\\
    sensei ni rei\\
    \tran{aan de meester, een bedanking}\\
    \hline
    お\ruby{互}{たが}いに\ruby{礼}{れい}\\
    otagai ni rei\\
    \tran{aan elkaar, een bedanking}
\end{tabular}
\end{center}
\end{table}


\newpage
\subsection{Algemeen}
\begin{table}[H]
\begin{center}
\begin{tabular}{c|c|c}
basis/oorsprong/standaard & \ruby{基本}{きほん} & kihon \\
\hline
de kunst van het veilig vallen & \ruby{受身}{うけみ} & ukemi \\
\hline
actieve partner & \ruby{取り}{どり} & dori\\
\hline
controle & \ruby{押}{お}さえ & osae\\
\hline
techniek & \ruby{技}{わざ} & waza\\
\hline
houding & \ruby{構}{かま}え & kamae 
\end{tabular}
\end{center}
\end{table}

\subsection{Richtingen en gebieden}
\begin{table}[H]
\begin{center}
\begin{tabular}{c|c|c}
rechts & \ruby{右}{みぎ} & migi \\
\hline
links & \ruby{左}{ひだり} & hidari\\
\hline
achterwaarts & \ruby{後}{うし}ろ & ushiro\\
\hline
voorwaarts & \ruby{前}{まえ} & mae\\
\hline
bovenste graad & \ruby{上段}{じょうだん} & j\={o}dan\\
\hline
midden & \ruby{中段}{ちゅうだん} & ch\={u}dan\\
\hline
grond & \ruby{土}{ど} & do
\end{tabular}
\end{center}
\end{table}

\subsection{Acties}
\begin{table}[H]
\begin{center}
\begin{tabular}{c|c|c}
worp & \ruby{投}{な}げ & nage\\
\hline
kopstoot & \ruby{頭突}{ずつ}き & zutsuki\\
\hline
trap & \ruby{蹴}{け}り & keri\\
\hline
wurging & \ruby{絞殺}{こうさつ} & k\={o}satsu\\
\hline
omkering/terug sturen & \ruby{返}{がえ}し & gaeshi\\
\hline
zijwaartse trap & \ruby{横蹴}{よこけ}り & yoko keri\\
\hline
stoot & \ruby{突}{つ}き & tsuki
\end{tabular}
\end{center}
\end{table}

\subsection{Lichaamsdelen}
\begin{table}[H]
\begin{center}
\begin{tabular}{c|c|c}
lichaam & \ruby{体}{たい} & tai \\
\hline
hand & \ruby{手}{て} & te \\
\hline
elleboog/elleboogstoot & \ruby{肘}{ひじ} & hiji\\
\hline
knie & \ruby{膝}{ひざ} & hiza\\
\hline
pols & \ruby{手首}{てくび} & tekubi\\
\hline
onderarm & \ruby{小手}{こて} & kote\\
\hline
schouder & \ruby{肩}{かた} & kata
\end{tabular}
\end{center}
\end{table}

\subsection{Andere begrippen}
\begin{table}[H]
\begin{center}
\begin{tabular}{c|c|c}
schroef/veer van horloge & \ruby{捻子}{ねじ} & neji\\
\hline
uitwendig & \ruby{表}{おもて} & omote\\
\hline
tabel & \ruby{表}{ひょう} & hy\={o}
\end{tabular}
\end{center}
\end{table}

\subsection{Leerstof}
\begin{table}[H]
\begin{center}
\begin{tabular}{c|c|c}
    verwijdering van het lichaam & \ruby{体捌}{たいさば}き & tai sabaki\\
\hline
valtechnieken & \ruby{受身技}{うけみわざ} & ukemi waza\\
\hline
stoottechnieken & \ruby{突}{つ}き\ruby{技}{わざ} & tsuki waza\\
\hline
traptechnieken & \ruby{蹴}{け}り\ruby{技}{わざ} & keri waza\\
\hline
? & ? [ほじょうんど] & hojo undo\\
\hline
vrij gevecht & \ruby{乱取}{らんど}り & randori
\end{tabular}
\end{center}
\end{table}


%%%%%%%%%%%%%%%%%%%%%%%%%%%%%%%%%%%%%%
% Kyuu techniques
%%%%%%%%%%%%%%%%%%%%%%%%%%%%%%%%%%%%%%
\newpage
\section{級の技・Ky\={u} technieken}
\subsection{6de ky\={u} [SH\={O}KY\={U}]}
\begin{table}[H]
\begin{center}
\begin{tabular}{c|c}
    \textbf{verwijdering van het lichaam} & ?\\
    \textbf{\ruby{体捌}{たいさば}き} & ? [いりみ]\\
    \textbf{tai sabaki} & irimi\\
    \cline{2-2}
    & ?\\
    & ? [おいりみ]\\
    & o irimi\\
    \hline
    \textbf{valtechnieken} & achterwaarts\\
    \textbf{\ruby{受身技}{うけみわざ}} & \ruby{後}{うし}ろ\\
    \textbf{ukemi waza} & ushiro\\
    \hline
    \textbf{stoottechnieken} & ?\\
    \textbf{\ruby{突}{つ}き\ruby{技}{わざ}} & ?\\
    \textbf{tsuki waza} & ? \\
    \hline
\end{tabular}
\end{center}
\label{kyuu_6}
\end{table}


\newpage
\subsection{5de ky\={u} [SH\={O}KY\={U}]}
\subsubsection{体捌き・Tai sabaki}
\begin{table}[H]
\begin{center}
\begin{tabular}{cc}
    ? & \ruby{}{}? \\
    hiraki & nagashi hiki\\
    ? & ? 
\end{tabular}
\end{center}
\label{kyuu_5_taisabaki}
\end{table}

\subsubsection{受身技・Ukemi waza}
\begin{table}[H]
\begin{center}
\begin{tabular}{c}
    \ruby{前}{まえ}\\
    mae\\
    voorwaarts
\end{tabular}
\end{center}
\label{kyuu_5_ukemi_waza}
\end{table}

\subsubsection{突き技・Tsuki waza}
\begin{table}[H]
\begin{center}
\begin{tabular}{c}
    \ruby{}{}\\
    yoko omoto men uchi\\
    ?\\
    \hline
    \ruby{}{}\\
    ura yoko men uchi
\end{tabular}
\end{center}
\label{kyuu_5_tsuki_waza}
\end{table}

\subsubsection{蹴り技・Keri waza}
\begin{table}[H]
\begin{center}
\begin{tabular}{c}
    \ruby{前}{まえ}\ruby{蹴}{け}り\\
    mawashi keri\\
    (cirkel?) trap
\end{tabular}
\end{center}
\label{kyuu_5_keri_waza}
\end{table}

\subsubsection{補助運動・Hojo und\={o}}
\begin{table}[H]
\begin{center}
\begin{tabular}{c}
    \ruby{}{}\ruby{}{}\\
    oshi kaeshi\\
    ?\\
    \hline
    \ruby{}{}\\
    tsuppari\\
    ?
\end{tabular}
\end{center}
\label{kyuu_5_hojo_undou}
\end{table}

\subsubsection{掴み型と手解き・Tsukami kata \& te hodoki}
\begin{table}[H]
\begin{center}
\begin{tabular}{rl}
    \ruby{}{}\ruby{}{} & (\ruby{}{})\\
    gyakute dori & (ry\={o} te ippo dori)\\
    ? & ?\\
    \hline
    & (\ruby{}{})\\
    & (ry\={o} te dori)\\
    & ?
\end{tabular}
\end{center}
\label{kyuu_5_te_hodoki}
\end{table}

\subsubsection{追加技・Aanvullende technieken}
\begin{table}[H]
\begin{center}
\begin{tabular}{c}
    \ruby{}{}\ruby{}{}\\
    ura ude nage\\
    ?
\end{tabular}
\end{center}
\label{kyuu_5_additional}
\end{table}

\subsubsection{基本投げ技・Kihon nage waza}
\begin{table}[H]
\begin{center}
\begin{tabular}{rl}
    \ruby{}{} & (\ruby{}{})\\
    shiho nage & (omote yoko men uchi)\\
    ? & (?)
\end{tabular}
\end{center}
\label{kyuu_5_kihon_nage_waza}
\end{table}

\subsubsection{基本押え技・Kihon osae waza}
\begin{table}[H]
\begin{center}
\begin{tabular}{rl}
    \ruby{}{} & (\ruby{}{})\\
    rofuse & (ry\={o} te ippo dori)\\
    ? & (?)
\end{tabular}
\end{center}
\label{kyuu_5_kihon_osae_waza}
\end{table}

\subsubsection{歴史的技・Historische technieken}
\begin{table}[H]
\begin{center}
\begin{tabular}{rl}
    \ruby{}{}\ruby{}{} & \\
    daito ryu aikijujutsu & ippon dori\\
    ? & ?
\end{tabular}
\end{center}
\label{kyuu_5_historic}
\end{table}


\newpage
\subsection{4de ky\={u} [CH\={U}KY\={U}]}
\subsubsection{体捌き・Tai sabaki}
\begin{table}[H]
\begin{center}
\begin{tabular}{rl}
    ? & (?)\\
    ? & (ichi ? ichi)\\
    esquives canalisation? &  
\end{tabular}
\end{center}
1 \leftrightarrow 1
\label{kyuu_4_taisabaki}
\end{table}
\begin{center}
    1 \leftrightarrow 1
\end{center}

\subsubsection{受身技・Ukemi waza}
\begin{table}[H]
\begin{center}
\begin{tabular}{c}
    \ruby{後}{うし}ろ\\
    ushiro\\
    achterwaarts
\end{tabular}
\end{center}
\label{kyuu_4_ukemi_waza}
\end{table}

\subsubsection{突き技・Tsuki waza}
\begin{table}[H]
\begin{center}
\begin{tabular}{c}
    \ruby{直}{ちょく}\\
    choku tsuki\\
    rechte stoot
\end{tabular}
\end{center}
\label{kyuu_4_tsuki_waza}
\end{table}

\subsubsection{蹴り技・Keri waza}
\begin{table}[H]
\begin{center}
\begin{tabular}{c}
    \ruby{前}{まえ}\ruby{蹴}{り}\\
    mae keri\\
    voorwaartse trap
\end{tabular}
\end{center}
\label{kyuu_4_keri_waza}
\end{table}

\subsubsection{補助運動・Hojo undo}
\begin{table}[H]
\begin{center}
\begin{tabular}{c}
    \ruby{}{}\ruby{}{}\\
    nigiri kaeshi\\
    ?\\
    \hline
    \ruby{}{}\\
    neji kaeshi\\
    ?
\end{tabular}
\end{center}
\label{kyuu_4_hojo_undo}
\end{table}

\subsubsection{掴み型と手解き・Tsukami kata \& te hodoki}
\begin{table}[H]
\begin{center}
\begin{tabular}{c}
    \ruby{}{}\ruby{}{}\\
    junte dori\\
    ?\\
    \hline
    \ruby{}{}\\
    dosoku te dori\\
    ?
\end{tabular}
\end{center}
\label{kyuu_4_te_hodoki}
\end{table}

\subsubsection{追加技・Aanvullende technieken}
\begin{table}[H]
\begin{center}
\begin{tabular}{c}
    \ruby{}{}\ruby{}{}\\
    ushiro kata otoshi\\
    ?
\end{tabular}
\end{center}
\label{kyuu_4_additional}
\end{table}

\subsubsection{基本投げ技・Kihon nage waza}
\begin{table}[H]
\begin{center}
\begin{tabular}{rl}
    \ruby{}{}\ruby{}{} & \\
    mukae daoshi & (ura yoko men uchi)\\
    ? & (?)
\end{tabular}
\end{center}
\label{kyuu_4_kihon_nage_waza}
\end{table}

\subsubsection{基本押え技・Kihon osae waza}
\begin{table}[H]
\begin{center}
\begin{tabular}{rl}
    \ruby{}{}\ruby{}{} & \\
    ushiro neji kudaki & (tsuki ch\={u}dan)\\
    ? & (?)
\end{tabular}
\end{center}
\label{kyuu_4_kihon_osae_waza}
\end{table}

\subsubsection{歴史的技・Historische technieken}
\begin{table}[H]
\begin{center}
\begin{tabular}{rl}
    \ruby{}{}\ruby{}{} & \\
    daito ryu aikijujutsu & ikkajo (idori)\\
    ? & ? (?)
\end{tabular}
\end{center}
\label{kyuu_4_historic}
\end{table}

\subsubsection{型・Kata}
\begin{table}[H]
\begin{center}
\begin{tabular}{rl}
    \ruby{}{}\ruby{}{} & \\
    daito ryu aikijujutsu & ikkajo (idori)\\
    ? & ? (?)
\end{tabular}
\end{center}
\label{kyuu_4_kata}
\end{table}

\subsubsection{乱取り・Randori}
\begin{table}[H]
\begin{center}
\begin{tabular}{rl}
    \ruby{}{}\ruby{}{} & \\
    daito ryu aikijujutsu & ikkajo (idori)\\
    ? & ? (?)
\end{tabular}
\end{center}
\label{kyuu_4_randori}
\end{table}


\newpage
\subsection{3de ky\={u} [CH\={U}KY\={U}]}
\subsubsection{・Tai sabaki}
\begin{table}[H]
\begin{center}
\begin{tabular}{cc}
    ? & \ruby{大}{おお}? \\
    irimi & \={o} irimi\\
    ? & ? 
\end{tabular}
\end{center}
\label{kyuu_1_taisabaki}
\end{table}

\subsubsection{・Ukemi waza}
\subsubsection{・Tsuki waza}
\subsubsection{・Keri waza}
\subsubsection{・Hojo undo}
\subsubsection{・Tsukami kata \& te hodoki}
\subsubsection{・Aanvullende technieken}
\subsubsection{基本投げ技・Kihon nage waza}
\subsubsection{基本押え技・Kihon osae waza}
\subsubsection{・Historische technieken}
\subsubsection{型・Kata}
\subsubsection{・Wa no seishin}
\subsubsection{乱取り・Randori}


\newpage
\subsection{2de ky\={u} [J\={O}KY\={U}]}
\subsubsection{体捌き・Tai sabaki}
\begin{table}[H]
\begin{center}
\begin{tabular}{rlrcl}
    ? & (?) & (\ruby{}{} & - & \ruby{}{})\\
    ? & (choku tsuki) & (soto & - & omote)\\
    esquives canalisation? & (?) & (? & - & ?)
\end{tabular}
\end{center}
\label{kyuu_2_taisabaki}
\end{table}
\begin{center}
    1 $\leftrightarrow$ 2
\end{center}

\subsubsection{受身技・Ukemi waza}
\begin{table}[H]
\begin{center}
\begin{tabular}{c}
    Alle valoefeningen.
\end{tabular}
\end{center}
\label{kyuu_2_ukemi_waza}
\end{table}

\subsubsection{突き技・Tsuki waza}
\begin{table}[H]
\begin{center}
\begin{tabular}{c}
    \ruby{}{}\\
    tsuki uchi no kata\\
    ?
\end{tabular}
\end{center}
\label{kyuu_2_tsuki_waza}
\end{table}

\subsubsection{蹴り技・Keri waza}
\begin{table}[H]
\begin{center}
\begin{tabular}{c}
    \ruby{}{}\\
    ura keri\\
    ? trap
\end{tabular}
\end{center}
\label{kyuu_2_keri_waza}
\end{table}

\subsubsection{補助運動・Hojo undo}
\begin{table}[H]
\begin{center}
\begin{tabular}{rl}
    \ruby{}{} & (2 \ruby{}{})\\
    nigiri kaeshi & (2 ?)\\
    ? & (2 vormen)
\end{tabular}
\end{center}
\label{kyuu_2_hojo_undo}
\end{table}

\subsubsection{掴み型と手解き・Tsukami kata \& te hodoki}
\begin{table}[H]
\begin{center}
\begin{tabular}{c}
    \ruby{}{}\ruby{}{}\\
    ushiro ry\={o} sode dori\\
    ?\\
    \hline
    \ruby{}{}\\
    ushiro kubi jime\\
    ?\\
    \hline
    \ruby{}{}\\
    ushiro katate dori\\
    ?
\end{tabular}
\end{center}
\label{kyuu_2_te_hodoki}
\end{table}

\subsubsection{追加技・Aanvullende technieken}
\begin{table}[H]
\begin{center}
\begin{tabular}{c}
    \ruby{}{}\\
    ?\\
    op elke vorm van aanval en inkomen
\end{tabular}
\end{center}
\label{kyuu_2_additional}
\end{table}

\subsubsection{基本投げ技・Kihon nage waza}
\begin{table}[H]
\begin{center}
\begin{tabular}{rl}
    \ruby{}{}\ruby{}{} & \\
    tenbin nage & (ry={o} sode dori)\\
    ? & (?)\\
    \hline
    \ruby{}{} &\\
    hachi mawashi &\\
    ? &
\end{tabular}
\end{center}
\label{kyuu_2_kihon_nage_waza}
\end{table}

\subsubsection{基本押え技・Kihon osae waza}
\begin{table}[H]
\begin{center}
\begin{tabular}{rl}
    \ruby{}{}\ruby{}{} & \\
    shiho nage & (ushiro ry\={o} te dori)\\
    ? & (?)
\end{tabular}
\end{center}
\label{kyuu_2_kihon_osae_waza}
\end{table}

\subsubsection{歴史的技・Historische technieken}
\begin{table}[H]
\begin{center}
\begin{tabular}{rl}
    \ruby{}{}\ruby{}{} & \\
    daito ryu aikijujutsu & kuruma daoshi\\
    ? & ?
\end{tabular}
\end{center}
\label{kyuu_2_historic}
\end{table}

\subsubsection{型・Kata}
\begin{table}[H]
\begin{center}
\begin{tabular}{rl}
    \ruby{}{}\ruby{}{}\\
    ken no kata\\
    ?
\end{tabular}
\end{center}
\label{kyuu_2_kata}
\end{table}

\subsubsection{和の精神・Wa no seishin}
\begin{table}[H]
\begin{center}
\begin{tabular}{c}
    \ruby{}{}\\
    ushiro\\
    ?
\end{tabular}
\end{center}
\label{kyuu_2_wa_no_seishin}
\end{table}

\subsubsection{?・Verdedigingsstok}
\begin{table}[H]
\begin{center}
\begin{tabular}{c}
    \ruby{}{}\\
    tanbo waza\\
    ?
\end{tabular}
\end{center}
\label{kyuu_2_defense_stick}
\end{table}

\subsubsection{乱取り・Randori}
\begin{table}[H]
\begin{center}
\begin{tabular}{rl}
    \ruby{}{} & (\ruby{}{})\\
    futari no randori & (tai sabaki)\\
    ? & (?)
\end{tabular}
\end{center}
\label{kyuu_2_randori}
\end{table}
\begin{center}
    1 $\leftrightarrow$ 3
\end{center}


\newpage
\subsection{1ste ky\={u} [J\={O}KY\={U}]}
\subsubsection{体捌き・Tai sabaki}
\begin{table}[H]
\begin{center}
\begin{tabular}{rl}
    ? & (\ruby{}{})\\
    ? & (tanuzi no randori)\\
    ontwijkingen kanaliseren & (vrijgevecht)
    %%esquives canalisation? & (?)
\end{tabular}
\end{center}
\label{kyuu_1_taisabaki}
\end{table}
\begin{center}
    1 $\leftrightarrow$ 2
\end{center}

\subsubsection{受身技・Ukemi waza}
\begin{table}[H]
\begin{center}
\begin{tabular}{c}
    Alle valtechnieken.
\end{tabular}
\end{center}
\label{kyuu_1_ukemi_waza}
\end{table}

\subsubsection{突き技・Tsuki waza}
\begin{table}[H]
\begin{center}
\begin{tabular}{c}
    Alle stoottechnieken.
\end{tabular}
\end{center}
\label{kyuu_1_tsuki_waza}
\end{table}

\subsubsection{蹴り技・Keri waza}
\begin{table}[H]
\begin{center}
\begin{tabular}{c}
    \ruby{蹴}{け}り\ruby{}{}の\ruby{型}{かた}\\
    keri goho no kata\\
    5-staps trap kata
\end{tabular}
\end{center}
\label{kyuu_1_keri_waza}
\end{table}

\subsubsection{補助運動・Hojo und\={o}}
\begin{table}[H]
\begin{center}
\begin{tabular}{c}
    Alle voorgaande technieken.
\end{tabular}
\end{center}
\label{kyuu_1_hojo_undou}
\end{table}

\subsubsection{掴み型と手解き・Tsukami kata \& te hodoki}
\begin{table}[H]
\begin{center}
\begin{tabular}{c}
    \ruby{}{}\ruby{}{}\\
    mae kumi tsuki\\
    ?\\
    \hline
    \ruby{}{}\\
    ude jime\\
    ?
\end{tabular}
\end{center}
\label{kyuu_1_te_hodoki}
\end{table}

\subsubsection{追加技・Aanvullende technieken}
\begin{table}[H]
\begin{center}
\begin{tabular}{c}
    \ruby{}{}\\
    ?\\
    op elke vorm van aanval en inkomen
\end{tabular}
\end{center}
\label{kyuu_1_additional}
\end{table}

\subsubsection{基本投げ技・Kihon nage waza}
\begin{table}[H]
\begin{center}
\begin{tabular}{rl}
    \ruby{}{}\ruby{}{} & \\
    koshi nage & (ry\={o} te dori)\\
    ? & (?)
\end{tabular}
\end{center}
\label{kyuu_1_kihon_nage_waza}
\end{table}

\subsubsection{基本押え技・Kihon osae waza}
\begin{table}[H]
\begin{center}
\begin{tabular}{rl}
    \ruby{}{}\ruby{}{} & \\
    mukae daoshi & (ushiro uwate)\\
    ? & (?)
\end{tabular}
\end{center}
\label{kyuu_1_kihon_osae_waza}
\end{table}

\subsubsection{歴史的技・Historische technieken}
\begin{table}[H]
\begin{center}
\begin{tabular}{c}
    \ruby{}{}\ruby{}{}\\
    shime gaeshi\\
    ?
\end{tabular}
\end{center}
\label{kyuu_1_historic}
\end{table}

\subsubsection{型・Kata}
\begin{table}[H]
\begin{center}
\begin{tabular}{c}
    \ruby{}{}\ruby{}{}\\
    ken jutsu (itsutsu no tachi)\\
    ?
\end{tabular}
\end{center}
\label{kyuu_1_kata}
\end{table}

\subsubsection{和の精神・Wa no seishin}
\begin{table}[H]
\begin{center}
\begin{tabular}{c}
    \ruby{}{}\ruby{}{}\\
    ?\\
    onder de vorm van een vrij gevecht
\end{tabular}
\end{center}
\label{kyuu_1_wa_no_seishin}
\end{table}

\subsubsection{?・Verdedigingsstok}
\begin{table}[H]
\begin{center}
\begin{tabular}{c}
    \ruby{}{}\\
    tanto waza\\
    ?
\end{tabular}
\end{center}
\label{kyuu_1_defense_stick}
\end{table}

\subsubsection{乱取り・Randori}
\begin{table}[H]
\begin{center}
\begin{tabular}{c}
    \ruby{他人}{たにん}\ruby{数}{すう}の\ruby{乱取}{らんど}り\\
    tanins\={u} no randori\\
    vrij gevecht tegen verschillende andere personen
\end{tabular}
\end{center}
\label{kyuu_1_randori}
\end{table}
\begin{center}
    1 $\leftrightarrow$ 1..$\infty$
\end{center}


%%%%%%%%%%%%%%%%%%%%%%%%%%%%%%%%%%%%%%
% Dan techniques
%%%%%%%%%%%%%%%%%%%%%%%%%%%%%%%%%%%%%%
\newpage
\section{段の技・Dan technieken}

\newcommand{\suwaristart}{Suwari houding. Rechterknie naar voor, linkerknie links onder een hoek van 45 graden. Hand open op het rechterbeen leggen, net voor de knie en met de handpalm naar boven gericht.\\
Linkerhand houdt de saya vast, met de wijsvinger achter de tsuba vanonder (zodat hij er niet kan uitvallen) en de duim er boven op, maar net iets meer langs de binnenkant. Je moet opletten dat je, bij het trekken van het zwaard, niet in je duim snijdt. En je duim (in combinatie met je wijsvinger) kan een boost geven aan de tsuba, zodat het handvat losklikt uit de saya en eruit schiet. Daarmee win je iets extra aan tijd.\\
Rechterhand komt boven je handvat en schuift 1 handlengte richting tsuba.\\
Dan draait de rechterhand om. Dit zou de perfecte lengte moeten zijn, zodat je het zwaard ineens kunt vastpakken net onder de tsuba.\\}

\newcommand{\suwaristop}{Finale slag (en ondertussen ``toooo'' roepen). Dieper zakken.\\
Terug iets rechter kopen.\\
Zanshin.\\
Subari.\\
Not\-{o}.}

\newcommand{\pA}{De leraar}
\newcommand{\pB}{De leerling}
\newcommand{\pa}{de leraar}
\newcommand{\pb}{de leerling}

\subsection{Ken jutsu}
(In de 2 rollen.)

\subsubsection{Algemene informatie}

Diegene met het hart het dichtst bij de Shinzen, is de leeraar.\\

\subsubsection{Itsutsu no tachi}

Genoeg afstand (4 matten).\\
Groeten, met katana links.\\
Dichterbij komen tot op zwaardlengte afstand.\\
Zwaard trekken als je dicht genoeg bij elkaar bent, zodat het 1 vloeiend samenkomen is naar kamae houding voor beide personen.\\
Allebei 2 maki uchis.\\
\pA: Drijft op met zwaard schuin om de kromming te gebruiken om een opening te cre\"{e}ren.\\
Kleine pas, grote pas steken (niet overdreven).\\
\pB: Gaat achteruit mee. Kleine pas, grote pas, naar inno kamae. En onmiddellijk naar seigan no kamae om de steek af te weren in een snelle en beschermende beweging. Dat is de reden waarom \pa niet overdreven mag steken.\\
\pA: Zwaard weg trekken naar de linkerheup.\\
\pB: Opwapenen en aanvallen met maki uchi.\\
\pA: Rechts uitstappen en slag opvangen in te ure gasumi gasumi.\\
Allebei naar j\-{o}dan no kamae.\\
\pA: \-{o} gasumi.\\
\pB: Achteruit in inno. En dan terugkomen naar seigan no kamae (aanvallend naar de pols).\\
\pA: Zwaard wegtrekken naar de rechter-heup.\\
\pB: Ziet hoofd \pa vrij en doet een maki uchi aanval.\\
\pA: Naar links stappen en onmiddellijk maki uchi aanval naar hoofd \pb.\\
\pB: Naar links springen en opvangen k\-{o} gasumi.\\
Gevaarlijke positie: \pb kan aan \pa met een aanval buik openrijten, maar omgekeerd niet.\\
\pA: Gaan hiki achteruit en zit op linker-knie, met rechter-knie van voor.\\
\pB: Komt stap naar voor en valt maki uchi aan.\\
\pA: Tori, rechts uitstappend rechtkomen en do aanvallen.\\
\pB: Linkerheup achteruit smijten (en ook achteruit springen indien nodig), om do aanval op te vangen.\\
\pA: Onderdoor draaien, zwaard ondertussen draaien en aanvallen naar de keel van \pb. Let er op dat je de katana goed stevig in je heup vast hebt. Het is kracht tegen kracht, dus wegtrekken katana naar boven of naar onder, mag er niet voor zorgen dat je katana naar voor floept.\\
Nu is het kracht tegen kracht, voor allebei gevaarlijk. Allebei naar inno kamae.\\
Afwerken.

\subsubsection{Nanatsu no tachi}

Genoeg afstand (4 matten).\\
Groeten, met katana links.\\
Dichterbij komen tot op zwaardlengte afstand.\\
Zwaard trekken als je dicht genoeg bij elkaar bent, zodat het 1 vloeiend samenkomen is naar kamae houding voor beide personen.\\
Allebei naar sha no kamae.\\
\pA gaat naar j\-{o}dan en dan aanvallend naar seigan no kamae.\\
\pB volgt. Vanaf nu zal persoon B opdrijven.\\
Zwaard horizontaal draaien en opdrijven, met de punt naar de keel.\\
\pA leunt achteruit en zet een stap naar achter.\\
\pB slaat op buik-hoogte, tot net voorbij het lichaam van \pa.\\
\pA zijn heup gaat achteruit bij de stap naar achter en hij weert af (niet tegen het zwaard, zijn zwaard komt voor hem ter bescherming).\\
\pB drijft weer zoo op en hetzelfde gebeurt.\\
Nu heeft \pa er genoeg van en hij komt af op dezelfde manier.\\
De beweging eindigt wel op dezelfde manier: \pb zijn zwaard net voorbij het lichaam van \pa.\\
Dan drijft \pa voor een 2de keer op.\\
Nu eindigt het echter zwaarda tegen zwaard. Niet horizontaal, katana's onder een hoek met de scherpe kant meer naar beneden gericht.\\
\pA: kleine kote.\\
\pB: kleine kote.\\
\pA: grote kote.\\
\pB: grote kote.\\
\pA: \-{o}-gasumi.\\
\pB: naar inno en onmiddelijk seigan no kamae naar de pols.\\
\pA: Zwaard wegtrekken naar de linkerheup.\\
\pB: Do aanval.\\
\pA: Achteruit hiki.\\
Allebei opwapenen hoog k\-{o}-gasumi om dan hoog aan te vallen, zwaard tegen zwaard.\\
Dan allebei naar inno.\\
Afwerken.

\subsubsection{Kasumi no tachi}

Genoeg afstand (4 matten).\\
Groeten, met katana links.\\
Dichterbij komen tot op zwaardlengte afstand.\\
Zwaard trekken als je dicht genoeg bij elkaar bent, zodat het 1 vloeiend samenkomen is naar kamae houding voor beide personen.\\
Allebei naar j\_{o}dan no kamae, met linkervoet naar achter.\\
Seigan no kamae.\\
\pA: Kleine maki uchi, maar slaat onder het zwaard, naar de pols langs onder, dan naar boven (pols over). Dan met de mune het zwaard weg zweepslagen. Opwapenen en maki uchi.
\pB: Rechterpols wegtrekken om zwaard te laten passeren. Mee achteruit gaan. Soort van roofblock langs links en maki uchi aanvallen.\\
\pA: Hiki achteruit, zwaard naar linkerheup trekken.\\
\pB: Inkomen en maki uchi naar hoofd aanvallen.\\
\pA: Rooftop block langs links, rechts inkomen en maki uchi naar hoofd aanvallen.\\
\pB: Te ure gasumi opvangen.\\
\pA: Zwaard terugtrekken en steken naar de buik/lies.\\
\pB: Rechtervoet naar achter en lichaam achteruit, in jodan no kamae. Zo ga je ver genoeg weg om de steek niet in je lichaam te krijgen. Onmiddellijk zwaardslag op zwaard (do), om de aanval op te vangen en je te beschermen.\\
\pA: Maki uchi naar het hoofd.\\
\pB: Tori opvangen.\\
\pA: Naar do aanval rechterheup tegenstander overgaan.\\
\pB: Tori volgt de aanval, ondertussen je rechterheup achteruit smijten, ter ontwijking. Nu staat je punt in zijn centrum en je neemt een dreigende houding aan.\\
Allebei naar inno kamae.\\
\pB: Do aanval naar linkerheup tegenstander.\\
\pA: Opvangen en kleine men.\\
\pB: Do aanval naar linkerheup tegenstander.\\
\pA: Opvangen, \-{o}-kachi onderdoor en dan maki uchi erna.\\
\pB: Linkerheup achteruit smijten en zwaard naar linkerheup trekken om te ontwijken. Hiki achteruit tegelijkertijd, zal ook nodig zijn.\\
\pA: Men aanval.\\
\pB: Insteken langs linkerkant verdediging.\\
\pA: \-{o}-gasumi aanval. Is meer een steek naar het linkeroog van de tegenstander.\\
\pB: Zwaard als verdediging langs onder, terwijl je naar rechts onder een hoek van 45 graden uitstapt. Je staat nu in te ure gasumi.\\
\pB: Inschuiven, maar contact houden met het zwaard van de tegenstander langs onder. Als je dicht genoeg bent, kan je do aanvallen van de rechterkant uit, door je zwaard eronder uit te trekken, op te wapenen en aan te vallen.\\
\pA: Linkerheup achteruit smijten, hiki achteruit en tegelijk de slag opvangen. Afwerken met tak-tak-tak afwerking: slag erboven, naar links do en dan recht erop en op de knie zitten.\\
\pB: 2de aanval tak-tak-tak afwerking opvangen met je katana punt naar beneden, de kromming naar boven en ver genoeg van je lichaam. Je draait er ook een beetje naartoe, zodat je rechterheup ver genoeg naar achter is. Als je dan niet genoeg opvangt, is je heup tenminste ver genoeg naar achter en kan het zwaard misschien voor je buik passeren. Het nadeel is wel, dat je bij de 3de aanval van tak-tak-tak, niet meer snel genoeg bent om de finale slag op te vangen, dus je bent dood.

\subsubsection{Hakka no tachi}

Genoeg afstand (4 matten).\\
Groeten, met katana links.\\
Dichterbij komen tot op zwaardlengte afstand.\\
Zwaard trekken als je dicht genoeg bij elkaar bent, zodat het 1 vloeiend samenkomen is naar kamae houding voor beide personen.\\
Allebei achteruit. Dan terug naar mekaar lopen. Zwaarden maken contact in seigan.\\
\pA: Opdrijvend zwaard wegduwen met kromming en dan do aanval.\\
\pB: Hiki achteruit en afweren. Dan op zwaard slagen en inkomen.\\
\pA: Gaat achteruit. Heeft net slag op zwaard gekregen en zijn hoofd-regio is open.\\
\pB: Inkomen en grote men aanval.\\
\pA: Afweren links, met gekruiste voorarmen. Zwaard gaat naar beneden. Volgen en dan steken naar de lies.\\
\pB: Trekt ondertussen het zwaard even in en steekt ook.\\
\pA: Draait verder om terug in seigan te komen.\\
\pB: Staat ook in seigan.\\
Allebei een beetje afstand nemen en naar shin no kamae gaan.\\
\pA: Yokomen aanval links naar het hoofd.\\
\pB: Langs onder de pols oversnijden.\\
\pA: Hand lossen, zwaard laten passeren.\\
\pB: Slaat zwaard weg met een zweepslag.\\
\pA: Laat zwaard naar beneden zakken.\\
\pB: Maki uchi aanval. Naar het hoofd dat nu open is.\\
\pA: Afweren links, met gekruiste voorarmen. Zwaard gaat naar beneden. Volgen en dan steken naar de lies.\\
\pB: Trekt ondertussen het zwaard even in en steekt ook.\\
\pA: Draait verder om terug in seigan te komen.\\
\pB: Staat ook in seigan, maar maakt er een rooftop block beweging van, terwijl hij rechts instapt en een grote men aanval doet.\\
\pA: Tori verdedigen.\\
\pB: Yokomen rechts.\\
\pA: Tori-achtig verdedigen.\\
\pB: Do rechts.\\
\pA: Lage tori verdediging.\\
\pB: \-{O}-gasumi.\\
\pA: Achteruit, naar sh\-{o}dan no kamae. Slag recht op het puntje van het zwaard.\\
\pB: Trekt zwaard naar linkerheup, terwijl hij hiki naar achter gaat. Je handen zijn nog steeds gekruist, maar nu voor je gordel.\\
\pA: Inkomen en grote men aanval.\\
\pB: Uitstappen links en maki uchi aanval.\\
\pA: Afweren links, door in te steken met gekruiste voorarmen. Dan steek naar vooren.\\
\pB: Zwaard langs onder ter bescherming en uitstappen rechts. En yokomen rechts aanvallen.\\
\pA: Afweren langs binnenkant. En do rechts aanvallen.\\
\pB: Achteruit en afweren do aanval. Gevolgd door kleine men.\\
\pA: Achteruit gaan en nog eens kleine men. Dan terug inkomen en do links aanvallen.\\
\pB: Achteruit gaan en afweren.\\
Allebei achteruit gaan en 2 maki uchis doen.\\
\pA: Zwaard naar de linkerheup trekken en zo blijven wachten.\\
\pB: Eerst naar sh\-{o}dan no kamae. En dan inkomen om naar het hoofd aan te vallen.\\
\pA: Rooftop block uitstappen naar rechts en dan naar de nek aanvallen, terwijl je gaat zitten op je knie. ``Toooooo''

\subsection{Bo jutsu}
(In de 2 rollen.)

\subsubsection{Algemene informatie}

Examens hebben elke keer 2 nieuwe kata's. D.w.z.\ dat de tweede kata bij elk examen langs links eindigt, zodat je daarna gemakkelijk kan groeten.\\
Dus de 2de, de 4de en de 6de eindigen links.

\subsubsection{Seri ai no bo}

\pB: houding rechts /\\
\pB: aanval, uitstappen naar links (praktisch ni, ne voet is al te ver) + bo erop\\
\pB: wapenen + uistappen naar rechts + yokomen van rechts\\
\pB: wapen naar achter smijten en dan aanval langs onder (mateage)\\
\pB: zitten met linkerknie naar voor + opening laten\\
\pB: aanval komt + uitspringen naar links en opvangen\\
\pB: wapenen + yokomen van rechts, maar is schijnbeweging om dan te zitten en een slag naar de enkel te doen\\
\pA: ze vangen op, schuiven naar voor om je pols aan te vallen\\
\pB: snel springen naar rechts en bo op het wapen leggen\\
\pB: wapenen en yokomen van rechts\\
\pB: achteruit smijten\\
klaar

\subsubsection{Sune hishigi no bo}

Start tegen mekaar.
\pA: Sha no kamae. Maki uchi met voeten samen brengen en dan naar sh\-{o}dan gaan.\\
\pB: punt laten zakken en houding rechts /\\
\pA: links langs de bo gaan en maki uchi aanval\\\
\pB: uitstappen naar links (praktisch ni, ne voet is al te ver) + bo erop\\
\pB: zakken + mune links\\
\pA: rechterheup naar achter smijten en laag opvangen\\
\pB: wapenen + uistappen naar rechts + yokomen van rechts\\
\pA: opvangen hoog links\\
\pB: zakken + tsune rechts\\
\pA: linkerheup naar achter smijten en laag opvangen\\
\pB: wapenen + shomen (recht naar het hoofd)\\
\pA: kogasumi opvangen (links uitgestapt, punt naar rechts), gaat dan naar sh\-{o}dan no kamae\\
\pB: steken, door de arm-opening\\
\pA: vangt speciaal op, zwaardpunt langs links naar beneden draaien om zo de bo naar rechts kletsend af te weren\\
\pB: achteruit wapenen\\
\pA: achteruit rechterbeen en zwaard laten zakken naar rechterheup\\
\pB: yokomen rechts\\
\pA: afweren langs binnenkant en dan maki uchi aanval. ``Toooo''\\
\pB: achteruit smijten\\
\pA: maki uchi achteruit\\
\pB: bo links van u laten recht naar beneden gaan\\
klaar

\subsection{Naginata jutsu}
(In de 2 rollen.)

\subsubsection{Itsutsu no naginata}

In groet houding links. Scherpe kant naar boven.\\
Groeten.\\
Punt omhoog en omdraaien naginata (zodat de scherpe kant naar beneden gericht is).\\
Naar mekaar toelopen ondertussen, allebei in seigan.\\
\pA: 2 stappen naar voor\\
\pB: 2 stappen naar achter\\
\pA: stop\\
\pB: rechtervoet naar achter bijtrekken en ondertussen naginata recht zetten. Dan wisselen van handen en terug in seigan, maar dan seigan langs je rechter kant.\\
\pA: naar achteren, maki uchi\\
\pB: punt naar beneden, klaar om aan te vallen. Linkerhand links van je hoofd. Scherpe kromming naginata naar boven gericht.\\
\pA: aanval naar voor, maki uchi\\
\pB: afweren langs de binnenkant en dan onmiddellijk horizontaal buik proberen open te rijten\\
\pA: naar inno kamae\\
\pB: naginata terug rechts van je recht zetten. Rechterhand boven, maar je wisselt ze weer om. Linkerhand is nu boven, ter hoogte van je slaap, rechterhand lager\\
\pB: aanval tsune (laag) langs de rechterkant, de naginata is bij de aanval nu links van je\\
\pA: afweren laag\\
\pB: yokomen rechts\\
\pA: hoog afweren\\
\pB: mateage\\
\pA: afweren, laag, met steekhouding om het tegen te houden en te glijden over de naginata. Daarna kleine men aanvallen\\
\pB: naginata in mateage houding dichter bij je lichaam trekken en je rechtervoet naar achteren bijtrekken. Er staat geen gewicht op, want je gaat zo metten ronddraaien.\\
\pA: j\-{o}dan no kamae\\
\pB: ronddraaien naar links achterom\\
\pA: horizontale aanval\\
\pB: na omdraaien onmiddellijk horizontaal naar links toe, de buik openrijten\\
\pA: voet naar achter intrekken en heup naar achter, om buik te besparen\\
\pB: naginata onmiddellijk op de katana leggen ter bescherming (met de kromming naar links)\\
\pA: het geheel naar boven draaien, rechtsom, om je katana vrijheid te geven. Doordraaien naar beneden. Dan een maki uchi aanval doen.\\
\pB: naginata draait naar beneden, je zet wat draaing bij en met de achterkant schwoeng je langs boven de katana naar beneden. Je eindigt met de naginata rechtop, links van u.\\
\pA: j\-{o}dan no kamae\\
\pB: Aanval naar buik\\
\pA: horizontale aanval\\ 
\pB: opwapenen\\
\pA: j\-{o}dan no kamae\\
\pB: Aanval naar buik\\
\pA: horizontale aanval\\ 
\pB: naginata op de katana leggen ter bescherming (met de kromming naar links)\\
\pA: het geheel naar boven draaien, rechtsom, om je katana vrijheid te geven. Doordraaien naar beneden. Dan een maki uchi aanval doen.\\
\pB: naginata draait naar beneden, je zet wat draaing bij en met de achterkant schwoeng je langs boven de katana naar beneden. Je eindigt met de naginata rechtop, links van u.\\
\pA: Inno kamae, even pauze en dan yokomen van rechts\\
\pB: weer soort van buik open rijten aanval, maar onmiddelijk terugkomen en de naginata weer op de katana leggen. Maar met zwepende kracht deze keer, want je maakt plaats voor de finale slag. Inkomen met scherpe kant naar de nek en op je knie zitten. ``Toooo''\\
\pA: zwaard naar beneden en achteruit, wachten, tik langs rechterkant (binnenkant) naginata

TBD

\subsection{Iai jutsu Katori}

\subsubsection{Algemene informatie}

Ceremonie?\\

Zwaard wegsteken:
\begin{itemize}
\item[--] over de linker-schouder
\item[--] naar beneden gericht, totdat hij in de saya kan
\item[--] dan naar boven richten (verticaal)
\item[--] in de saya steken, de zwaartekracht helpt door het verticale aspect
\end{itemize}


\subsubsection{Kusa nagi no ken}

\suwaristart
Zwaard trekken, horizontaal. Je kapt naar de tegenstander zijn enkel of onderbeen.\\
Je hebt ondertussen je rechtervoet plat op de grond gezet, je knie net rechts van je centrum.\\
Een aanval kan terugkomen naar je rechterbeen, omdat dit van voor staat. Je reageert daarop, door je rechterbeen naar links te brengen om een zwaardslag te ontwijken. Ondertussen zal je ook opwapenen over je linkerschouder. Daarna zet je het rechterbeen terug, terwijl je een aanval naar beneden doet. Je zakt hierbij lichtjes door je benen, om mooi te flow van het zwaard te volgen.\\
Tori verdedigen.\\
Linkerhandpalm naar boven gericht, ter hoogte van het voorhoofd. Je linkerduim is lichtjes gebogen. Dit vormt een plooi in je duim, waar je de mune van het zwaard in kan leggen, zodat het niet naar opzij kan schuiven.\\
Je rechterhand heeft het handvat vast en is naar voren gericht.\\
Zwaard in de plooi van je duim naar boven schuiven om op te wapenen (en ondertussen ``yeeeep'' roepen).\\
\suwaristop

\subsubsection{Nuki tsuke no ken}

\suwaristart
Zwaard trekken, horizontaal. Je kapt naar de tegenstander zijn enkel of onderbeen.\\
Linkerknie naar voren, handpalm onder mune van het zwaard en je brengt het geheel horizontaal op schouderhoogte, met de scherpe kant naar boven en de punt naar de tegenstander gericht.\\
De mune ligt op je hand. Die hand vormt een soort van kuiltje in de lengte van je handpalm. Hierin kan het zwaard naar voren schuiven, zonder dat het weg kan schuiven naar de zijkanten.\\
Naar voren schuiven en ondertussen prikken. Het zwaard schuift naar voren door het kuiltje, je hand blijft op dezelfde plaats. Arm is gestrekt en je prikt naar de tegenstander zijn keel.\\
Dan onmiddelijk terugtrekken van het zwaard.\\
Sprong en wisselen benen.\\
Wapenen (ondertussen ``yeeeep'' roepen).\\
\suwaristop

\subsubsection{Nuki uchi no ken}

\suwaristart
Springen, zo hoog als je kan.\\
Benen optrekken. Je moet een slag naar je benen ontwijken, door erover te spingen.\\
Ondertussen je zwaard trekken en wapenen. Dit gebeurt allemaal in de lucht.\\
Neerkomen door je benen terug te strekken, zodat je eerst met je voeten op de grond land. En dan zakken en op je linkerknie gaan zitten.\\
Dit gaat gepaard met een finale slag. ``Yeeeep'' roepen.\\
\suwaristop

\subsubsection{Uken}

\suwaristart
TBD

\subsubsection{Saken}

\suwaristart
TBD

\subsubsection{Happoken}

TBD

\subsection{Extra}

\subsubsection{Nomenclatuur Katana onderdelen}

TBD

\subsubsection{Geschiedenis van de school}

TBD

\subsubsection{Noties over de geschiedenis van Japan en krijgskunst}

TBD


\newpage
\subsection{2de dan}
\subsubsection{Algemeen}
\begin{table}[H]
\begin{center}
\begin{tabular}{c}
    \ruby{第二段}{だいにだん}\\
    daini dan\\
    \tran{2de dan}\\
    \hline
    \ruby{追加}{ついか}\\
    tsuika\\
    \tran{aanvullend}
\end{tabular}
\end{center}
\label{dan_2_gen}
\end{table}

\subsubsection{基本投げ技・Kihon nage waza}
\noindent Werptechnieken onder de vorm van een kata
\\
\begin{table}[H]
\begin{center}
\scriptsize
\begin{tabular}{BB}
    \ruby{}{} & \ruby{裏横面打}{うらよこめんう}ち\\
    ude garami & ura yoko men uchi\\
    ? & \tran{(met) achterkant (vuist) (naar) zijkant van het gezicht slagen}\\
    \\
    \ruby{前}{まえ}\ruby{}{} & \ruby{}{}\\
    mae hiki otoshi & shomen men uchi\\
    ? & ?\\
    \\
    \ruby{手}{て}?\ruby{返}{がえ}し& \ruby{突}{つき}\ruby{上段}{じょうだん}\\
    te uchi mata gaeshi & tsuki j\={o}dan\\
    ? & \tran{hoge stoot}\\
    \\
    \ruby{}{}\ruby{前}{まえ} & \ruby{突}{つき}\ruby{中}{ちゅう}\ruby{段}{だん}\\
    ude kake mae hiki otoshi & tsuki ch\={u}dan\\
    ? & \tran{stoot naar het midden}\\
    \\
    \ruby{裏}{うら}\ruby{向}{むか}え\ruby{倒}{だお}し & \ruby{前}{まえ}\ruby{蹴}{け}り\\
    ura mukae daoshi & mae keri\\
    ? & \tran{voorwaartse trap}\\
    \\
    \ruby{}{} & \ruby{}{}\\
    gyaku kote gaeshi & junte dori + (tsuki j\={o}dan)\\
    ? & ?\\
    \\
    \ruby{}{} & \ruby{前}{まえ}\ruby{}{}\\
    ashi tori oshi taoshi & mae eri\\
    ? & ?
\end{tabular}
\end{center}
\label{kihonnagewaza}
\end{table}

\subsubsection{追加の技・Aanvullende technieken}
\begin{table}[H]
\begin{center}
\begin{tabular}{BB}
    ? & ?\\
    ude garami & ura kataha\\
    ? & ?\\
    \hline\\
    ? & ?\\
    mae hiki otoshi & hiji gaeshi\\
    ? & ?\\
    \hline\\
    ? & ?\\
    ushiro hiki otoshi & mae hiji kudaki\\
    ? & ?\\
    \hline\\
    ? & \ruby{肩}{かた}はおとし\\
    te uchi mata gaeshi & kata ha otoshi\\
    ? & \tran{schouder verliezen}\\
    \hline\\
    ? & ?\\
    gyaku te uchi mata gaeshi & mae tobu nage\\
    ? & ?\\
    \hline\\
    ? & ?\\
    ude kake mae hiki otoshi & ura mae tobu nage\\
    ? & ?\\
    \hline\\
    ? & ?\\
    gyaku kote gaeshi & do gaeshi\\
    ? & ?\\
    \hline\\
    \ruby{足}{あし}\ruby{取}{と}り\ruby{押}{お}し\ruby{倒}{たお}し & ?\\
    ashitori oshi taoshi & gyaku do gaeshi\\
    \tran{tegenstander neerhalen door zijn been vast te pakken en neerhalen door frontaal duwen} & ?
\end{tabular}
\end{center}
\label{dan_2_gen}
\end{table}

\subsubsection{Han sutemi (Kihon)}
\begin{table}[H]
\begin{center}
\begin{tabular}{ll}
    \ruby{}{} & \ruby{}{}\\
    kubi otoshi sutemi & tchoku tsuki\\
    ? & ?\\
    \\
    \ruby{}{} & \ruby{}{}\\
    hazu oshi sutemi & tchoku tsuki\\
    ? & ?\\
    \\
    \ruby{}{} & \ruby{}{}\\
    harite sutemi & tchoku tsuki\\
    ? & ?
\end{tabular}
\end{center}
\label{dan_2_bukidori_tanto}
\end{table}

\subsubsection{Buki dori}
Tanto dori
\begin{table}[H]
\begin{center}
\begin{tabular}{lll}
    \ruby{}{} & \ruby{}{} & (\ruby{}{} - \ruby{}{})\\
    kote gaeshi & tsuki ch\={u}dan & (nage - osae)\\
    ? & ? & ?\\
    \\
    \ruby{}{} & \ruby{}{} & (\ruby{}{} - \ruby{}{})\\
    kataha otoshi & tsuki ch\={u}dan & (nage - osae)\\
    ? & ? & ?\\
    \\
    \ruby{}{} & \ruby{}{} & (\ruby{}{} - \ruby{}{})\\
    shih\={o} nage & omote yoko men uchi & (nage - osae)\\ 
    ? & ? & ?
\end{tabular}
\end{center}
\label{dan_2_bukidori_tanto}
\end{table}

Hanbo dori
\begin{table}[H]
\begin{center}
\begin{tabular}{lll}
    \ruby{}{} & \ruby{}{} & (\ruby{}{} - \ruby{}{})\\
    kote gaeshi & tsuki ch\={u}dan & \\
    ? & ? & ?\\
    \\
    \ruby{}{} & \ruby{}{} & (\ruby{}{} - \ruby{}{})\\
    tenbin nage & tsuki ch\={u}dan & \\
    ? & ? & ?\\
    \\
    \ruby{}{} & \ruby{}{} & (\ruby{}{} - \ruby{}{})\\
    shih\={o} nage & tsuki ch\={u}dan & \\ 
    ? & ? & ?
\end{tabular}
\end{center}
\label{dan_2_bukidori_hanbo}
\end{table}

\subsubsection{Wa no seishin}

\subsubsection{型・Kata}
\begin{table}[H]
\begin{center}
\begin{tabular}{lcc}
    Zonder wapen: & test & test \\
    \hline
    Met wapen: & test & test
\end{tabular}
\end{center}
\label{kata_dan_2}
\end{table}

\subsubsection{Randori}
\begin{table}[H]
\begin{center}
\begin{tabular}{rl}
    ? & 1 tegen 1 - toepassen van diverse technieken \\
    j\={u} no randori & en ma / chika ma - (soepel) \\
    ? & \\
    \hline
    ? & 1 tegen 3 - Ontwijkingen/kanalisaties \\
    randori tai sabaki & \\
    ? & \\
    \hline
    ? &  \\
    randori wa no sei shin & \\
    ? & \\
    \hline
    ? &  \\
    taninzu dori randori & Realistische, verdediging en toepassing van technieken tegen meerdere partners\\
    ? &
\end{tabular}
\end{center}
\label{randori_dan_2}
\end{table}


\subsubsection{Extra informatie}


\newpage
\subsection{3de dan}
\subsubsection{Algemeen}
\begin{table}[H]
\begin{center}
\begin{tabular}{c}
3de dan\\
\ruby{第三段}{だいさんだん}\\
daisan dan
\end{tabular}
\end{center}
\label{dan_3_gen}
\end{table}

\subsubsection{Ude waza}

\subsubsection{Ashi waza}

\subsubsection{Hikitate waza}

\subsubsection{Shime waza}

\subsubsection{Sutemi waza}

\subsubsection{Kaeshi waza}

\subsubsection{Buko dori}

\subsubsection{型・Kata}
\begin{table}[H]
\begin{center}
\begin{tabular}{lcc}
    Zonder wapen: & test & test \\
    \hline
    Met wapen: & test & test
\end{tabular}
\end{center}
\label{kata_dan_3}
\end{table}

\subsubsection{Randori}

\subsubsection{Extra informatie}


\newpage
\subsection{4de dan}
\subsubsection{Algemeen}
\begin{table}[H]
\begin{center}
\begin{tabular}{c}
4th dan\\
\ruby{第四段}{だいよんだん}\\
daiyon dan
\end{tabular}
\end{center}
\label{dan_4_gen}
\end{table}

\subsubsection{追加の技 Aanvullende technieken}


\newpage
\subsection{5de en 6de dan}
\subsubsection{Algemeen}
\begin{table}[H]
\begin{center}
\begin{tabular}{c}
5de en 6de dan\\
\ruby{第}{だい}\ruby{五と六}{ごとろく}の\ruby{段}{だん}\\
daigo to roku no dan
\end{tabular}
\end{center}
\label{dan_5and6_gen}
\end{table}

\subsubsection{追加の技 Aanvullende technieken}


%%%%%%%%%%%%%%%%%%%%%%%%%%%%%%%%%%%%%%
% Omote waza
%%%%%%%%%%%%%%%%%%%%%%%%%%%%%%%%%%%%%%
\label{omotewaza}
\section{表技・Omote waza}

\newpage
\begin{landscape}
\thispagestyle{empty} %should remove the page number
\begin{center}
    \textbf{基本投げ技・Kihon nage waza (7 technieken)}
\end{center}
\addcontentsline{toc}{subsection}{\protect 基本投げ技・Kihon nage waza (7 technieken)}
\label{kihonnagewaza}
\begin{table}[H]
\begin{center}
\scriptsize
\begin{tabular}{ccc}
\multicolumn{3}{c}{pause tussen laten en afstand - lijn van het lichaam}\\
\multicolumn{3}{c}{間合と間 - 体の線}\\
\multicolumn{3}{c}{ma ai to ma - tai no sen}\\
\\
\ruby{裏横面打}{うらよこめんう}ち & \ruby{向}{むか}え\ruby{倒}{だお}し & \ruby{後受身}{うしろうけみ}\\
ura yoko men uchi & mukae daoshi & ushiro ukemi\\
(met) achterkant (vuist) (naar) zijkant van het gezicht slagen & naartoe gaan
en neerhalen & achterwaartse val\\
\\
\ruby{表横面打}{おもてよこめんう}ち & \ruby{四方}{しほう}\ruby{投}{な}げ &
\ruby{後}{うしろ}\ruby{受身}{うけみ}\\
omote yoko men uchi & shih\={o} nage & ushiro ukemi\\
(met) voorkant (vuist) (naar) zijkant van het gezicht slagen & worp in elke richting & achterwaartse val\\
\\
\ruby{突}{つき}\ruby{上段}{じょうだん} & \ruby{行}{ゆ}き\ruby{違}{ちが}え & \ruby{後受身}{うしろうけみ}\\
tsuki j\={o}dan & yuki chigae & ushiro ukemi\\
hoge stoot & elkaar kruisen & achterwaartse val\\
\\
\ruby{突}{つき}\ruby{中}{ちゅう}\ruby{段}{だん} &
\ruby{捻}{ねじ}\ruby{小手}{こて}\ruby{返}{がえ}し &
\ruby{前}{まえ}\ruby{受身}{うけみ}\\
tsuki ch\={u}dan & neji kote gaeshi & mae ukemi\\
stoot naar het midden & veer/schroef van horloge onderarm omkering & voorwaartse val\\
\\
\ruby{両}{りょう}\ruby{袖}{そで}\ruby{取}{ど}り & \ruby{天秤投}{てんびんな}げ & \ruby{前}{まえ}\ruby{受身}{うけみ}\\
ry\={o} sode dori & tenbin nage & mae ukemi\\
beide mouwen vastpakken & weegschaal worp & voorwaartse val\\
\\
\ruby{土}{ど}\ruby{足}{そく}\ruby{手}{て}\ruby{取}{ど}り &
\ruby{鉢}{はち}\ruby{廻}{まわ}し& \ruby{後}{うしろ}\ruby{受身}{うけみ}\\
do soku te dori & hachi mawashi & ushiro ukemi\\
shoenen hand actieve partner & hersenpan roteren & achterwaartse val\\
\\
\ruby{両}{りょう}\ruby{手}{て}\ruby{取}{ど}り &
\ruby{腰}{こし}\ruby{投}{な}げ & \ruby{前}{まえ}\ruby{受身}{うけみ}\\ 
ry\={o} te dori & koshi nage & mae ukemi \\
beide handen vastpakken & heup worp & voorwaartse val\\
\end{tabular}
\end{center}
\label{kihonnagewaza}
\end{table}

\end{landscape}

\newpage
\thispagestyle{empty} %should remove the page number
\begin{landscape}
\begin{center}
    \textbf{基本押え技・Kihon osae waza (6 technieken)}
\end{center}
\addcontentsline{toc}{subsection}{\protect 基本押え技・Kihon osae waza (6 technieken)}
\label{kihonosaewaza}
\begin{table}[H]
\begin{center}
\small
\begin{tabular}{LLL}
\multicolumn{3}{c}{間合と間 - 体の線}\\
\multicolumn{3}{c}{ma ai to ma - tai no sen}\\
\multicolumn{3}{c}{\tran{pause tussen laten en afstand - lijn van het lichaam}}\\
\ruby{突}{つき}\ruby{中}{ちゅう}\ruby{段}{だん} & \ruby{後}{うしろ}\ruby{捻}{ねじ}\ruby{砕}{くだ}き & \ruby{肩}{かた}\ruby{関}{かん}\ruby{節}{せつ}\\
tsuki ch\={u}dan & ushiro neji kudaki & kata kansetsu\\
\tran{stoot op middenhoogte} & \tran{achterwaarts draaien breken} & \tran{schouder gewricht}\\
\ruby{両}{りょう}\ruby{手}{て}\ruby{一方}{いっぽう}\ruby{取}{ど}り &
\ruby{呂}{ろ}\ruby{伏}{ふ}せ & \ruby{肩}{かた}\ruby{関}{かん}\ruby{節}{せつ}\\
ry\={o} te ipp\={o} dori & rofuse & kata kansetsu\\
\tran{beide handen 1 kant vastpakken} & \tran{(rug)wervels naar beneden buigen} & \tran{schouder gewricht}\\
\ruby{片}{かた}\ruby{袖}{そで}\ruby{取}{ど}り & \ruby{小手}{こて}\ruby{砕}{くだ}き & \ruby{肩}{かた}\ruby{関}{かん}\ruby{節}{せつ}\\
kata sode dori & kote kudaki & kata kansetsu\\
\tran{schouder mouw vastpakken} & \tran{voorarm breken} & \tran{schouder gewricht}\\
\\
\multicolumn{3}{c}{間合と近間 - 体の線}\\
\multicolumn{3}{c}{ma ai to chikama - tai no sen}\\
\multicolumn{3}{c}{\tran{pause tussen laten en dichtbij - lijn van het lichaam}}\\
\ruby{後}{うしろ}\ruby{片}{かた}\ruby{取}{ど}り & えり\ruby{鰤}{し}め\ruby{行}{ゆ}き\ruby{違}{ちが}え & \ruby{腕}{うで}\ruby{関}{かん}\ruby{節}{せつ}\\
ushiro kata dori & eri shime yuki chigae & ude kansetsu\\
\tran{achterwaarts schouder vastpakken} & \tran{kraag vastklemmen elkaar kruisen} & \tran{arm gewricht}\\
\ruby{後}{うしろ}\ruby{両}{りょう}\ruby{手}{て}\ruby{取}{ど}り &
\ruby{四方}{しほう}\ruby{投}{な}げ& \ruby{肩}{かた}\ruby{関}{かん}\ruby{節}{せつ}\\
ushiro ry\={o} te dori& shih\={o} nage & kata kansetsu\\
\tran{achterwaarts beide handen vastpakken} & \tran{worp in elke richting} & \tran{schouder gewricht}\\
\ruby{後}{うしろ}\ruby{上手}{うわて}\ruby{取}{ど}り &
\ruby{向}{む}え\ruby{倒}{だお}し& \ruby{肩}{かた}\ruby{関}{かん}\ruby{節}{せつ}\\
ushiro uwate dori & mukae daoshi & kata kansetsu\\
\tran{achterwaarts bovenste deel (van het lichaam) vastpakken} & \tran{naartoe gaan en neerhalen} & \tran{schouder gewricht}\\
\end{tabular}
\end{center}
\label{kihonosaewaza}
\end{table}

\end{landscape}

\newpage
\begin{center}
    \textbf{雄用技投げ技・Oy\={o} waza nage waza (間 - 近間)}
\end{center}
\addcontentsline{toc}{subsection}{雄用技投げ技・Oy\={o} waza nage waza (間 - 近間)}
\begin{table}[H]
\begin{center}
\begin{tabular}{c}
technieken voor mannen werp technieken(ruimte laten - dichtbij)\\
雄用技投げ技(間 - 近間)\\
oy\={o} waza nage waza (ma - chikama)\\
\end{tabular}
\end{center}
\label{oyouwazanagewaza}
\end{table}


\begin{center}
    \textbf{雄用技押え技・Oy\={o} waza osae waza (間 - 近間)}
\end{center}
\addcontentsline{toc}{subsection}{雄用技押え技・Oy\={o} waza osae waza (間 - 近間)}
\begin{table}[H]
\begin{center}
\begin{tabular}{c}
technieken voor mannen werp technieken(ruimte laten - dichtbij)\\
\ruby{雄用}{およう}\ruby{技}{わざ}\ruby{押}{おさ}え\ruby{技}{わざ}(\ruby{間}{ま} - \ruby{近間}{ちかま})\\
oy\={o} waza osae waza (ma - chikama)\\
\end{tabular}
\end{center}
\label{oyouwazaosaewaza}
\end{table}


\begin{center}
    \textbf{和の精神・Wa no sei shin}
\end{center}
\addcontentsline{toc}{subsection}{和の精神・Wa no sei shin}
\begin{table}[H]
\begin{center}
\begin{tabular}{c}
geest in harmonie\\
\ruby{和}{わ}の\ruby{精神}{せいしん}\\
wa no sei shin\\
\hline
voorwaarts en achterwaarts - lijn van het lichaam\\
前 と 後 - 体の線\\
mae to ushiro - tai no sen
\end{tabular}
\end{center}
\label{wanoseishin}
\end{table}


\begin{center}
    \textbf{乱取り・Randori}
\end{center}
\addcontentsline{toc}{subsection}{乱取り・Randori}
\begin{table}[H]
\begin{center}
\begin{tabular}{c}
vrij gevecht\\
乱取り[らんどり]\\
randori\\
\hline
zacht gevecht (ruimte laten - lijn van het lichaam)\\
\ruby{柔}{じゅう}の\ruby{乱取}{らんど}り (\ruby{間}{ま} - \ruby{体}{たい}の\ruby{線}{せん}) \\
j\={u} no randori (ma - tai no sen)
\end{tabular}
\end{center}
\label{randori}
\end{table}


\begin{center}
    \textbf{形大東流 - ?・Kata dait\={o} ry\={u} - ikajo}
\end{center}
\addcontentsline{toc}{subsection}{形大東流 - ?・Kata dait\={o} ry\={u} - ikajo}
\input{daitoryuu}

%%%%%%%%%%%%%%%%%%%%%%%%%%%%%%%%%%%%%%
% Katori
%%%%%%%%%%%%%%%%%%%%%%%%%%%%%%%%%%%%%%
\newpage
\section{Tenshin Sh\={o}den Katori Shint\={o} Ry\={u}}
\subsection{Algemeen}
\begin{table}[H]
\begin{center}
\begin{tabular}{c}
de weg van de hemel gewijdt aan de positieve legende van het Katori altaar\\
%TODO: see if the katori furigana are with the correct kanji!
\ruby{天真}{てんしん}\ruby{正傳}{しょうでん}\ruby{香}{かとり}\ruby{取神}{しんとう}\ruby{道流}{りゅう}\\
tenshin sh\={o}den katori shint\={o} ry\={u}
\end{tabular}
\end{center}
\label{katori}
\end{table}


\subsection{6de ky\={u}}
\subsubsection{構え・kamae}
\begin{table}[H]
\begin{center}
\begin{tabular}{c}
    bovenste/hoge houding\\
    \ruby{上段}{じょうだん}の\ruby{構}{かま}え\\
    j\={o}dan no kamae\\
    \hline
    houding met de katana gericht naar de ogen van de tegenstander\\
    \ruby{青眼}{せいがん}の\ruby{構}{かま}え\\
    seigan no kamae\\
    \hline
    boog onder een hoek\\
    \ruby{弧}{こ}がすみ\\
    ko gasumi\\
    \hline
    hand naar achteren onder een hoek\\
    \ruby{手裏}{てうらあ}がすみ\\
    te ura gasumi\\
    \hline
    rechter lagere houding\\
    \ruby{右下段}{みぎげだん}の\ruby{構}{かま}え\\
    migi gedan no kamae\\
    \hline
    linker lagere houding (omgekeerde lagere)\\
    \ruby{左下段}{ひだりげだん}の\ruby{構}{おかま}え\\
    hidari gedan no kamae (gyaku gedan)\\
    \hline
    schaduw houding\\
    \ruby{陰}{いん}の\ruby{構}{かま}え\\
    in no kamae\\
    \hline
    schuine houding\\
    \ruby{斜}{しゃ}の\ruby{構}{かま}え\\
    sha no kamae\\
    \hline
    rechter hoge houding\\
    \ruby{右上段}{みぎじょうだん}の\ruby{構}{かま}え\\
    migi j\={o}dan no kamae\\
    \hline
    linker hoge houding\\
    \ruby{左上段}{ひだりじょうだん}の\ruby{構}{かま}え\\
    hidari j\={o}dan no kamae\\
    \hline
    ? houding\\
    ? [しんのかまえ]\\
    shin no kamae\\
    \hline
    ? onder een hoek\\
    ? [おがすみ]\\
    o gasumi\\
    \hline
    lager zitten (in houding)\\
    \ruby{座}{すわ}り\ruby{下段}{げだん}(の\ruby{構}{かま}え)\\
    suwari gedan (no kamae)\\
    \hline
    vogel(stand)\\
    \ruby{鳥}{とり}\\
    tori
\end{tabular}
\end{center}
\label{kyuu_6_katori_kamae}
\end{table}

\subsubsection{技・Technieken}
\noindent Alleen en in kumitachi.
\begin{table}[H]
\begin{center}
\begin{tabular}{c}
    uitrol van een winding\\
    \ruby{巻き打}{まきう}ち\\
    maki uchi\\
    \hline
    uitrol zijkant van het gezicht\\
    \ruby{横面打}{よこめんう}ち\\
    yokomen uchi\\
    \hline
    uitrol naar de grond (volledige snijbeweging van boven naar onder)\\
    \ruby{土打}{どう}ち\\
    do uchi\\
    \hline
    bodem ontvangen\\
    \ruby{土受}{どう}け\\
    do uke\\
    \hline
    punt katana laten zakken en katana wegsteken\\
    ?[?]\\
    noto
\end{tabular}
\end{center}
\label{kyuu_6_katori_other}
\end{table}

\subsection{5de ky\={u}}
\subsubsection{kata}
\begin{table}[H]
\begin{center}
\begin{tabular}{c}
    ?\\
    ?\\
    ken no kata\\
    \hline
    ?\\
    ?\\
    bo no kata\\
    \hline
    ?\\
    ?\\
    kusanagi no ken (iai goshi)
\end{tabular}
\end{center}
\label{kyuu_5_katori_kata}
\end{table}

\subsubsection{bo jutsu}
\noindent Alleen en in kumitachi.
\begin{table}[H]
\begin{center}
\begin{tabular}{c}
    ?\\
    ?\\
    yokomen uchi\\
    \hline
    ?\\
    ?\\
    do uchi\\
    \hline
    ?\\
    ?\\
    sune uchi
\end{tabular}
\end{center}
\label{kyuu_5_katori_bo}
\end{table}

\subsection{4de ky\={u}}
\subsubsection{kata}
\begin{table}[H]
\begin{center}
\begin{tabular}{c}
    ?\\
    ?\\
    itsutsu no tachi (kiri komi)\\
    \hline
    ?\\
    ?\\
    nuki tsuke no ken (iai goshi)
\end{tabular}
\end{center}
\label{kyuu_4_katori_kata}
\end{table}

\subsection{3de ky\={u}}
\subsubsection{kata}
\begin{table}[H]
\begin{center}
\begin{tabular}{c}
    ?\\
    ?\\
    nanatsu no tachi (kiri komi)\\
    \hline
    ?\\
    ?\\
    nuki uchi no ken (iai goshi)\\
    \hline
    ?\\
    ?\\
    seri ai no bo (uchi komi)
\end{tabular}
\end{center}
\label{kyuu_3_katori_kata}
\end{table}

\subsection{2de ky\={u}}
\subsubsection{kata}
\begin{table}[H]
\begin{center}
\begin{tabular}{c}
    ?\\
    ?\\
    kasumi no tachi (kiri komi)\\
    \hline
    ?\\
    ?\\
    sune hishigi no bo (uchi komi)\\
    \hline
    ?\\
    ?\\
    itsutsu no naginata (kiri komi)\\
    \hline
    ?\\
    ?\\
    uken (iai goshi)
\end{tabular}
\end{center}
\label{kyuu_2_katori_kata}
\end{table}

\subsection{1ste ky\={u}}
\subsubsection{kata}
\begin{table}[H]
\begin{center}
\begin{tabular}{c}
    ?\\
    ?\\
    haka no tachi (kiri komi)\\
    \hline
    ?\\
    ?\\
    saken (iai goshi)\\
    \hline
    ?\\
    ?\\
    happo ken (iai goshi)
\end{tabular}
\end{center}
\label{kyuu_1_katori_kata}
\end{table}

\subsection{第一段・1ste dan}
\subsubsection{kata}
\noindent Een perfecte kennis is vereist van alle voorgaande technieken.\\

\subsubsection{extra}
\begin{itemize}
    \item Houding van het lichaam en perfectie van de posities: shisei
    \item Uitgesproken expressie van kime, zanshin en kiai
    \item Kennis van de kobudo termen en de verschillende onderdelen waaruit een katana is opgebouwd
\end{itemize}

\end{CJK*}
\end{document}

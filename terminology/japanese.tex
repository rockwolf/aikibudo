\noindent Alvorens we de berippen bekijken, dient een korte uitleg gegeven te worden over de Japanse taal.\\

\noindent Japans is opgebouwd uit 3 grote onderdelen: \textit{hiragana, katakana} en \textit{kanji}.\\

\noindent Hiragana en katakana zijn allebei een \textit{phonetisch} alfabet. Dat wil zeggen dat het symbolen voorstelt, die een klank uitbeelden.
Japans heeft dus geen aparte letters, zoals de meeste Westerse alfabetten.\\

\noindent \textbf{Hiragana} wordt gebruikt om zinnen te vormen en grammatica toe te passen op de Japanse taal. Het toont hoe Japans wordt uitgesproken.\\

\noindent \textbf{Katakana} bevat dezelfde klanken en enkele extra klanken. Het is ontstaan om de verschillende buitenlandse termen te kunnen uitspreken. Zo worden bvb. namen van niet Japanners in katakana geschreven.\\

\noindent \textbf{Kanji} zijn de meer complexe Japanse symbolen, die woorden en begrippen voorstellen. Ook Japanse namen en plaatsnamen worden meestal met kanji geschreven. Om te weten hoe deze symbolen worden uitgesproken, wordt gebruik gemaakt van hiragana, omdat hiragana een phonetisch alfabet is dat klanken in beeld brengt.\\

\noindent \textbf{Romaji} is een westerse schrijfwijze van Japanse klanken. Het wordt in Japan ook gebruikt om op een computer te typen.
Het is deze schrijfwijze, die het mogelijk maakt voor mensen die geen Japans kennen, om Japanse termen correct uit te spreken.

\noindent \textbf{Furigana} zijn kleine symbolen boven de kanji die in hiragana weergeven hoe de kanji moet worden uitgesproken.

\noindent In de tabel hieronder, kan je een overzicht vinden van de verschillende alfabetten, ter verduidelijking van bovenstaande uitleg.\\

\begin{table}[H]
\begin{center}
\begin{tabular}{c|c|c|c}
Nederlands alfabet & Hiragana & Katakana & Romaji\\
\hline
a &  あ & ア & a\\
k & bestaat niet & bestaat niet & bestaat niet\\
bestaat niet & か & カ & ka
\end{tabular}
\end{center}
\caption{Een kleine vergelijking ter verduidelijking}
\label{vergelijking_alfabetten}
\end{table}

\noindent De begrippen in dit document, zijn op volgende manier weergegeven:\\

\textit{Nederlandse omschrijving} $|$ \textit{Japanse term [met hiragana uitspraak boven de kanji]} $|$ \textit{romaji uitspraak}\\

\noindent De technieken zijn echter te lang om op die manier weer te geven. Zij staan dus onder elkaar:\\

\begin{center}
\textit{Nederlandse term}\\
\textit{Japanse term [met hiragana uitspraak boven de kanji]}\\
\textit{Romaji uitspraak}
\end{center}

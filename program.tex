\documentclass[a4paper,12pt]{article}
\usepackage[dutch]{babel}
\usepackage[T1]{CJKutf8}
\usepackage[CJK, overlap]{ruby} %furigana support
%\usepackage{multirow}
\usepackage{pdflscape}
\usepackage[left=2cm,top=1cm,right=2cm,bottom=2cm, nohead,nofoot]{geometry}

\renewcommand{\rubysize}{0.5}
\renewcommand{\rubysep}{-0.2ex}
\usepackage{sfkanbun}

\begin{document}

%%\begin{CJK*}[dnp]{JIS}{min}
\begin{CJK}{UTF8}{min}
\CJKtilde
%\CJKcaption{JIS}
%%%%%%%%%%%%%%%%%%%%%%%%%%%%%%%%%%%%%%
% Omote waza 
%%%%%%%%%%%%%%%%%%%%%%%%%%%%%%%%%%%%%%
\begin{landscape}
\thispagestyle{empty} %should remove the page number
TEST123
\rotatebox{-90}{
合気武道テスト
    \ruby{合気武道テスト}{あいきぶどうてすと}
}


\begin{center}
    \textbf{基本投げ技 Kihon nage waza (7 technieken)}
\end{center}
\addcontentsline{toc}{subsection}{\protect 基本投げ技 Kihon nage waza (7 technieken)}
\begin{table}[H]
\begin{center}
\scriptsize
\begin{tabular}{ccc}
\multicolumn{3}{c}{interval en ruimte laten - lijn van het lichaam}\\
\multicolumn{3}{c}{間合と間 - 体の線}\\
\multicolumn{3}{c}{ma ai to ma - tai no sen}\\
\\
(met) achterkant (vuist) (naar) zijkant van het gezicht slagen & naartoe gaan en neerslaan & achterwaartse val\\
裏横面打ち[うらよこめぬち] & 向え倒し[むかえだおし] & 後受身[うしろうけみ]\\
ura yoko men uchi & mukae daoshi & ushiro ukemi\\
\\
(met) voorkant (vuist) (naar) zijkant van het gezicht slagen & ? & achterwaartse val\\
\\
omote yoko men uchi & shiho nage & ushiro ukemi\\
\\
hoge stoot & elkaar kruisen & achterwaartse val\\
突き上段 & 行き違え & 後受身[うしろうけみ]\\
tsuki j\={o}dan & yuki chigae & ushiro ukemi\\
\\
stoot naar het midden & veer/schroef van horloge onderarm omkering & voorwaartse val\\
\\
tsuki ch\={u}dan & neji kote gaeshi & mae ukemi\\
\\
? mouw vastpakken & ? & voorwaartse val\\
?袖取り[りょうそでどり]  &   &   \\
ry\={o} sode dori & tenbin nage & mae ukemi\\
\\
? & ? & achterwaartse val\\
\\
do soku te dori & hachi mawashi & ushiro ukemi\\
\\
? handen vastpakken & heup worp & voorwaartse val\\
\\
ry\={o} te dori & koshi nage & mae ukemi
\end{tabular}
\end{center}
\label{kihonnagewaza}
\end{table}
\end{landscape}

\newpage
\thispagestyle{empty} %should remove the page number
\begin{landscape}
\rotatebox{-90}{
    %%\ruby{テスト}{test}
    test 123
}
\begin{center}
    \textbf{基本押え技 Kihon osae waza (6 technieken)}
\end{center}
\addcontentsline{toc}{subsection}{\protect 基本押え技 Kihon osae waza (6 technieken)}
\begin{table}[H]
\begin{center}
\small
\begin{tabular}{ccc}
\multicolumn{3}{c}{interval en ruimte laten - lijn van het lichaam}\\
\multicolumn{3}{c}{間合と間 - 体の線}\\
\multicolumn{3}{c}{ma ai to ma - tai no sen}\\
\\
stoot op middenhoogte & in een achterwaartse beweging de elleboog breken & schouder gewricht\\
?[?] & ?[?] & ?[?]\\
tsuki ch\={u}dan & ushiro hiji kudaki & kata kansetsu
\\
(beide handen achter de rug vastpakken) & elleboog naar gezicht, opzij trekken na verlies evenwicht & schouder gewricht\\
ry\={o} te ippo dori & robuse & kata kansetsu\\
\\
(gi onder de schouder vastpakken) & pols breken & schouder gewricht\\
kata sode dori & kote kudaki & kata kansetsu\\
\\
\hline
\\
\multicolumn{3}{c}{interval en dichtbij - lijn van het lichaam}\\
\multicolumn{3}{c}{間合と近間 - 体の線}\\
\multicolumn{3}{c}{ma ai to chikama - tai no sen}\\
\\
\\
\\
ushiro kata dori & eri shime yuki chigae & ude kansetsu\\
\\
\\
ushiro ry\={o} te dori& shiho nage & kata kansetsu\\
\\
\\
ushiro uwate dori & mukae daoshi & kata kansetsu
\end{tabular}
\end{center}
\label{kihonosaewaza}
\end{table}
\end{landscape}

\newpage
\begin{center}
    \textbf{雄用技投げ技 Oy\={o} waza nage waza (間 - 近間)}
\end{center}
\addcontentsline{toc}{subsection}{雄用技投げ技 Oy\={o} waza nage waza (間 - 近間)}
\begin{table}[H]
\begin{center}
\begin{tabular}{c}
technieken voor mannen werp technieken(ruimte laten - dichtbij)\\
雄用技投げ技(間 - 近間)\\
oy\={o} waza nage waza (ma - chikama)\\
\end{tabular}
\end{center}
\label{oyouwazanagewaza}
\end{table}

\begin{center}
    \textbf{雄用技押え技 Oy\={o} waza osae waza (間 - 近間)}
\end{center}
\addcontentsline{toc}{subsection}{雄用技押え技 Oy\={o} waza osae waza (間 - 近間)}
\begin{table}[H]
\begin{center}
\begin{tabular}{c}
technieken voor mannen werp technieken(ruimte laten - dichtbij)\\
雄用技押え技(間 - 近間)[おようわざおさえわざ (ま - ちかま)]\\
oy\={o} waza osae waza (ma - chikama)\\
\end{tabular}
\end{center}
\label{oyouwazaosaewaza}
\end{table}

\begin{center}
    \textbf{和の精神 Wa no sei shin}
\end{center}
\addcontentsline{toc}{subsection}{和の精神 Wa no sei shin}
\begin{table}[H]
\begin{center}
\begin{tabular}{c}
geest in harmonie\\
和の精神[わのせいしん]\\
wa no sei shin\\
\hline
voorwaarts en achterwaarts - lijn van het lichaam\\
前 と 後 - 体の線\\
mae to ushiro - tai no sen
\end{tabular}
\end{center}
\label{wanoseishin}
\end{table}

%%%%%%%%%%%%%%%%%%%%%%%%%%%%%%%%%%%%%%
% Katori
%%%%%%%%%%%%%%%%%%%%%%%%%%%%%%%%%%%%%%
\newpage
\section{Tenshin Sh\={o}den Katori Shint\={o} Ry\={u}}
\subsection{Algemeen}
\begin{table}[H]
\begin{center}
\begin{tabular}{c}
de weg van de hemel gewijdt aan de positieve legende van het Katori altaar\\ 
天真正傳香取神道流[てんしんしょうでんかとりしんとうりゅう]\\
tenshin sh\={o}den katori shint\={o} ry\={u}
\end{tabular}
\end{center}
\label{katori}
\end{table}

\end{CJK}
\end{document}

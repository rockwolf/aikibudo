\subsubsection{Algemeen}
\begin{table}[H]
\begin{center}
\begin{tabular}{c}
    \ruby{第二段}{だいにだん}\\
    daini dan\\
    \tran{2de dan}\\
    \hline
    \ruby{追加}{ついか}\\
    tsuika\\
    \tran{aanvullend}
\end{tabular}
\end{center}
\label{dan_2_gen}
\end{table}

\subsubsection{基本投げ技・Kihon nage waza}
\noindent Werptechnieken onder de vorm van een kata
\\
\begin{table}[H]
\begin{center}
\scriptsize
\begin{tabular}{BB}
    \ruby{}{} & \ruby{裏横面打}{うらよこめんう}ち\\
    ude garami & ura yoko men uchi\\
    ? & \tran{(met) achterkant (vuist) (naar) zijkant van het gezicht slagen}\\
    \\
    \ruby{前}{まえ}\ruby{}{} & \ruby{}{}\\
    mae hiki otoshi & shomen men uchi\\
    ? & ?\\
    \\
    \ruby{手}{て}?\ruby{返}{がえ}し& \ruby{突}{つき}\ruby{上段}{じょうだん}\\
    te uchi mata gaeshi & tsuki j\={o}dan\\
    ? & \tran{hoge stoot}\\
    \\
    \ruby{}{}\ruby{前}{まえ}\ruby{落}{おと}し & \ruby{突}{つき}\ruby{中}{ちゅう}\ruby{段}{だん}\\
    ude kake mae hiki otoshi & tsuki ch\={u}dan\\
    ? & \tran{stoot naar het midden}\\
    \\
    \ruby{裏}{うら}\ruby{向}{むか}え\ruby{倒}{だお}し & \ruby{前}{まえ}\ruby{蹴}{け}り\\
    ura mukae daoshi & mae keri\\
    ? & \tran{voorwaartse trap}\\
    \\
    \ruby{}{} & \ruby{}{}\\
    gyaku kote gaeshi & junte dori + (tsuki j\={o}dan)\\
    ? & ?\\
    \\
    \ruby{}{} & \ruby{前}{まえ}\ruby{}{}\\
    ashi tori oshi taoshi & mae eri\\
    ? & ?
\end{tabular}
\end{center}
\label{kihonnagewaza}
\end{table}

\subsubsection{追加の技・Aanvullende technieken}
\begin{table}[H]
\begin{center}
\begin{tabular}{BB}
    ? & ?\\
    ude garami & ura kataha\\
    ? & ?\\
    \hline\\
    ? \ruby{落}{おと}し & ?\\
    mae hiki otoshi & hiji gaeshi\\
    ? & ?\\
    \hline\\
    \ruby{後}{うしろ}?\ruby{落}{おと}し & ?\\
    ushiro hiki otoshi & mae hiji kudaki\\
    ? & ?\\
    \hline\\
    ? & \ruby{肩}{かた}は\ruby{落}{おと}し\\
    te uchi mata gaeshi & kata ha otoshi\\
    ? & \tran{schouder verliezen}\\
    \hline\\
    ? & ?\\
    gyaku te uchi mata gaeshi & mae tobu nage\\
    ? & ?\\
    \hline\\
    ?\ruby{落}{おと}し & ?\\
    ude kake mae hiki otoshi & ura mae tobu nage\\
    ? & ?\\
    \hline\\
    ? & ?\\
    gyaku kote gaeshi & do gaeshi\\
    ? & ?\\
    \hline\\
    \ruby{足}{あし}\ruby{取}{と}り\ruby{押}{お}し\ruby{倒}{たお}し & ?\\
    ashitori oshi taoshi & gyaku do gaeshi\\
    \tran{tegenstander neerhalen door zijn been vast te pakken en neerhalen door frontaal duwen} & ?
\end{tabular}
\end{center}
\label{dan_2_gen}
\end{table}

\subsubsection{Han sutemi (Kihon)}
\begin{table}[H]
\begin{center}
\begin{tabular}{ll}
    \ruby{}{}\ruby{落}{おと}し & \ruby{}{}\\
    kubi otoshi sutemi & tchoku tsuki\\
    ? & ?\\
    \\
    \ruby{}{} & \ruby{}{}\\
    hazu oshi sutemi & tchoku tsuki\\
    ? & ?\\
    \\
    \ruby{}{} & \ruby{}{}\\
    harite sutemi & tchoku tsuki\\
    ? & ?
\end{tabular}
\end{center}
\label{dan_2_bukidori_tanto}
\end{table}

\subsubsection{Buki dori}
Tanto dori
\begin{table}[H]
\begin{center}
\begin{tabular}{lll}
    \ruby{}{} & \ruby{}{} & (\ruby{}{} - \ruby{}{})\\
    kote gaeshi & tsuki ch\={u}dan & (nage - osae)\\
    ? & ? & ?\\
    \\
    \ruby{}{}\ruby{落}{おと}し & \ruby{}{} & (\ruby{}{} - \ruby{}{})\\
    kataha otoshi & tsuki ch\={u}dan & (nage - osae)\\
    ? & ? & ?\\
    \\
    \ruby{}{} & \ruby{}{} & (\ruby{}{} - \ruby{}{})\\
    shih\={o} nage & omote yoko men uchi & (nage - osae)\\ 
    ? & ? & ?
\end{tabular}
\end{center}
\label{dan_2_bukidori_tanto}
\end{table}

Hanbo dori
\begin{table}[H]
\begin{center}
\begin{tabular}{lll}
    \ruby{}{} & \ruby{}{} & (\ruby{}{} - \ruby{}{})\\
    kote gaeshi & tsuki ch\={u}dan & \\
    ? & ? & ?\\
    \\
    \ruby{}{} & \ruby{}{} & (\ruby{}{} - \ruby{}{})\\
    tenbin nage & tsuki ch\={u}dan & \\
    ? & ? & ?\\
    \\
    \ruby{}{} & \ruby{}{} & (\ruby{}{} - \ruby{}{})\\
    shih\={o} nage & tsuki ch\={u}dan & \\ 
    ? & ? & ?
\end{tabular}
\end{center}
\label{dan_2_bukidori_hanbo}
\end{table}

\subsubsection{Wa no seishin}

\subsubsection{型・Kata}
\begin{table}[H]
\begin{center}
\begin{tabular}{lcc}
    Zonder wapen: & test & test \\
    \hline
    Met wapen: & test & test
\end{tabular}
\end{center}
\label{kata_dan_2}
\end{table}

\subsubsection{Randori}
\begin{table}[H]
\begin{center}
\begin{tabular}{rl}
    ? & 1 tegen 1 - toepassen van diverse technieken \\
    j\={u} no randori & en ma / chika ma - (soepel) \\
    ? & \\
    \hline
    ? & 1 tegen 3 - Ontwijkingen/kanalisaties \\
    randori tai sabaki & \\
    ? & \\
    \hline
    ? &  \\
    randori wa no sei shin & \\
    ? & \\
    \hline
    ? &  \\
    taninzu dori randori & Realistische, verdediging en toepassing van technieken tegen meerdere partners\\
    ? &
\end{tabular}
\end{center}
\label{randori_dan_2}
\end{table}


\subsubsection{Extra informatie}

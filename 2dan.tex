\subsubsection{Algemeen}
\begin{table}[H]
\begin{center}
\begin{tabular}{c}
    \ruby{第二段}{だいにだん}\\
    daini dan\\
    2de dan\\
    \hline
    \ruby{追加}{ついか}\\
    tsuika\\
    aanvullend
\end{tabular}
\end{center}
\label{dan_2_gen}
\end{table}

\subsubsection{基本投げ技 Kihon nage waza}
\noindent Werptechnieken onder de vorm van een kata
\\
\begin{table}[H]
\begin{center}
\scriptsize
\begin{tabular}{ll}
    \ruby{}{} & \ruby{}{}\\
    ude garami & ura yoko men uchi\\
    ? & ?\\
    \\
    \ruby{}{} & \ruby{}{}\\
    mae hiki otoshi & shomen men uchi
    ? & ?\\
    \\
    \ruby{}{} & \ruby{}{}\\
    te uchi mata gaeshi & tsuki j\={o}dan\\
    ? & ?\\
    \\
    \ruby{}{} & \ruby{}{}\\
    ude kake mae hiki otoshi & tsuki ch\={u}dan\\
    ? & ?\\
    \\
    \ruby{}{} & \ruby{}{}\\
    ura mukae daoshi & mae geri\\
    ? & ?\\
    \\
    \ruby{}{} & \ruby{}{}\\
    gyaku kote gaeshi & junte dori + (tsuki j\={o}dan)\\
    ? & ?\\
    \\
    \ruby{}{} & \ruby{}{}\\
    ashi tori oshi taoshi & mae eri\\
    ? & ?\\
    \\
    \ruby{裏横面打}{うらよこめんう}ち & \ruby{向}{むか}え\ruby{倒}{だお}し & \ruby{後受身}{うしろうけみ}\\
    ura yoko men uchi & mukae daoshi & ushiro ukemi\\
    (met) achterkant (vuist) (naar) zijkant van het gezicht slagen & naartoe gaan
    en neerhalen & achterwaartse val\\
    \\
    \ruby{表横面打}{おもてよこめんう}ち & \ruby{四方投}{しほうな}げ &
    \ruby{後}{うしろ}\ruby{受身}{うけみ}\\
    omote yoko men uchi & shih\={o} nage & ushiro ukemi\\
    (met) voorkant (vuist) (naar) zijkant van het gezicht slagen & worp in elke richting & achterwaartse val\\
    \\
    \ruby{突}{つき}\ruby{上段}{じょうだん} & \ruby{行}{ゆ}き\ruby{違}{ちが}え & \ruby{後受身}{うしろうけみ}\\
    tsuki j\={o}dan & yuki chigae & ushiro ukemi\\
    hoge stoot & elkaar kruisen & achterwaartse val\\
    \\
    \ruby{突}{つき}\ruby{中}{ちゅう}\ruby{段}{だん} &
    \ruby{捻}{ねじ}\ruby{小手}{こて}\ruby{返}{がえ}し &
    \ruby{前}{まえ}\ruby{受身}{うけみ}\\
    tsuki ch\={u}dan & neji kote gaeshi & mae ukemi\\
    stoot naar het midden & veer/schroef van horloge onderarm omkering & voorwaartse val\\
    \\
    \ruby{両}{りょう}\ruby{袖}{そで}\ruby{取}{ど}り & \ruby{天秤投}{てんびんな}げ & \ruby{前}{まえ}\ruby{受身}{うけみ}\\
    ry\={o} sode dori & tenbin nage & mae ukemi\\
    beide mouwen vastpakken & weegschaal worp & voorwaartse val\\
    \\
    \ruby{土}{ど}\ruby{足}{そく}\ruby{手}{て}\ruby{取}{ど}り &
    \ruby{鉢}{はち}\ruby{廻}{まわ}し& \ruby{後}{うしろ}\ruby{受身}{うけみ}\\
    do soku te dori & hachi mawashi & ushiro ukemi\\
    shoenen hand actieve partner & hersenpan roteren & achterwaartse val\\
    \\
    \ruby{両}{りょう}\ruby{手}{て}\ruby{取}{ど}り &
    \ruby{腰}{こし}\ruby{投}{な}げ & \ruby{前}{まえ}\ruby{受身}{うけみ}\\
    ry\={o} te dori & koshi nage & mae ukemi \\
    beide handen vastpakken & heup worp & voorwaartse val\\
\end{tabular}
\end{center}
\label{kihonnagewaza}
\end{table}

\subsubsection{追加の技 Aanvullende technieken}
\begin{table}[H]
\begin{center}
\begin{tabular}{ll}
    ? & ?\\
    ude garami & ura kataha\\
    ? & ?\\
    \hline\\
    ? & ?\\
    mae hiki otoshi & hiji gaeshi\\
    ? & ?\\
    \hline\\
    ? & ?\\
    ushiro hiki otoshi & mae hiji kudaki\\
    ? & ?\\
    \hline\\
    ? & \ruby{肩}{かた}はおとし\\
    te uchi mata gaeshi & kata ha otoshi\\
    ? & schouder verliezen\\
    \hline\\
    ? & ?\\
    gyaku te uchi mata gaeshi & mae tobu nage\\
    ? & ?\\
    \hline\\
    ? & ?\\
    ude kake mae hiki otoshi & ura mae tobu nage\\
    ? & ?\\
    \hline\\
    ? & ?\\
    gyaku kote gaeshi & do gaeshi\\
    ? & ?\\
    \hline\\
    ? & ?\\
    ashi tori oshi taoshi & gyaku do gaeshi\\
    ? & ?\\
    \hline\\
\end{tabular}
\end{center}
\label{dan_2_gen}
\end{table}

\subsubsection{Han sutemi (Kihon)}

\subsubsection{Buki dori}
\subsubsubsection{Tanto dori}
\begin{table}[H]
\begin{center}
\begin{tabular}{lll}
\ruby{}{} & \ruby{}{} & (\ruby{}{} - \ruby{}{})\\
kote gaeshi & tsuki ch\={u}dan & (nage - osae)\\
? & ? & ?\\
\\
\ruby{}{} & \ruby{}{} & (\ruby{}{} - \ruby{}{})\\
kataha otoshi & tsuki ch\={u}dan & (nage - osae)\\
? & ? & ?\\
\\
\ruby{}{} & \ruby{}{} & (\ruby{}{} - \ruby{}{})\\
shih\={o} nage & omote yoko men uchi & (nage - osae)\\ 
? & ? & ?\\
\\
\end{tabular}
\end{center}
\label{dan_2_bukidori_tanto}
\end{table}

\subsubsubsection{Hanbo dori}
\begin{table}[H]
\begin{center}
\begin{tabular}{lll}
\ruby{}{} & \ruby{}{} & (\ruby{}{} - \ruby{}{})\\
kote gaeshi & tsuki ch\={u}dan & \\
? & ? & ?\\
\\
\ruby{}{} & \ruby{}{} & (\ruby{}{} - \ruby{}{})\\
tenbin nage & tsuki ch\={u}dan & \\
? & ? & ?\\
\\
\ruby{}{} & \ruby{}{} & (\ruby{}{} - \ruby{}{})\\
shih\={o} nage & tsuki ch\={u}dan & \\ 
? & ? & ?\\
\\
\end{tabular}
\end{center}
\label{dan_2_bukidori_hanbo}
\end{table}

\subsubsection{Wa no seishin}

\subsubsection{型 Kata}
\begin{table}[H]
\begin{center}
\begin{tabular}{lcc}
    Zonder wapen: & test & test \\
    \hline
    Met wapen: & test & test
\end{tabular}
\end{center}
\label{kata_dan_2}
\end{table}

\subsubsection{Randori}

\subsubsection{Extra informatie}

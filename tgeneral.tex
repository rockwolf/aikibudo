\subsection{Algemeen}
\begin{table}[H]
\begin{center}
\begin{tabular}{c|c|c}
    \ruby{基本}{きほん} & kihon & basis/oorsprong/standaard\\
    \hline
    \ruby{受身}{うけみ} & ukemi & de kunst van het veilig vallen\\
    \hline
    \ruby{取り}{どり} & dori & actieve partner\\
    \hline
    \ruby{押}{お}さえ & osae & controle\\
    \hline
    \ruby{技}{わざ} & waza & techniek\\
    \hline
    \ruby{構}{かま}え & kamae & houding
\end{tabular}
\end{center}
\end{table}

\subsection{Richtingen en gebieden}
\begin{table}[H]
\begin{center}
\begin{tabular}{c|c|c}
    \ruby{右}{みぎ} & migi & rechts\\
    \hline
    \ruby{左}{ひだり} & hidari & links\\
    \hline
    \ruby{後}{うし}ろ & ushiro & achterwaarts\\
    \hline
    \ruby{前}{まえ} & mae & voorwaarts\\
    \hline
    \ruby{上段}{じょうだん} & j\={o}dan & bovenste laag\\
    \hline
    \ruby{中段}{ちゅうだん} & ch\={u}dan & midden\\
    \hline
    \ruby{土}{ど} & do & grond
\end{tabular}
\end{center}
\end{table}

\subsection{Acties}
\begin{table}[H]
\begin{center}
\begin{tabular}{c|c|c}
    \ruby{投}{な}げ & nage & worp\\
    \hline
    \ruby{頭突}{ずつ}き & zutsuki & kopstoot\\
    \hline
    \ruby{蹴}{け}り & keri & trap\\
    \hline
    \ruby{絞殺}{こうさつ} & k\={o}satsu & wurging\\
    \hline
    \ruby{返}{がえ}し & gaeshi & omkering/terug sturen\\
    \hline
    \ruby{横蹴}{よこけ}り & yoko keri & zijwaartse trap\\
    \hline
    \ruby{突}{つ}き & tsuki & stoot
\end{tabular}
\end{center}
\end{table}

\subsection{Lichaamsdelen}
\begin{table}[H]
\begin{center}
\begin{tabular}{c|c|c}
    \ruby{体}{たい} & tai & lichaam\\
    \hline
    \ruby{手}{て} & te & hand\\
    \hline
    \ruby{肘}{ひじ} & hiji & elleboog/elleboogstoot\\
    \hline
    \ruby{膝}{ひざ} & hiza & knie\\
    \hline
    \ruby{手首}{てくび} & tekubi & pols\\
    \hline
    \ruby{小手}{こて} & kote & onderarm\\
    \hline
    \ruby{肩}{かた} & kata & schouder
\end{tabular}
\end{center}
\end{table}

\subsection{Andere begrippen}
\begin{table}[H]
\begin{center}
\begin{tabular}{c|c|c}
    \ruby{捻子}{ねじ} & neji & schroef/veer van horloge\\
    \hline
    \ruby{表}{おもて} & omote & uitwendig\\
    \hline
    \ruby{表}{ひょう} & hy\={o} & tabel\\
    \hline
    \ruby{型}{かた} & kata & standaard vorm van een beweging in de krijgskunst
\end{tabular}
\end{center}
\end{table}

\subsection{Leerstof}
\begin{table}[H]
\begin{center}
\begin{tabular}{c|c|c}
    \ruby{体捌}{たいさば}き & tai sabaki & verwijdering van het lichaam\\
    \hline
    \ruby{受身}{うけみ}\ruby{技}{わざ} & ukemi waza & valtechnieken\\
    \hline
    \ruby{突}{つ}き\ruby{技}{わざ} & tsuki waza & stoottechnieken\\
    \hline
    \ruby{蹴}{け}り\ruby{技}{わざ} & keri waza & traptechnieken\\
    \hline
    \ruby{補}{ほ}\ruby{助}{じょ}\ruby{運}{うん}\ruby{動}{どう} & hojo und\={o} & hulp-oefening\\
    \hline
    \ruby{掴}{つか}み\ruby{型}{かた}と\ruby{手}{て}\ruby{解}{ほど}き & tsukami kata to te hodoki & grip kata en hand lossen\\
    \hline
    \ruby{追加}{ついか}の\ruby{技}{わざ} & tsuika no waza & aanvullende technieken\\
    \hline
    \ruby{基本}{きほん}\ruby{投}{な}げ\ruby{技}{わざ} & kihon nage waza & basis werp technieken\\
    \hline
    \ruby{基本}{きほん}\ruby{押}{お}さえ\ruby{技}{わざ} & kihon osae waza & basis controle technieken\\
    \hline
    \ruby{型}{かた} & kata & standaard vorm van een beweging in de krijgskunst\\
    \hline
    \ruby{和}{わ}\ruby{精}{せい}\ruby{神}{しん} & wa no seishin & geest in harmonie\\ 
    \hline
    ? & ? & verdedigingsstok\\
    \hline
    \ruby{乱取}{らんど}り & randori & vrij gevecht
\end{tabular}
\end{center}
\end{table}

\subsubsection{体捌き・Tai sabaki}
\begin{table}[H]
\begin{center}
\begin{tabular}{cc}
    ? & \ruby{大}{おお}? \\
    irimi & \={o} irimi\\
    ? & ? 
\end{tabular}
\end{center}
\label{kyuu_6_taisabaki}
\end{table}

\subsubsection{受身技・Ukemi waza}
\begin{table}[H]
\begin{center}
\begin{tabular}{c}
    \ruby{後}{うし}ろ\\
    ushiro\\
    achterwaarts
\end{tabular}
\end{center}
\label{kyuu_6_ukemi_waza}
\end{table}

\subsubsection{突き技・Tsuki waza}
\begin{table}[H]
\begin{center}
\begin{tabular}{c}
    \ruby{直}{ちょく}\\
    choku tsuki\\
    rechte stoot 
\end{tabular}
\end{center}
\label{kyuu_6_ukemi_waza}
\end{table}

\subsubsection{蹴り技・Keri waza}
\subsubsection{補助運動・Hojo und\={o}}
\subsubsection{掴み型と手解き・Tsukami kata \& te hodoki}
\subsubsection{追加技・Aanvullende technieken}
\subsubsection{基本投げ技・Kihon nage waza}
\subsubsection{基本押え技・Kihon osae waza}
\subsubsection{歴史的・Historische technieken}

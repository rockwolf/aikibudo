\subsubsection{体捌き・Tai sabaki}
\begin{table}[H]
\begin{center}
\begin{tabular}{rlrcl}
    ? & (?) & (\ruby{}{} & - & \ruby{}{})\\
    ? & (choku tsuki) & (soto & - & omote)\\
    esquives canalisation? & (?) & (? & - & ?)
\end{tabular}
\end{center}
\label{kyuu_2_taisabaki}
\end{table}
\begin{center}
    1 $\leftrightarrow$ 2
\end{center}

\subsubsection{受身技・Ukemi waza}
\begin{table}[H]
\begin{center}
\begin{tabular}{c}
    Alle valoefeningen.
\end{tabular}
\end{center}
\label{kyuu_2_ukemi_waza}
\end{table}

\subsubsection{突き技・Tsuki waza}
\begin{table}[H]
\begin{center}
\begin{tabular}{c}
    \ruby{}{}\\
    tsuki uchi no kata\\
    ?
\end{tabular}
\end{center}
\label{kyuu_2_tsuki_waza}
\end{table}

\subsubsection{蹴り技・Keri waza}
\begin{table}[H]
\begin{center}
\begin{tabular}{c}
    \ruby{}{}\\
    ura keri\\
    ? trap
\end{tabular}
\end{center}
\label{kyuu_2_keri_waza}
\end{table}

\subsubsection{補助運動・Hojo undo}
\begin{table}[H]
\begin{center}
\begin{tabular}{rl}
    \ruby{}{} & (2 \ruby{}{})\\
    nigiri kaeshi & (2 ?)\\
    ? & (2 vormen)
\end{tabular}
\end{center}
\label{kyuu_2_hojo_undo}
\end{table}

\subsubsection{掴み型と手解き・Tsukami kata \& te hodoki}
\begin{table}[H]
\begin{center}
\begin{tabular}{c}
    \ruby{}{}\ruby{}{}\\
    ushiro ry\={o} sode dori\\
    ?\\
    \hline
    \ruby{}{}\\
    ushiro kubi jime\\
    ?\\
    \hline
    \ruby{}{}\\
    ushiro katate dori\\
    ?
\end{tabular}
\end{center}
\label{kyuu_2_te_hodoki}
\end{table}

\subsubsection{追加技・Aanvullende technieken}
\begin{table}[H]
\begin{center}
\begin{tabular}{c}
    \ruby{}{}\\
    ?\\
    op elke vorm van aanval en inkomen
\end{tabular}
\end{center}
\label{kyuu_2_additional}
\end{table}

\subsubsection{基本投げ技・Kihon nage waza}
\begin{table}[H]
\begin{center}
\begin{tabular}{rl}
    \ruby{}{}\ruby{}{} & \\
    tenbin nage & (ry={o} sode dori)\\
    ? & (?)\\
    \hline
    \ruby{}{} &\\
    hachi mawashi &\\
    ? &
\end{tabular}
\end{center}
\label{kyuu_2_kihon_nage_waza}
\end{table}

\subsubsection{基本押え技・Kihon osae waza}
\begin{table}[H]
\begin{center}
\begin{tabular}{rl}
    \ruby{}{}\ruby{}{} & \\
    shiho nage & (ushiro ry\={o} te dori)\\
    ? & (?)
\end{tabular}
\end{center}
\label{kyuu_2_kihon_osae_waza}
\end{table}

\subsubsection{歴史的技・Historische technieken}
\begin{table}[H]
\begin{center}
\begin{tabular}{rl}
    \ruby{}{}\ruby{}{} & \\
    daito ryu aikijujutsu & kuruma daoshi\\
    ? & ?
\end{tabular}
\end{center}
\label{kyuu_2_historic}
\end{table}

\subsubsection{型・Kata}
\begin{table}[H]
\begin{center}
\begin{tabular}{rl}
    \ruby{}{}\ruby{}{}\\
    ken no kata\\
    ?
\end{tabular}
\end{center}
\label{kyuu_2_kata}
\end{table}

\subsubsection{和の精神・Wa no seishin}
\begin{table}[H]
\begin{center}
\begin{tabular}{c}
    \ruby{}{}\\
    ushiro\\
    ?
\end{tabular}
\end{center}
\label{kyuu_2_wa_no_seishin}
\end{table}

\subsubsection{?・Verdedigingsstok}
\begin{table}[H]
\begin{center}
\begin{tabular}{c}
    \ruby{}{}\\
    tanbo waza\\
    ?
\end{tabular}
\end{center}
\label{kyuu_2_defense_stick}
\end{table}

\subsubsection{乱取り・Randori}
\begin{table}[H]
\begin{center}
\begin{tabular}{rl}
    \ruby{}{} & (\ruby{}{})\\
    futari no randori & (tai sabaki)\\
    ? & (?)
\end{tabular}
\end{center}
\label{kyuu_2_randori}
\end{table}
\begin{center}
    1 $\leftrightarrow$ 3
\end{center}

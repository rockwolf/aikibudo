\subsubsection{体捌き・Tai sabaki}
\begin{table}[H]
\begin{center}
\begin{tabular}{rlrcl}
    ? & (?) & (\ruby{}{} & - & \ruby{}{})\\
    ? & (choku tsuki) & (soto & - & omote)\\
    esquives canalisation? & (?) & (? & - & ?)
\end{tabular}
\end{center}
\label{kyuu_2_taisabaki}
\end{table}
\begin{center}
    1 $\leftrightarrow$ 2
\end{center}

\subsubsection{受身技・Ukemi waza}
\begin{table}[H]
\begin{center}
\begin{tabular}{c}
    すべての\ruby{受身}{うけみ}\ruby{技}{わざ}\\
    subete no ukemi waza\\
    alle valtechnieken
\end{tabular}
\end{center}
\label{kyuu_2_ukemi_waza}
\end{table}

\subsubsection{突き技・Tsuki waza}
\begin{table}[H]
\begin{center}
\begin{tabular}{c}
    \ruby{突}{つ}き\ruby{内}{うち}の\ruby{型}{かた}\\
    tsuki uchi no kata\\
    stoot binnenkant kata
\end{tabular}
\end{center}
\label{kyuu_2_tsuki_waza}
\end{table}

\subsubsection{蹴り技・Keri waza}
\begin{table}[H]
\begin{center}
\begin{tabular}{c}
    \ruby{裏}{うら}\ruby{蹴}{け}り\\
    ura keri\\
    achterkant trap
\end{tabular}
\end{center}
\label{kyuu_2_keri_waza}
\end{table}

\subsubsection{補助運動・Hojo und\={o}}
\begin{table}[H]
\begin{center}
\begin{tabular}{rl}
    \ruby{握}{にぎ}り\ruby{返}{かえ}し & (2 \ruby{}{})\\
    nigiri kaeshi & (2 ?)\\
    handgreep omkering & (2 vormen)
\end{tabular}
\end{center}
\label{kyuu_2_hojo_undou}
\end{table}

\subsubsection{掴み型と手解き・Tsukami kata \& te hodoki}
\begin{table}[H]
\begin{center}
\begin{tabular}{c}
    \ruby{}{}\ruby{}{}\\
    ushiro ry\={o} sode dori\\
    ?\\
    \hline
    \ruby{}{}\\
    ushiro kubi shime\\
    ?\\
    \hline
    \ruby{}{}\\
    ushiro katate dori\\
    ?
\end{tabular}
\end{center}
\label{kyuu_2_te_hodoki}
\end{table}

\subsubsection{追加技・Aanvullende technieken}
\begin{table}[H]
\begin{center}
\begin{tabular}{c}
    \ruby{}{}\\
    ?\\
    op elke vorm van aanval en inkomen
\end{tabular}
\end{center}
\label{kyuu_2_additional}
\end{table}

\subsubsection{基本投げ技・Kihon nage waza}
\begin{table}[H]
\begin{center}
\begin{tabular}{rl}
    \ruby{天秤投}{てんびんな}げ & (\ruby{両}{りょう}\ruby{袖}{そで}\ruby{取}{ど}り)\\
    tenbin nage & (ry\={o} sode dori)\\
    weegschaal worp & (beide mouwen vastpakken)\\
    \hline
    \ruby{鉢}{はち}\ruby{廻}{まわ}し &\\
    hachi mawashi &\\
    hersenpan roteren &
\end{tabular}
\end{center}
\label{kyuu_2_kihon_nage_waza}
\end{table}

\subsubsection{基本押え技・Kihon osae waza}
\begin{table}[H]
\begin{center}
\begin{tabular}{rl}
    \ruby{四方}{しほう}\ruby{投}{な}げ & (\ruby{後}{うし}ろ\ruby{両}{りょう}\ruby{手}{て}\ruby{取}{ど}り)\\
    shiho nage & (ushiro ry\={o} te dori)\\
    worp in elke richting & (achterkant beide handen vastpakken)
\end{tabular}
\end{center}
\label{kyuu_2_kihon_osae_waza}
\end{table}

\subsubsection{歴史的技・Historische technieken}
\begin{table}[H]
\begin{center}
\begin{tabular}{rl}
    \ruby{大}{だい}\ruby{東}{とう}\ruby{流}{りゅう}\ruby{合気柔術}{あいきじゅうじゅつ} & \ruby{車}{くるま}\ruby{倒}{だお}し\\
    daito ry\={u} aiki j\={u}jutsu & kuruma daoshi\\
    grote oosterse school van de zachte technieken & wiel neerhaling
\end{tabular}
\end{center}
\label{kyuu_2_historic}
\end{table}

\subsubsection{型・Kata}
\begin{table}[H]
\begin{center}
\begin{tabular}{rl}
    \ruby{剣}{けん}の\ruby{型}{かた}\\
    ken no kata\\
    zwaard kata
\end{tabular}
\end{center}
\label{kyuu_2_kata}
\end{table}

\subsubsection{和の精神・Wa no seishin}
\begin{table}[H]
\begin{center}
\begin{tabular}{c}
    \ruby{後}{うし}ろ\\
    ushiro\\
    achterwaarts/achterkant
\end{tabular}
\end{center}
\label{kyuu_2_wa_no_seishin}
\end{table}

\subsubsection{防衛の棒・Verdedigingsstok}
\begin{table}[H]
\begin{center}
\begin{tabular}{c}
    \ruby{短棒}{たんぼう}\ruby{技}{わざ}\\
    tanb\={o} waza\\
    korte stok technieken
\end{tabular}
\end{center}
\label{kyuu_2_defense_stick}
\end{table}

\subsubsection{乱取り・Randori}
\begin{table}[H]
\begin{center}
\begin{tabular}{rl}
    \ruby{二人}{ふたり}の\ruby{乱取}{らんど}り & (\ruby{体}{たい}\ruby{捌}{さば}き)\\ 
    futari no randori & (tai sabaki)\\
    vrij gevecht met 2 personen & (lichaam hanteren)
\end{tabular}
\end{center}
\label{kyuu_2_randori}
\end{table}
\begin{center}
    1 $\leftrightarrow$ 2
\end{center}

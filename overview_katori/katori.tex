%%%%%%%%%%%%%%%%%%%%%%%%%%%%%%%%%%%%%%
% Imports
%%%%%%%%%%%%%%%%%%%%%%%%%%%%%%%%%%%%%%
% Note: compile dvi with uplatex katori.tex
\documentclass[dvipdfmx, a4paper, 12pt]{utarticle}
%\documentclass{utarticle}
%\usepackage{color} % requires divpdfmx
\usepackage{CJK}
\usepackage[CJK, overlap]{ruby} %furigana support
%\usepackage{float}
%\usepackage{graphicx}
%\usepackage{multirow}
\usepackage{pdflscape}
%\usepackage[left=2cm,top=2cm,right=2cm, bottom=2cm, nohead, nofoot]{geometry}
%\usepackage[usenames,dvipsnames]{color}
%\usepackage{array}


%%%%%%%%%%%%%%%%%%%%%%%%%%%%%%%%%%%%%%
% Types
%%%%%%%%%%%%%%%%%%%%%%%%%%%%%%%%%%%%%%
%\newcolumntype{s}{>{\centering\arraybackslash}m{1cm}}
\newcolumntype{B}{>{\centering\arraybackslash}m{4.0cm}}
\newcolumntype{L}{>{\centering\arraybackslash}m{7.0cm}}


%%%%%%%%%%%%%%%%%%%%%%%%%%%%%%%%%%%%%%
% Commands
%%%%%%%%%%%%%%%%%%%%%%%%%%%%%%%%%%%%%%
%\renewcommand{\rubysize}{0.5}
\renewcommand{\rubysep}{-0.1ex}
\newcommand{\tran}[1]{{\itshape{\color{Gray}{#1}}}}

%\renewcommand{\rubysize}{0.2}
\renewcommand{\rubysep}{-0.2ex}

%%%%%%%%%%%%%%%%%%%%%%%%%%%%%%%%%%%%%%
% Start of document + settings
%%%%%%%%%%%%%%%%%%%%%%%%%%%%%%%%%%%%%%
\begin{document}
\begin{CJK*}{UTF8}{min}
\CJKtilde
\begin{landscape}

%%%%%%%%%%%%%%%%%%%%%%%%%%%%%%%%%%%%%%
% Kanji
%%%%%%%%%%%%%%%%%%%%%%%%%%%%%%%%%%%%%%
\newpage
\pagestyle{empty}
天真正傳
香取神道流
\section{居合術、表六箇條}
\noindent 草薙之剣\\
拔附之剣\\
拔討之剣\\
右剣\\
左剣\\
八方剣\\
\section{立合術、表五箇條}
\noindent 行合逆抜之太刀\\
前後千鳥之太刀\\
行合右千鳥之太刀\\
逆拔之太刀\\
拔討之太刀\\
\\
畠山五郎?

%%%%%%%%%%%%%%%%%%%%%%%%%%%%%%%%%%%%%%
% Kanji + furigana
%%%%%%%%%%%%%%%%%%%%%%%%%%%%%%%%%%%%%%
\setcounter{section}{0}
\newpage
\pagestyle{empty}
\ruby{天真正傳香取神道流}{てんしんしょうでんかとりしんとうりゅう}
\section{\ruby{居合術、表六箇條}{いあいじゅつ、おもてろっかじょう}}
\noindent \ruby{草薙之剣}{くさなぎのけん}\\
\ruby{拔附之剣}{ぬきづけのけん}\\
\ruby{拔討之剣}{ぬきうちのけん}\\
\ruby{右剣}{うけん}\\
\ruby{左剣}{さけん}\\
\ruby{八方剣}{はっぽうけん}\\
\section{\ruby{立合術、表五箇條}{たちあいじゅつ、おもてごごかじょう}}
\noindent \ruby{行合逆抜之太刀}{ゆきあいぎゃくぬきのたち}\\
\ruby{前後千鳥之太刀}{ぜんごちどりのたち}\\
\ruby{行合右千鳥之太刀}{ゆきあいみぎちどりのたち}\\
\ruby{逆拔之太刀}{ぎゃくぬきのたち}\\
\ruby{拔討之太刀}{ぬきうちのたち}\\
\\
\ruby{畠山五郎?}{はたけやまごろう?}\\

%%%%%%%%%%%%%%%%%%%%%%%%%%%%%%%%%%%%%%
% Kanji + roumaji
%%%%%%%%%%%%%%%%%%%%%%%%%%%%%%%%%%%%%%
\setcounter{section}{0}
\newpage
\pagestyle{empty}
\ruby{天真正傳香取神道流}{tenshin sh\={o}den katori shinto ry\={u}}
\section{\ruby{居合術表六箇條}{iai jutsu omote rokkaj\={o}}}
\noindent \ruby{草薙之剣}{kusanagi no ken}\\
\ruby{拔附之剣}{nuki zuke no ken}\\
\ruby{拔討之剣}{nuki uchi no ken}\\
\ruby{右剣}{u ken}\\
\ruby{左剣}{sa ken}\\
\ruby{八方剣}{happ\={o} ken}\\
\section{\ruby{立合術、表五箇條}{tachi ai jutsu, omote go kaj\={o}}}
\noindent \ruby{行合逆抜之太刀}{yuki ai gyaku nuki no tachi}\\
\ruby{前後千鳥之太刀}{zengo chidori no tachi}\\
\ruby{行合右千鳥之太刀}{yuki ai migi chidori no tachi}\\
\ruby{逆拔之太刀}{gyaku nuki no tachi}\\
\ruby{拔討之太刀}{nuki uchi no tachi}\\
\\
\ruby{畠山五郎?}{hatakeyama gor\={o} ?}

%%%%%%%%%%%%%%%%%%%%%%%%%%%%%%%%%%%%%%
% Kanji + translation
%%%%%%%%%%%%%%%%%%%%%%%%%%%%%%%%%%%%%%
\setcounter{section}{0}
\newpage
\pagestyle{empty}
天真正傳・???
香取神道流・???
\section{居合術、表六箇條・sword drawing techniques, 6 part table}
\noindent 草薙之剣・sword, grasscutter (cfr. Kusanagi no tsurugi from Japanese mythology)\\
拔附之剣・sword, ?\\
拔討之剣・sword, piercing attack\\
右剣・sword, right hand side\\
左剣・sword, left hand side\\
八方剣・sword, all sides (the 4 cardinal and ordinal directions)\\
\section{立合術、表五箇條・engagement techniques, 5 part table}
\noindent 行合逆抜之太刀・long sword, ?\\
前後千鳥之太刀・long sword, ?\\
行合右千鳥之太刀・long sword, ?\\
逆拔之太刀・long sword, ?\\
拔討之太刀・long sword, piercing attack\\
\\
畠山五郎?・Hatakeyama Gor\={o} ?


%%%%%%%%%%%%%%%%%%%%%%%%%%%%%%%%%%%%%%
% Kanji + furigana + translation
%%%%%%%%%%%%%%%%%%%%%%%%%%%%%%%%%%%%%%
\setcounter{section}{0}
\newpage
\pagestyle{empty}
\ruby{天真正傳香取神道流}{てんしんしょうでんかとりしんとうりゅう}・\\
\section{\ruby{居合術表六箇條}{いあいじゅつおもてろっかじょう}・iai techniques six part table}
\noindent \ruby{草薙之剣}{くさなぎのけん}・Grasscutter sword (cfr. Kusanagi no tsurugi in Japanese mythology)\\
\ruby{拔附之剣}{ぬきづけのけん}・\\
\ruby{拔討之剣}{ぬきうちのけん}・\\
\ruby{右}{う}\ruby{剣}{けん}・sword, right hand side\\
\ruby{左}{さ}\ruby{剣}{けん}・sword, left hand side\\
\ruby{八方剣}{はっぽうけん}・sword, all sides (the 4 cardinal and ordinal directions)\\
\section{\ruby{立合術、表五箇條}{たちあいじゅつ、おもてごかじょう}・engagement techniques, 5 part table}
\noindent \ruby{行合逆抜之太刀}{TBD}\\
\ruby{前後千鳥之太刀}{TBD}\\
\ruby{行合右千鳥之太刀}{TBD}\\
\ruby{逆拔之太刀}{TBD}\\
\ruby{拔討之太刀}{TBD}\\
\\
\ruby{畠山五郎?}{はたけやまごろう?}・Hatakeyama Gor\={o} ?\\

%%%%%%%%%%%%%%%%%%%%%%%%%%%%%%%%%%%%%%
% Kanji + roumaji + translation
%%%%%%%%%%%%%%%%%%%%%%%%%%%%%%%%%%%%%%
\setcounter{section}{0}
\newpage
\pagestyle{empty}
\ruby{天真正傳香取神道流}{tenshin sh\={o}den katori shinto ryuu}・\\
\section{\ruby{居合術表六箇條}{iai jutsu omote rokkaj\={o}}・iai techniques six part table}
\noindent TBD1\\
TBD2\\
TBD3\\
TBD4\\
TBD5\\
TBD6\\
\section{\ruby{立合術、表五箇條}{tachi ai jutsu omote go kaj\={o}}・engagement techniques, 5 part table}
\noindent TBD1\\
TBD2\\
TBD3\\
TBD4\\
TBD5\\
\\
\ruby{畠山五郎}{hatake yama gor\={o}}?・Hatakeyama Gor\={o}?

\end{landscape}
\end{CJK*}
\end{document}

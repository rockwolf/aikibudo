%%%%%%%%%%%%%%%%%%%%%%%%%%%%%%%%%%%%%%
% Imports
%%%%%%%%%%%%%%%%%%%%%%%%%%%%%%%%%%%%%%
% Note: compile dvi with uplatex katori.tex
\documentclass[dvipdfmx, a4paper, 12pt]{utarticle}
%\documentclass{utarticle}
%\usepackage{color} % requires divpdfmx
\usepackage{CJK}
\usepackage[CJK, overlap]{ruby} %furigana support
%\usepackage{float}
%\usepackage{graphicx}
%\usepackage{multirow}
\usepackage{pdflscape}
%\usepackage[left=2cm,top=2cm,right=2cm, bottom=2cm, nohead, nofoot]{geometry}
%\usepackage[usenames,dvipsnames]{color}
%\usepackage{array}


%%%%%%%%%%%%%%%%%%%%%%%%%%%%%%%%%%%%%%
% Types
%%%%%%%%%%%%%%%%%%%%%%%%%%%%%%%%%%%%%%
%\newcolumntype{s}{>{\centering\arraybackslash}m{1cm}}
\newcolumntype{B}{>{\centering\arraybackslash}m{4.0cm}}
\newcolumntype{L}{>{\centering\arraybackslash}m{7.0cm}}


%%%%%%%%%%%%%%%%%%%%%%%%%%%%%%%%%%%%%%
% Commands
%%%%%%%%%%%%%%%%%%%%%%%%%%%%%%%%%%%%%%
%\renewcommand{\rubysize}{0.5}
\renewcommand{\rubysep}{-0.1ex}
\newcommand{\tran}[1]{{\itshape{\color{Gray}{#1}}}}

\renewcommand{\rubysize}{0.2}
\renewcommand{\rubysep}{-0.1ex}

%%%%%%%%%%%%%%%%%%%%%%%%%%%%%%%%%%%%%%
% Start of document + settings
%%%%%%%%%%%%%%%%%%%%%%%%%%%%%%%%%%%%%%
\begin{document}
\begin{CJK*}{UTF8}{min}
\CJKtilde
\begin{landscape}

% Kanji
\newpage
\pagestyle{empty}
天真正傳
香取神道流
\section{居合術表六箇條}
TBD1
TBD2
TBD3
TBD4
TBD5
TBD6
\section{}
TBD1
TBD2
TBD3
TBD4
TBD5

畠山???

% Kanji + furigana
\newpage
\pagestyle{empty}
\ruby{天真}{てんしん}\ruby{正傳}{しょうでん}\ruby{香}{かとり}\ruby{取神}{しんとう}\ruby{道流}{りゅう}
\section{\ruby{居合}{いあい}\ruby{術}{じゅつ}\ruby{表}{おもて}\ruby{六箇條}{ろっかじょう}}
TBD1
TBD2
TBD3
TBD4
TBD5
TBD6
\section{}
TBD1
TBD2
TBD3
TBD4
TBD5

\ruby{畠}{はたけ}\ruby{山}{やま}???

% Kanji + roumaji
\newpage
\pagestyle{empty}
\ruby{天真}{tenshin}\ruby{正傳}{shouden}\ruby{香}{katori}\ruby{取神}{shinto}\ruby{道流}{ryuu}
\section{\ruby{居合}{iai}\ruby{術}{jutsu}\ruby{表}{omote}\ruby{六箇條}{rokkajou}}
TBD1
TBD2
TBD3
TBD4
TBD5
TBD6
\section{}
TBD1
TBD2
TBD3
TBD4
TBD5

\ruby{畠}{hatake}\ruby{山}{yama}???

% Kanji + roumaji + translation
\newpage
\pagestyle{empty}
\ruby{天真}{tenshin}\ruby{正傳}{shouden}\ruby{香}{katori}\ruby{取神}{shinto}\ruby{道流}{ryuu}
TBD
\section{\ruby{居合}{iai}\ruby{術}{jutsu}\ruby{表}{omote}\ruby{六箇條}{rokkajou}}
iai (the art of drawing one's sword) techniques six part table
TBD1
TBD2
TBD3
TBD4
TBD5
TBD6
\section{}
5 part table
TBD1
TBD2
TBD3
TBD4
TBD5

\ruby{畠}{はたけ}\ruby{山}{やま}???
Hatakeyama ???

\end{landscape}
\end{CJK*}
\end{document}

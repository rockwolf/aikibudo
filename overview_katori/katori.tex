%%%%%%%%%%%%%%%%%%%%%%%%%%%%%%%%%%%%%%
% Imports
%%%%%%%%%%%%%%%%%%%%%%%%%%%%%%%%%%%%%%
% Note: compile dvi with uplatex katori.tex
\documentclass[dvipdfmx, a4paper, 12pt]{utarticle}
%\documentclass{utarticle}
%\usepackage{color} % requires divpdfmx
\usepackage{CJK}
\usepackage[CJK, overlap]{ruby} %furigana support
%\usepackage{float}
%\usepackage{graphicx}
%\usepackage{multirow}
\usepackage{pdflscape}
%\usepackage[left=2cm,top=2cm,right=2cm, bottom=2cm, nohead, nofoot]{geometry}
%\usepackage[usenames,dvipsnames]{color}
%\usepackage{array}


%%%%%%%%%%%%%%%%%%%%%%%%%%%%%%%%%%%%%%
% Types
%%%%%%%%%%%%%%%%%%%%%%%%%%%%%%%%%%%%%%
%\newcolumntype{s}{>{\centering\arraybackslash}m{1cm}}
\newcolumntype{B}{>{\centering\arraybackslash}m{4.0cm}}
\newcolumntype{L}{>{\centering\arraybackslash}m{7.0cm}}


%%%%%%%%%%%%%%%%%%%%%%%%%%%%%%%%%%%%%%
% Commands
%%%%%%%%%%%%%%%%%%%%%%%%%%%%%%%%%%%%%%
%\renewcommand{\rubysize}{0.5}
\renewcommand{\rubysep}{-0.1ex}
\newcommand{\tran}[1]{{\itshape{\color{Gray}{#1}}}}

%\renewcommand{\rubysize}{0.2}
\renewcommand{\rubysep}{-0.2ex}

%%%%%%%%%%%%%%%%%%%%%%%%%%%%%%%%%%%%%%
% Start of document + settings
%%%%%%%%%%%%%%%%%%%%%%%%%%%%%%%%%%%%%%
\begin{document}
\begin{CJK*}{UTF8}{min}
\CJKtilde
\begin{landscape}

%%%%%%%%%%%%%%%%%%%%%%%%%%%%%%%%%%%%%%
% Kanji
%%%%%%%%%%%%%%%%%%%%%%%%%%%%%%%%%%%%%%
\newpage
\pagestyle{empty}
天真正傳
香取神道流
\section{居合術、表六箇條}
\noindent 草薙之剣\\
拔附之剣\\
拔討之剣\\
右剣\\
左剣\\
八方剣\\
\section{立合術、表五箇條}
\noindent 行合逆抜之太刀\\
前後千鳥之太刀\\
行合右千鳥之太刀\\
逆拔之太刀\\
拔討之太刀\\
\\
畠山???

%%%%%%%%%%%%%%%%%%%%%%%%%%%%%%%%%%%%%%
% Kanji + furigana
%%%%%%%%%%%%%%%%%%%%%%%%%%%%%%%%%%%%%%
\setcounter{section}{0}
\newpage
\pagestyle{empty}
\ruby{天真}{てんしん}\ruby{正傳}{しょうでん}\ruby{香}{かとり}\ruby{取神}{しんとう}\ruby{道流}{りゅう}
\section{\ruby{居合}{いあい}\ruby{術}{じゅつ}、\ruby{表}{おもて}\ruby{六箇條}{ろっかじょう}}
\noindent \ruby{草}{くさ}\ruby{薙}{なぎ}\ruby{之}{の}\ruby{剣}{けん}\\
\ruby{拔}{ぬき}\ruby{附}{づけ}\ruby{之}{の}\ruby{剣}{けん}\\
\ruby{拔}{ぬき}\ruby{討}{うち}\ruby{之}{の}\ruby{剣}{けん}\\
\ruby{右剣}{うけん}\\
\ruby{左剣}{さけん}\\
\ruby{八方}{はっぽう}\ruby{剣}{けん}\\
\section{\ruby{立}{たち}\ruby{合}{ai}\ruby{術}{じゅつ}、\ruby{表}{おもて}\ruby{五}{ご}\ruby{箇條}{ごかじょう}}
\noindent \ruby{行}{ゆき}\ruby{合}{あい}\ruby{逆}{ぎゃく}\ruby{抜}{ぬき}\ruby{之}{の}\ruby{太刀}{たち}\\
TBD2\\
TBD3\\
TBD4\\
TBD5\\
\\
\ruby{畠}{はたけ}\ruby{山}{やま}???

%%%%%%%%%%%%%%%%%%%%%%%%%%%%%%%%%%%%%%
% Kanji + roumaji
%%%%%%%%%%%%%%%%%%%%%%%%%%%%%%%%%%%%%%
\setcounter{section}{0}
\newpage
\pagestyle{empty}
天真正傳香取神道流・tenshin sh\={o}den katori shinto ry\={u}
\section{居合術表六箇條・iai jutsu omote rokkaj\={o}}
\noindent TBD1\\
TBD2\\
TBD3\\
TBD4\\
TBD5\\
TBD6\\
\section{}
\noindent TBD1\\
TBD2\\
TBD3\\
TBD4\\
TBD5\\
\\
畠山???・hatakeyama ???

%%%%%%%%%%%%%%%%%%%%%%%%%%%%%%%%%%%%%%
% Kanji + translation
%%%%%%%%%%%%%%%%%%%%%%%%%%%%%%%%%%%%%%
\setcounter{section}{0}
\newpage
\pagestyle{empty}
天真正傳・???
香取神道流・???
\section{居合術、表六箇條・iai techniques, 6 part table}
\noindent 草薙之剣・sword, grasscutter (cfr. Kusanagi no tsurugi in Japanese mythology)\\
抜之剣\\
抜討之剣\\
右剣・sword, right hand side\\
左剣・sword, left hand side\\
八方剣・sword, all sides (the 4 cardinal and ordinal directions)\\
\section{立合術、表五箇條・engagement techniques, 5 part table}
\noindent TBD1\\
TBD2\\
TBD3\\
TBD4\\
TBD5\\
\\
畠山???


%%%%%%%%%%%%%%%%%%%%%%%%%%%%%%%%%%%%%%
% Kanji + furigana + translation
%%%%%%%%%%%%%%%%%%%%%%%%%%%%%%%%%%%%%%
\setcounter{section}{0}
\newpage
\pagestyle{empty}
\ruby{天真}{てんしん}\ruby{正傳}{しょうでん}\ruby{香}{かとり}\ruby{取神}{しんとう}\ruby{道流}{りゅう}\\
TBD
\section{\ruby{居合}{いあい}\ruby{術}{じゅつ}\ruby{表}{おもて}\ruby{六箇條}{ろっかじょう}・iai techniques six part table}
\noindent \ruby{草薙}{くさなぎ}\ruby{之}{の}\ruby{剣}{けん}・Grasscutter sword (cfr. Kusanagi no tsurugi in Japanese mythology)\\
o
TBD2\\
\ruby{右}{う}\ruby{剣}{けん}・sword, right hand side\\
\ruby{左}{さ}\ruby{剣}{けん}・sword, left hand side\\
\ruby{八方}{はっぽう}\ruby{剣}{けん}・sword, all sides (the 4 cardinal and ordinal directions)\\
TBD6\\
\section{\ruby{立}{たち}\ruby{合}{あい}\ruby{術}{じゅつ}、\ruby{表}{おもて}\ruby{五}{ご}\ruby{箇條}{ごかじょう}・engagement techniques, 5 part table}
\noindent TBD1\\
TBD2\\
TBD3\\
TBD4\\
TBD5\\
\\
\ruby{畠}{はたけ}\ruby{山}{やま}???\\
Hatakeyama ???

%%%%%%%%%%%%%%%%%%%%%%%%%%%%%%%%%%%%%%
% Kanji + roumaji + translation
%%%%%%%%%%%%%%%%%%%%%%%%%%%%%%%%%%%%%%
\setcounter{section}{0}
\newpage
\pagestyle{empty}
\ruby{天真}{tenshin}\ruby{正傳}{sh\={o}den}\ruby{香}{katori}\ruby{取神}{shinto}\ruby{道流}{ryuu}\\
TBD
\section{\ruby{居合}{iai}\ruby{術}{jutsu}\ruby{表}{omote}\ruby{六箇條}{rokkaj\={o}}・iai techniques six part table}
\noindent TBD1\\
TBD2\\
TBD3\\
TBD4\\
TBD5\\
TBD6\\
\section{\ruby{立}{tachi}\ruby{合}{ai}\ruby{術}{jutsu}、\ruby{表}{omote}\ruby{五}{go}\ruby{箇條}{gokaj\={o}}・engagement techniques, 5 part table}
\noindent TBD1\\
TBD2\\
TBD3\\
TBD4\\
TBD5\\
\\
\ruby{畠}{hatake}\ruby{山}{yama}???・Hatakeyama ???

\end{landscape}
\end{CJK*}
\end{document}

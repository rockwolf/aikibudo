\subsection{初段・1ste dan}
\subsubsection{Algemeen}
\begin{table}[H]
\begin{center}
\begin{tabular}{c}
    \ruby{有段者}{ゆうだんしゃ}\\
    y\={u}dansha\\
    zwarte gordel/persoon die een rang heeft\\
    \hline
    \ruby{初段}{しょだん}\\
    shodan\\
    1ste graad, wordt gebruikt om een persoon met 1ste dan te omschrijven\\
\end{tabular}
\end{center}
\label{dan_1_gen}
\end{table}

\subsubsection{体捌き・Tai sabaki/Verplaatsingen}
\begin{table}[H]
\begin{center}
\begin{tabular}{c|c|c|c}
    \ruby{入}{い}り\ruby{身} & \ruby{開}{ひら}き & \ruby{大}{おお}\ruby{入}{い}り\ruby{身} & \ruby{引}{ひ}き\\
    irimi & hiraki & \={o}-irimi & hiki\\
    het inkomen van het lichaam & opening  & het groot inkomen van het lichaam & terugtrekking
    %%the entering of the body & opening  & the big entering of the body & terugtrekking
\end{tabular}
\end{center}
\label{dan_1_taisabaki}
\end{table}

\subsubsection{受身技・Ukemi waza/Valtechnieken}
\begin{table}[H]
\begin{center}
\begin{tabular}{c|c|c}
    \ruby{前}{まえ}\ruby{受身}{うけみ} & \ruby{後}{うしろ}\ruby{受身}{うけみ} & \ruby{横}{よこ}\ruby{受身}{うけみ}\\
    mae ukemi & usiro ukemi & yoko ukemi\\
    voorwaartse val & achterwaartse val & zijwaarte val
\end{tabular}
\end{center}
\label{dan_1_ukemiwaza}
\end{table}

\subsubsection{蹴り技・Trap technieken}
\begin{table}[H]
\begin{center}
\begin{tabular}{c|c|c|c|c}
    \ruby{}{}\ruby{蹴}{け}り & \ruby{回}{まわ}し\ruby{蹴}{け}り & \ruby{横}{よこ}\ruby{蹴}{け}り & \ruby{後}{うし}ろ\ruby{蹴}{け}り & \ruby{裏}{うら}\ruby{蹴}{け}り\\
    mae keri & mawashi keri & yoko keri & ushiro keri & ura keri\\
    voorwaartse trap & cirkeltrap & zijwaartse trap & achterwaarte trap & achterzijde trap (hiel trap)
\end{tabular}
\end{center}
\label{dan_1_keriwaza}
\end{table}

\subsubsection{突きと打ち技・Tsuki to uchi waza/Stoot- en slagtechnieken}
\begin{table}[H]
\begin{center}
\begin{tabular}{ccc}
    \multicolumn{3}{c}{{\bfseries\ruby{突}{つ}き・Stoot}}\\
    \hline
    \ruby{ちょく}{直}\ruby{突}{つ}き & \ruby{引}{ひ}き\ruby{突}{つ}き & \ruby{腰}{こし}\ruby{突}{つ}き\\
    choku tsuki & hiki tsuki & koshi tsuki\\
    rechte stoot & terugtrek-stoot & heup stoot\\
    \multicolumn{3}{c}{}\\
    \multicolumn{3}{c}{{\bfseries\ruby{打}{う}ち・Slag}}\\
    \hline
    \ruby{順}{じゅん}\ruby{打}{う}ち & \ruby{捻}{ひね}り\ruby{打}{う}ち & \ruby{逆}{ぎゃく}\ruby{打}{う}ち\\
    jyun uchi & hineri uchi & gyaku uchi\\
    onderdanige slag & verdraaide inworp slag & omgekeerde slag\\
    \multicolumn{3}{c}{\ruby{表}{おもて}\ruby{横}{よこ}\ruby{面}{めん}\ruby{打}{う}ち}\\
    \multicolumn{3}{c}{omote yoko men uchi}\\
    \multicolumn{3}{c}{buitenkant zijkant gezicht slag}\\
    \multicolumn{3}{c}{\ruby{裏}{うら}\ruby{打}{う}ち}\\
    \multicolumn{3}{c}{ura yoko men uchi}\\
    \multicolumn{3}{c}{achterkant zijkant gezicht slag}\\
\end{tabular}
\end{center}
\label{dan_1_tsukiuchi}
\end{table}

\subsubsection{補助運動・Hojo und\={o}/Ondersteunende beweging}
\begin{table}[H]
\begin{center}
\begin{tabular}{c|c|c|c|c}
    \ruby{握}{にぎ}り\ruby{返}{かえ}し & \ruby{捻}{ねじ}\ruby{返}{かえ}し & \ruby{押}{お}し\ruby{返}{かえ}し & \ruby{突}{つ}っ\ruby{張}{ぱ}り & \ruby{鎬}{しのぎ}\\
    nigiri kaeshi & neji kaeshi & oshi kaeshi & tsuppari & shinogi\\
    grip omkering & draaien omkering & duw omkering & stuwkracht omkering & de kanten van een lemmet
\end{tabular}
\end{center}
\label{dan_1_hojoundou}
\end{table}

\subsubsection{掴み型と手ほどき・Tsukami kata to te hodoki/Grijp kata en hand bevrijding}

\subsubsection{基本投げ技・Kihon nage waza}
\begin{table}[H]
\begin{center}
\begin{tabular}{c|p{0.5cm}|p{0.5cm}}
    \multicolumn{3}{c}{{\bfseries\ruby{突}{つ}き・TBD}}\\
    \hline
    \ruby{ちょく}{直}\ruby{突}{つ}き &  & \\
    choku tsuki & $\star$ & $\star$\\
    rechte stoot &  & \\
    \hline
    \ruby{ちょく}{直}\ruby{突}{つ}き &  & \\
    choku tsuki & $\star$ & $\star$\\
    rechte stoot &  &
\end{tabular}
\end{center}
\label{dan_1_kihonnagewaza}
\end{table}

\subsubsection{基本押さえ技・Kihon osae waza}

\subsubsection{1段の技・Technieken 1ste dan}
\begin{table}[H]
\begin{center}
\begin{tabular}{c}
    \ruby{開}{ひら}き\\
    hiraki\\
    opening\\
    \hline
    \ruby{天秤投}{てんびんな}げ\\
    tenbin nage\\
    weegschaal worp\\
    \hline
    \ruby{捻小手返}{ねじこてがえ}し\\
    neji kote kaeshi\\
    draaien onderarm omkering
\end{tabular}
\end{center}
\label{dan_1}
\end{table}

\subsubsection{和の精神・Geest van harmonie}

\subsubsection{型・Kata}
\begin{table}[H]
\begin{center}
\begin{tabular}{c|c|c|c}
    \multicolumn{4}{c}{Zonder wapen}\\
    \hline
    \ruby{八歩}{はっぽ}\ruby{拳}{けん}\ruby{型}{かた} & \ruby{突}{つ}き\ruby{打}{う}ちの\ruby{型}{かた} & \ruby{座}{すわ}り\ruby{技}{わざ}の\ruby{型}{かた} & \ruby{蹴}{け}り\ruby{五歩}{ごほ}の\ruby{型}{かた}\\
    happoken kata & tsuki uchi no kata & suwari waza no kata & keri goho no kata\\
    8-stappen vuist kata & stoot/slag kata & zittende technieken kata & trap 5 stappen kata\\
    \multicolumn{4}{c}{Met wapen}\\
    \hline
    \ruby{}{}\ruby{}{} & \ruby{}{}\ruby{}{} & \ruby{}{}\ruby{}{} & \ruby{}{}\ruby{}{}\\
    shiho giri & shiho nage & ken no kata & tanbo no kata\\
    ? & ? & ? & ?
\end{tabular}
\end{center}
\label{kata_dan_1}
\end{table}

\subsubsection{Extra informatie}

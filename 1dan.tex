\subsection{初段・1ste dan}
\subsubsection{Algemeen}
\begin{table}[H]
\begin{center}
\begin{tabular}{c}
    \ruby{有段者}{ゆうだんしゃ}\\
    y\={u}dansha\\
    \tran{zwarte gordel/persoon die een rang heeft}\\
    \hline
    \ruby{初段}{しょだん}\\
    shodan\\
    \tran{1ste graad, wordt gebruikt om een persoon met 1ste dan te omschrijven}\\
\end{tabular}
\end{center}
\label{dan_1_gen}
\end{table}

\subsubsection{体捌き・Tai sabaki/Verplaatsingen}
\begin{table}[H]
\begin{center}
\begin{tabular}{c|c|c|c}
    \ruby{入}{い}り\ruby{身} & \ruby{開}{ひら}き & \ruby{大}{おお}\ruby{入}{い}り\ruby{身} & \ruby{引}{ひ}き\\
    irimi & hiraki & \={o}-irimi & hiki\\
    \tran{het inkomen van het lichaam} & \tran{opening}  & \tran{het groot inkomen van het lichaam} & \tran{terugtrekking}
    %%the entering of the body & opening  & the big entering of the body & terugtrekking
\end{tabular}
\end{center}
\label{dan_1_taisabaki}
\end{table}

\subsubsection{受身技・Ukemi waza/Valtechnieken}
\begin{table}[H]
\begin{center}
\begin{tabular}{c|c|c}
    \ruby{前}{まえ}\ruby{受身}{うけみ} & \ruby{後}{うしろ}\ruby{受身}{うけみ} & \ruby{横}{よこ}\ruby{受身}{うけみ}\\
    mae ukemi & usiro ukemi & yoko ukemi\\
    \tran{voorwaartse val} & \tran{achterwaartse val} & \tran{zijwaarte val}
\end{tabular}
\end{center}
\label{dan_1_ukemiwaza}
\end{table}

\subsubsection{蹴り技・Trap technieken}
\begin{table}[H]
\begin{center}
\begin{tabular}{c|c|c|c|c}
    \ruby{}{}\ruby{蹴}{け}り & \ruby{回}{まわ}し\ruby{蹴}{け}り & \ruby{横}{よこ}\ruby{蹴}{け}り & \ruby{後}{うし}ろ\ruby{蹴}{け}り & \ruby{裏}{うら}\ruby{蹴}{け}り\\
    mae keri & mawashi keri & yoko keri & ushiro keri & ura keri\\
    \tran{voorwaartse trap} & \tran{cirkeltrap} & \tran{zijwaartse trap} & \tran{achterwaarte trap} & \tran{achterzijde trap (hiel trap)}
\end{tabular}
\end{center}
\label{dan_1_keriwaza}
\end{table}

\subsubsection{突きと打ち技・Tsuki to uchi waza/Stoot- en slagtechnieken}
\begin{table}[H]
\begin{center}
\begin{tabular}{ccc}
    \multicolumn{3}{c}{{\bfseries\ruby{突}{つ}き・Stoot}}\\
    \hline
    \ruby{ちょく}{直}\ruby{突}{つ}き & \ruby{引}{ひ}き\ruby{突}{つ}き & \ruby{腰}{こし}\ruby{突}{つ}き\\
    choku tsuki & hiki tsuki & koshi tsuki\\
    \tran{rechte stoot} & \tran{terugtrek-stoot} & \tran{heup stoot}\\
    \multicolumn{3}{c}{}\\
    \multicolumn{3}{c}{{\bfseries\ruby{打}{う}ち・Slag}}\\
    \hline
    \ruby{順}{じゅん}\ruby{打}{う}ち & \ruby{捻}{ひね}り\ruby{打}{う}ち & \ruby{逆}{ぎゃく}\ruby{打}{う}ち\\
    jyun uchi & hineri uchi & gyaku uchi\\
    \tran{onderdanige slag} & \tran{verdraaide inworp slag} & \tran{omgekeerde slag}\\
    \multicolumn{3}{c}{\ruby{表}{おもて}\ruby{横}{よこ}\ruby{面}{めん}\ruby{打}{う}ち}\\
    \multicolumn{3}{c}{omote yoko men uchi}\\
    \multicolumn{3}{c}{\tran{buitenkant zijkant gezicht slag}}\\
    \multicolumn{3}{c}{\ruby{裏}{うら}\ruby{打}{う}ち}\\
    \multicolumn{3}{c}{ura yoko men uchi}\\
    \multicolumn{3}{c}{\tran{achterkant zijkant gezicht slag}}\\
\end{tabular}
\end{center}
\label{dan_1_tsukiuchi}
\end{table}

\subsubsection{補助運動・Hojo und\={o}/Ondersteunende beweging}
\begin{table}[H]
\begin{center}
\begin{tabular}{c|c|c|c|c}
    \ruby{握}{にぎ}り\ruby{返}{かえ}し & \ruby{捻}{ねじ}\ruby{返}{かえ}し & \ruby{押}{お}し\ruby{返}{かえ}し & \ruby{突}{つ}っ\ruby{張}{ぱ}り & \ruby{鎬}{しのぎ}\\
    nigiri kaeshi & neji kaeshi & oshi kaeshi & tsuppari & shinogi\\
    \tran{grip omkering} & \tran{draaien omkering} & \tran{duw omkering} & \tran{stuwkracht omkering} & \tran{het overbruggen}
\end{tabular}
\end{center}
\label{dan_1_hojoundou}
\end{table}

\subsubsection{掴み型と手ほどき・Tsukami kata to te hodoki/Grijp kata en hand bevrijding}

\subsubsection{基本投げ技・Kihon nage waza}
Zie \ref{kihonnagewaza}.

\subsubsection{基本押さえ技・Kihon osae waza}
Zie \ref{kihonosaewaza}.

\subsubsection{1段の技・Technieken 1ste dan}
\begin{table}[H]
\begin{center}
\begin{tabular}{c|s|s|c|s|s}
    {\bfseries 技・waza} & {\bfseries nage} & {\bfseries osae} & {\bfseries 技・waza} & {\bfseries nage} & {\bfseries osae}\\
    \hline
    \ruby{}{}\ruby{}{} &  &  & \ruby{}{}\ruby{}{} &  & \\
    mukae daoshi & $\star$ & $\star$ & koshi nage & $\star$ & \\
    ? &  &  & ? &  & \\
    \ruby{}{}\ruby{}{} &  &  & \ruby{}{}\ruby{}{} &  & \\
    shih\={o} nage & $\star$ & $\star$ & ura ude nage & $\star$ & \\
    ? &  &  & ? &  & \\
    \ruby{}{}\ruby{}{} &  &  & \ruby{}{}\ruby{}{} &  & \\
    yuki chigae & $\star$ & $\star$ & ushiro kata otoshi & $\star$ & \\
    ? &  &  & ? &  & \\
    \ruby{捻}{ねじ}\ruby{小手}{こて}\ruby{}{} &  &  & \ruby{}{}\ruby{}{} &  & \\
    neji kote kaeshi & $\star$ & $\star$ & rofuse irimi &  & $\star$\\
    \tran{draaien onderarm omkering} &  &  & \tran{? inkomen} &  & \\
    \ruby{小手}{こて}\ruby{}{} &  &  & \ruby{}{}\ruby{転換}{てんかん} &  & \\
    kote kaeshi & $\star$ & $\star$ & rofuse tenkan &  & $\star$\\
    \tran{onderarm omkering} &  &  & \tran{? omleiden} &  & \\
    \ruby{}{}\ruby{}{} &  &  & \ruby{}{}\ruby{}{} &  & \\
    tenbin nage & $\star$ &  & ushiro neji kudaki &  & $\star$\\
    \tran{weegschaal worp} &  &  & \tran{?} &  & \\
    \ruby{}{}\ruby{}{} &  &  & \ruby{小手}{こて}\ruby{}{} &  & \\
    hachi mawashi & $\star$ &  & kote kudaki &  & $\star$\\
    \tran{?} &  &  & \tran{onderarm breken} & 
\end{tabular}
\end{center}
\label{dan_1_kihonnagewaza}
\end{table}

\subsubsection{和の精神・Geest van harmonie}

\subsubsection{型・Kata}
\begin{table}[H]
\begin{center}
    \begin{tabular}{B|B|B|B}
    \multicolumn{4}{c}{{\bfseries Zonder wapen}}\\
    \hline
    \ruby{八歩}{はっぽ}\ruby{拳}{けん}\ruby{型}{かた} & \ruby{突}{つ}き\ruby{打}{う}ちの\ruby{型}{かた} & \ruby{座}{すわ}り\ruby{技}{わざ}の\ruby{型}{かた} & \ruby{蹴}{け}り\ruby{五歩}{ごほ}の\ruby{型}{かた}\\
    happoken kata & tsuki uchi no kata & suwari waza no kata & keri goho no kata\\
    \tran{8-stappen vuist kata} & \tran{stoot/slag kata} & \tran{zittende technieken kata} & \tran{trap 5 stappen kata}\\
    \multicolumn{4}{c}{}\\
    \multicolumn{4}{c}{{\bfseries Met wapen}}\\
    \hline
    \ruby{四方}{しほう}\ruby{斬}{ぎ}り & \ruby{四方}{しほう}\ruby{投}{な}げ & \ruby{剣}{けん}の\ruby{型}{かた} & \ruby{短}{たん}\ruby{棒}{ぼう}の\ruby{型}{かた}\\
    shih\={o} giri & shih\={o} nage & ken no kata & tanbo no kata\\
    \tran{in elke richting, iemand afmaken met een zwaard} & \tran{worp in elke richting} & \tran{zwaard kata} & \tran{korte stok kata}
\end{tabular}
\end{center}
\label{kata_dan_1}
\end{table}

\subsubsection{Extra informatie}

\subsection{初段・1ste dan}
\subsubsection{Algemeen}
\begin{table}[H]
\begin{center}
\begin{tabular}{c}
    \ruby{有段者}{ゆうだんしゃ}\\
    y\={u}dansha\\
    zwarte gordel/persoon die een rang heeft\\
    \hline
    \ruby{初段}{しょだん}\\
    shodan\\
    1ste graad, wordt gebruikt om een persoon met 1ste dan te omschrijven\\
\end{tabular}
\end{center}
\label{dan_1_gen}
\end{table}

\subsubsection{体捌き・Tai sabaki/Verplaatsingen}
\begin{table}[H]
\begin{center}
\begin{tabular}{c|c|c|c}
    \ruby{}{} & \ruby{}{} & \ruby{}{} & \ruby{ひ}{引}き\\
    irimi & hiraki & \={o}-irimi & hiki\\
    ? & ? & ? & terugtrekking
\end{tabular}
\end{center}
\label{dan_1_taisabaki}
\end{table}

\subsubsection{受身技・Ukemi waza/Valtechnieken}
\begin{table}[H]
\begin{center}
\begin{tabular}{c|c|c}
    \ruby{}{} & \ruby{}{} & \ruby{}{}\\
    mae ukemi & usiro ukemi & yoko ukei\\
    ? & ? & ?
\end{tabular}
\end{center}
\label{dan_1_ukemiwaza}
\end{table}

\subsubsection{蹴り技・Trap technieken}
\begin{table}[H]
\begin{center}
\begin{tabular}{c|c|c|c|c}
    \ruby{}{} & \ruby{}{} & \ruby{}{} & \ruby{}{} & \ruby{}{}\\
    mae keri & mawashi keri & yoko keri & ushiro keri & ura keri\\
    voorwaartse trap & cirkeltrap & zijwaartse trap & achterwaarte trap & ? 
\end{tabular}
\end{center}
\label{dan_1_keriwaza}
\end{table}

\subsubsection{突きと打ち技・Tsuki to uchi waza/Stoot- en slagtechnieken}
\subsubsubsection
\begin{table}[H]
\begin{center}
\begin{tabular}{c|c|c}
    \multicolumn{3}{*}{\ruby{つ}{突}き・Stoot}\\
    \hline
    \ruby{ちょく}{直}\ruby{つ}{突}き & \ruby{ひ}{引}き\ruby{つ}{突}き & \ruby{こし}{腰}\ruby{つ}{突}き\\
    choku tsuki & hiki tsuki & koshi tsuki\\
    rechte stoot & terugtrek-stoot & heup stoot\\
    \hline
    \multicolumn{3}{*}{\ruby{う}{打}ち・Slag}\\
    \hline
    \ruby{じゅん}{順}\ruby{う}{打}ち & \ruby{ひね}{捻}り\ruby{う}{打}ち & \ruby{ぎゃく}{逆}\ruby{う}{打}ち\\
    jyun uchi & hineri uchi & gyaku uchi\\
    onderdanige slag & draai/rotatie slag & omgekeerde slag\\
    \hline
    \multicolumn{3}{*}{\ruby{おもて}{表}\ruby{よこ}{横}\ruby{めん}{面}\ruby{う}{打}ち & \ruby{うら}{裏}\ruby{う}{打}ち}\\
    \multicolumn{3}{*}{omote yoko men uchi & ura yoko men uchi}\\
    \multicolumn{3}{*}{buitenkant horizontaal gezicht slag & achterkant horizontaal gezicht slag}
\end{tabular}
\end{center}
\label{dan_1_tsukiuchi}

\subsubsection{補助運動・Hojo und\={o}/Ondersteunende beweging}
\begin{table}[H]
\begin{center}
\begin{tabular}{c|c|c|c|c}
    \ruby{にぎ}{握}り\ruby{かえ}{返}し & \ruby{}{} & \ruby{}{} & \ruby{}{} & \ruby{}{}\\
    nigiri kaeshi & neji kaeshi & oshi kaeshi & tsuppari & shinogi\\
    grip omkering & ? & ? & ? & ?
\end{tabular}
\end{center}
\label{dan_1_hojoundou}
\end{table}

\subsubsection{掴み型と手ほどき・Tsukami kata to te hodoki/Grijp kata en hand bevrijding}

\subsubsection{基本投げ技・Kihon nage waza}

\subsubsection{基本押さえ技・Kihon osae waza}

\subsubsection{1段の技・Technieken 1ste dan}
\begin{table}[H]
\begin{center}
\begin{tabular}{c}
    ? [ひらき]\\
    hiraki\\
    ?\\
    \hline
    \ruby{天秤投}{てんびんな}げ\\
    tenbin nage\\
    weegschaal worp\\
    \hline
    \ruby{捻小手返}{ねじこてがえ}し\\
    neji kote kaeshi\\
    draaien onderarm omkering
\end{tabular}
\end{center}
\label{dan_1}
\end{table}

\subsubsection{和の精神・Geest van harmonie}

\subsubsection{型・Kata}
\begin{table}[H]
\begin{center}
\begin{tabular}{lcc}
    Zonder wapen: & test & test \\
    \hline
    Met wapen: & test & test
\end{tabular}
\end{center}
\label{kata_dan_1}
\end{table}

\subsubsection{Extra informatie}

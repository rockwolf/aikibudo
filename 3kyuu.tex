\subsubsection{体捌き・Tai sabaki}
\begin{table}[H]
\begin{center}
\begin{tabular}{rl}
    ? & (\ruby{}{})\\
    ? & (omote yoko men uchi)\\
    esquives canalisation? & (?)
\end{tabular}
\end{center}
\label{kyuu_3_taisabaki}
\end{table}
\begin{center}
    1 $\leftrightarrow$ 1
\end{center}

\subsubsection{受身技・Ukemi waza}
\begin{table}[H]
\begin{center}
\begin{tabular}{c}
    \ruby{}{}\\
    ?\\
    gebroken val
\end{tabular}
\end{center}
\label{kyuu_3_ukemi_waza}
\end{table}

\subsubsection{突き技・Tsuki waza}
\begin{table}[H]
\begin{center}
\begin{tabular}{c}
    \ruby{}{}\\
    hiki tsuki\\
    ? stoot\\
    \hline
    \ruby{}{}\\
    gyaku tsuki\\
    ? stoot
\end{tabular}
\end{center}
\label{kyuu_3_tsuki_waza}
\end{table}

\subsubsection{蹴り技・Keri waza}
\begin{table}[H]
\begin{center}
\begin{tabular}{c}
    \ruby{}{}\ruby{}{}\\
    ushiro keri\\
    achterwaartse trap
\end{tabular}
\end{center}
\label{kyuu_3_keri_waza}
\end{table}

\subsubsection{補助運動・Hojo und\={o}}
\begin{table}[H]
\begin{center}
\begin{tabular}{rl}
    \ruby{}{}\ruby{}{} & (2 \ruby{}{})\\
    neji kaeshi & (2 ?)\\
    ? & (2 vormen)
\end{tabular}
\end{center}
\label{kyuu_3_hojo_undou}
\end{table}

\subsubsection{掴み型と手解き・Tsukami kata \& te hodoki}
\begin{table}[H]
\begin{center}
\begin{tabular}{c}
    \ruby{}{}\ruby{}{}\\
    ushiro uwate dori\\
    ?\\
    \hline
    \ruby{}{}\\
    ushiro shita te dori\\
    ?\\
    \hline
    \ruby{}{}\\
    ushiro eri dori\\
    ?
\end{tabular}
\end{center}
\label{kyuu_3_te_hodoki}
\end{table}

\subsubsection{追加技・Aanvullende technieken}
\begin{table}[H]
\begin{center}
\begin{tabular}{c}
    \ruby{}{}\\
    ?\\
    op elke vorm van aanval en inkomen
\end{tabular}
\end{center}
\label{kyuu_3_additional}
\end{table}

\subsubsection{基本投げ技・Kihon nage waza}
\begin{table}[H]
\begin{center}
\begin{tabular}{rl}
    \ruby{}{}\ruby{}{} & \\
    neji kote gaeshi & (tsuki ch\={u}dan)\\
    ? & (?)
\end{tabular}
\end{center}
\label{kyuu_3_kihon_nage_waza}
\end{table}

\subsubsection{基本押え技・Kihon osae waza}
\begin{table}[H]
\begin{center}
\begin{tabular}{rl}
    \ruby{}{}\ruby{}{} & \\
    yuki chigae & (ushiro eri jime)\\
    ? & (?)
\end{tabular}
\end{center}
\label{kyuu_3_kihon_osae_waza}
\end{table}

\subsubsection{歴史的技・Historische technieken}
\begin{table}[H]
\begin{center}
\begin{tabular}{rl}
    \ruby{}{}\ruby{}{} & \\
    daito ryu aikijujutsu & hiji kaeshi\\
    ? & ?
\end{tabular}
\end{center}
\label{kyuu_3_historic}
\end{table}

\subsubsection{型・Kata}
\begin{table}[H]
\begin{center}
\begin{tabular}{c}
    \ruby{}{}\ruby{}{}\\
    tsuki uchi no kata\\
    ?
\end{tabular}
\end{center}
\label{kyuu_3_kata}
\end{table}

\subsubsection{和の精神・Wa no seishin}
\begin{table}[H]
\begin{center}
\begin{tabular}{c}
    \ruby{}{}\\
    mae\\
    ?
\end{tabular}
\end{center}
\label{kyuu_3_wa_no_seishin}
\end{table}

\subsubsection{乱取り・Randori}
\begin{table}[H]
\begin{center}
\begin{tabular}{rl}
    \ruby{}{} & (\ruby{}{})\\
    j\={u} no randori & (?)\\
    ? & (soepel)
\end{tabular}
\end{center}
\label{kyuu_3_randori}
\end{table}
\begin{center}
    1 $\leftrightarrow$ 1
\end{center}

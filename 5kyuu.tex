\subsubsection{体捌き・Tai sabaki}
\begin{table}[H]
\begin{center}
\begin{tabular}{cc}
    ? & \ruby{大}{おお}? \\
    irimi & \={o} irimi\\
    ? & ? 
\end{tabular}
\end{center}
\label{kyuu_5_taisabaki}
\end{table}

\subsubsection{受身技・Ukemi waza}
\begin{table}[H]
\begin{center}
\begin{tabular}{c}
    \ruby{後}{うし}ろ\\
    ushiro\\
    achterwaarts
\end{tabular}
\end{center}
\label{kyuu_5_ukemi_waza}
\end{table}

\subsubsection{突き技・Tsuki waza}
\begin{table}[H]
\begin{center}
\begin{tabular}{c}
    \ruby{直}{ちょく}\\
    choku tsuki\\
    rechte stoot
\end{tabular}
\end{center}
\label{kyuu_5_tsuki_waza}
\end{table}

\subsubsection{蹴り技・Keri waza}
\begin{table}[H]
\begin{center}
\begin{tabular}{c}
    \ruby{前}{まえ}\ruby{蹴}{り}\\
    mae keri\\
    voorwaartse trap
\end{tabular}
\end{center}
\label{kyuu_5_keri_waza}
\end{table}

\subsubsection{補助運動・Hojo undo}
\begin{table}[H]
\begin{center}
\begin{tabular}{c}
    \ruby{}{}\ruby{}{}\\
    nigiri kaeshi\\
    ?\\
    \hline
    \ruby{}{}\\
    neji kaeshi\\
    ?
\end{tabular}
\end{center}
\label{kyuu_5_hojo_undo}
\end{table}

\subsubsection{掴み型と手解き・Tsukami kata \& te hodoki}
\begin{table}[H]
\begin{center}
\begin{tabular}{c}
    \ruby{}{}\ruby{}{}\\
    junte dori\\
    ?\\
    \hline
    \ruby{}{}\\
    dosoku te dori\\
    ?
\end{tabular}
\end{center}
\label{kyuu_5_te_hodoki}
\end{table}

\subsubsection{追加技・Aanvullende technieken}
\begin{table}[H]
\begin{center}
\begin{tabular}{c}
    \ruby{}{}\ruby{}{}\\
    ushiro kata otoshi\\
    ?
\end{tabular}
\end{center}
\label{kyuu_5_additional}
\end{table}

\subsubsection{基本投げ技・Kihon nage waza}
\begin{table}[H]
\begin{center}
\begin{tabular}{rl}
    \ruby{}{}\ruby{}{} & \\
    mukae daoshi & (ura yoko men uchi)\\
    ? & (?)
\end{tabular}
\end{center}
\label{kyuu_5_kihon_nage_waza}
\end{table}

\subsubsection{基本押え技・Kihon osae waza}
\begin{table}[H]
\begin{center}
\begin{tabular}{rl}
    \ruby{}{}\ruby{}{} & \\
    ushiro neji kudaki & (tsuki ch\={u}dan)\\
    ? & (?)
\end{tabular}
\end{center}
\label{kyuu_5_kihon_osae_waza}
\end{table}

\subsubsection{歴史的技・Historische technieken}
\begin{table}[H]
\begin{center}
\begin{tabular}{rl}
    \ruby{}{}\ruby{}{} & \\
    daito ryu aikijujutsu & ikkajo (idori)\\
    ? & ? (?)
\end{tabular}
\end{center}
\label{kyuu_5_historic}
\end{table}

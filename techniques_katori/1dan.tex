
\newcommand{\suwaristart}{
Suwari houding. Rechterknie naar voor, linkerknie links onder een hoek van 45 graden. Hand open op het rechterbeen leggen, net voor de knie en met de handpalm naar boven gericht.\\
Linkerhand houdt de saya vast, met de wijsvinger achter de tsuba vanonder (zodat hij er niet kan uitvallen) en de duim er boven op, maar net iets meer langs de binnenkant. Je moet opletten dat je, bij het trekken van het zwaard, niet in je duim snijdt. En je duim (in combinatie met je wijsvinger) kan een boost geven aan de tsuba, zodat het handvat losklikt uit de saya en eruit schiet. Daarmee win je iets extra aan tijd.\\
Rechterhand komt boven je handvat en schuift 1 handlengte richting tsuba.\\
Dan draait de rechterhand om. Dit zou de perfecte lengte moeten zijn, zodat je het zwaard ineens kunt vastpakken net onder de tsuba.\\}

\subsection{Ken jutsu}
(In de 2 rollen.)

\subsubsection{Algemene informatie}

Diegene met het hart het dichtst bij de Shinzen, is de leeraar.\\

\subsubsection{Itsutsu no tachi}

Genoeg afstand (4 matten).\\
Groeten, met katana links.\\
Dichterbij komen tot op zwaardlengte afstand.\\
Zwaard trekken als je dicht genoeg bij elkaar bent, zodat het 1 vloeiend samenkomen is naar kamae houding voor beide personen.\\
Allebei 2 maki uchis.\\
Persoon A: Drijft op met zwaard schuin om de kromming te gebruiken om een opening te cre\"{e}ren.\\
TODO: finish

\subsubsection{Nanatsu no tachi}

Genoeg afstand (4 matten).\\
Groeten, met katana links.\\
Dichterbij komen tot op zwaardlengte afstand.\\
Zwaard trekken als je dicht genoeg bij elkaar bent, zodat het 1 vloeiend samenkomen is naar kamae houding voor beide personen.\\
Allebei naar sha no kamae.\\
Persoon A gaat naar j\-{o}dan en dan aanvallend naar seigan no kamae.\\
Persoon B volgt. Vanaf nu zal persoon B opdrijven.\\
Zwaard horizontaal draaien en opdrijven, met de punt naar de keel.\\
Persoon A leunt achteruit en zet een stap naar achter.\\
Persoon B slaat op buik-hoogte, tot net voorbij het lichaam van Persoon A.\\
Persoon A zijn heup gaat achteruit bij de stap naar achter en hij weert af (niet tegen het zwaard, zijn zwaard komt voor hem ter bescherming).\\
Persoon B drijft weer zoo op en hetzelfde gebeurt.\\
Nu heeft persoon A er genoeg van en hij komt af op dezelfde manie.\\
De beweging eindigt wel op dezelfde manier: persoon B zijn zwaard net voorbij het lichaam van persoon A.\\
Dan drijft persoon A voor een 2de keer op.\\
Nu eindigt het echter zwaarda tegen zwaard. Niet horizontaal, katana's onder een hoek met de scherpe kant meer naar beneden gericht.\\
TODO: finish

\subsubsection{Kasumi no tachi}

Genoeg afstand (4 matten).\\
Groeten, met katana links.\\
Dichterbij komen tot op zwaardlengte afstand.\\
Zwaard trekken als je dicht genoeg bij elkaar bent, zodat het 1 vloeiend samenkomen is naar kamae houding voor beide personen.\\
Allebei naar j\_{o}dan no kamae.\\
TODO: finish

\subsubsection{Hakka no tachi}

TBD

\subsection{Bo jutsu}
(In de 2 rollen.)

\subsubsection{Algemene informatie}

Examens hebben elke keer 2 nieuwe kata's. D.w.z. dat de tweede kata bij elk examen langs links eindigt, zodat je daarna gemakkelijk kan groeten.\\
Dus de 2de, de 4de en de 6de eindigen links.

\subsubsection{Seri ai no bo}

houding rechts /\\
aanval, uitstappen naar links (praktisch ni, ne voet is al te ver) + bo erop\\
wapenen + uistappen naar rechts + yokomen van rechts\\
zitten + opening\\
aanval komt + uitspringen naar links en opvangen\\
wapenen + yokomen van links\\
wapenen + yokomen van rechts, maar is schijnbeweging om dan te zitten en een slag naar de enkel te doen\\
ze vangen op, schuiven naar voor om je pols aan te vallen\\
snel springen naar rechts en bo op het wapen leggen\\
achteruit smijten\\
klaar

\subsubsection{Sune hishigi no bo}

houding rechts /\\
aanval, uitstappen naar links (praktisch ni, ne voet is al te ver) + bo erop\\
zakken + mune links\\
wapenen + uistappen naar rechts + yokomen van rechts\\
zakken + tsune rechts\\
wapenen + shomen (recht naar het hoofd)\\
vangt speciaal op\\
steken, door de armen\\
achteruit smijten\\
klaar

\subsection{Naginata jutsu}
(In de 2 rollen.)

\subsubsection{Itsutsu no naginata}

TBD

\subsection{Iai jutsu Katori}

\subsubsection{Algemene informatie}

Ceremonie?\\

Zwaard wegsteken:
\begin{itemize}
\item[--] over de linker-schouder
\item[--] naar beneden gericht, totdat hij in de saya kan
\item[--] dan naar boven richten (verticaal)
\item[--] in de saya steken, de zwaartekracht helpt door het verticale aspect
\end{itemize}


\subsubsection{Kusa nagi no ken}

\suwaristart
Zwaard trekken, horizontaal. Je kapt naar de tegenstander zijn enkel of onderbeen.\\
Je hebt ondertussen je rechtervoet plat op de grond gezet, je knie net rechts van je centrum.\\
Een aanval kan terugkomen naar je rechterbeen, omdat dit van voor staat. Je reageert daarop, door je rechterbeen naar links te brengen om een zwaardslag te ontwijken. Ondertussen zal je ook opwapenen over je linkerschouder. Daarna zet je het rechterbeen terug, terwijl je een aanval naar beneden doet. Je zakt hierbij lichtjes door je benen, om mooi te flow van het zwaard te volgen.\\
Tori verdedigen.\\
Linkerhandpalm naar boven gericht, ter hoogte van het voorhoofd. Je linkerduim is lichtjes gebogen. Dit vormt een plooi in je duim, waar je de mune van het zwaard in kan leggen, zodat het niet naar opzij kan schuiven.\\
Je rechterhand heeft het handvat vast en is naar voren gericht.\\
Zwaard in de plooi van je duim naar boven schuiven om op te wapenen (en ondertussen "yeeeep" roepen).\\
Finale slag (en ondertussen "toooo" roepen). Dieper zakken.\\
Terug iets rechter kopen.\\
Zanshin.\\
Subari.\\
Not\-{o}.

\subsubsection{Nuki tsuke no ken}

\suwaristart
TBD


\subsubsection{Nuki uchi no ken}

\suwaristart
TBD

\subsubsection{Uken}

\suwaristart
TBD

\subsubsection{Saken}

\suwaristart
TBD

\subsubsection{Happoken}

TBD

\subsection{Extra}

\subsubsection{Nomenclatuur Katana onderdelen}

TBD

\subsubsection{Geschiedenis van de school}

TBD

\subsubsection{Noties over de geschiedenis van Japan en krijgskunst}

TBD


\newcommand{\suwaristart}{Suwari houding. Rechterknie naar voor, linkerknie links onder een hoek van 45 graden. Hand open op het rechterbeen leggen, net voor de knie en met de handpalm naar boven gericht.\\
Linkerhand houdt de saya vast, met de wijsvinger achter de tsuba vanonder (zodat hij er niet kan uitvallen) en de duim er boven op, maar net iets meer langs de binnenkant. Je moet opletten dat je, bij het trekken van het zwaard, niet in je duim snijdt. En je duim (in combinatie met je wijsvinger) kan een boost geven aan de tsuba, zodat het handvat losklikt uit de saya en eruit schiet. Daarmee win je iets extra aan tijd.\\
Rechterhand komt boven je handvat en schuift 1 handlengte richting tsuba.\\
Dan draait de rechterhand om. Dit zou de perfecte lengte moeten zijn, zodat je het zwaard ineens kunt vastpakken net onder de tsuba.\\}

\newcommand{\suwaristop}{Finale slag (en ondertussen ``toooo'' roepen). Dieper zakken.\\
Terug iets rechter kopen.\\
Zanshin.\\
Subari.\\
Not\-{o}.}

\newcommand{\pA}{De leraar}
\newcommand{\pB}{De leerling}
\newcommand{\pa}{de leraar}
\newcommand{\pb}{de leerling}

\subsection{Ken jutsu}
(In de 2 rollen.)

\subsubsection{Algemene informatie}

Diegene met het hart het dichtst bij de Shinzen, is de leeraar.\\

\subsubsection{Itsutsu no tachi}

Genoeg afstand (4 matten).\\
Groeten, met katana links.\\
Dichterbij komen tot op zwaardlengte afstand.\\
Zwaard trekken als je dicht genoeg bij elkaar bent, zodat het 1 vloeiend samenkomen is naar kamae houding voor beide personen.\\
Allebei 2 maki uchis.\\
\pA: Drijft op met zwaard schuin om de kromming te gebruiken om een opening te cre\"{e}ren.\\
Kleine pas, grote pas steken (niet overdreven).\\
\pB: Gaat achteruit mee. Kleine pas, grote pas, naar inno kamae. En onmiddellijk naar seigan no kamae om de steek af te weren in een snelle en beschermende beweging. Dat is de reden waarom \pa niet overdreven mag steken.\\
\pA: Zwaard weg trekken naar de linkerheup.\\
\pB: Opwapenen en aanvallen met maki uchi.\\
\pA: Rechts uitstappen en slag opvangen in te ure gasumi gasumi.\\
Allebei naar j\-{o}dan no kamae.\\
\pA: \-{o} gasumi.\\
\pB: Achteruit in inno. En dan terugkomen naar seigan no kamae (aanvallend naar de pols).\\
\pA: Zwaard wegtrekken naar de rechter-heup.\\
\pB: Ziet hoofd \pa vrij en doet een maki uchi aanval.\\
\pA: Naar links stappen en onmiddellijk maki uchi aanval naar hoofd \pb.\\
\pB: Naar links springen en opvangen k\-{o} gasumi.\\
Gevaarlijke positie: \pb kan aan \pa met een aanval buik openrijten, maar omgekeerd niet.\\
\pA: Gaan hiki achteruit en zit op linker-knie, met rechter-knie van voor.\\
\pB: Komt stap naar voor en valt maki uchi aan.\\
\pA: Tori, rechts uitstappend rechtkomen en do aanvallen.\\
\pB: Linkerheup achteruit smijten (en ook achteruit springen indien nodig), om do aanval op te vangen.\\
\pA: Onderdoor draaien, zwaard ondertussen draaien en aanvallen naar de keel van \pb. Let er op dat je de katana goed stevig in je heup vast hebt. Het is kracht tegen kracht, dus wegtrekken katana naar boven of naar onder, mag er niet voor zorgen dat je katana naar voor floept.\\
Nu is het kracht tegen kracht, voor allebei gevaarlijk. Allebei naar inno kamae.\\
Afwerken.

\subsubsection{Nanatsu no tachi}

Genoeg afstand (4 matten).\\
Groeten, met katana links.\\
Dichterbij komen tot op zwaardlengte afstand.\\
Zwaard trekken als je dicht genoeg bij elkaar bent, zodat het 1 vloeiend samenkomen is naar kamae houding voor beide personen.\\
Allebei naar sha no kamae.\\
\pA gaat naar j\-{o}dan en dan aanvallend naar seigan no kamae.\\
\pB volgt. Vanaf nu zal persoon B opdrijven.\\
Zwaard horizontaal draaien en opdrijven, met de punt naar de keel.\\
\pA leunt achteruit en zet een stap naar achter.\\
\pB slaat op buik-hoogte, tot net voorbij het lichaam van \pa.\\
\pA zijn heup gaat achteruit bij de stap naar achter en hij weert af (niet tegen het zwaard, zijn zwaard komt voor hem ter bescherming).\\
\pB drijft weer zoo op en hetzelfde gebeurt.\\
Nu heeft \pa er genoeg van en hij komt af op dezelfde manier.\\
De beweging eindigt wel op dezelfde manier: \pb zijn zwaard net voorbij het lichaam van \pa.\\
Dan drijft \pa voor een 2de keer op.\\
Nu eindigt het echter zwaarda tegen zwaard. Niet horizontaal, katana's onder een hoek met de scherpe kant meer naar beneden gericht.\\
\pA: kleine kote.\\
\pB: kleine kote.\\
\pA: grote kote.\\
\pB: grote kote.\\
\pA: \-{o}-gasumi.\\
\pB: naar inno en onmiddelijk seigan no kamae naar de pols.\\
\pA: Zwaard wegtrekken naar de linkerheup.\\
\pB: Do aanval.\\
\pA: Achteruit hiki.\\
Allebei opwapenen hoog k\-{o}-gasumi om dan hoog aan te vallen, zwaard tegen zwaard.\\
Dan allebei naar inno.\\
Afwerken.

\subsubsection{Kasumi no tachi}

Genoeg afstand (4 matten).\\
Groeten, met katana links.\\
Dichterbij komen tot op zwaardlengte afstand.\\
Zwaard trekken als je dicht genoeg bij elkaar bent, zodat het 1 vloeiend samenkomen is naar kamae houding voor beide personen.\\
Allebei naar j\_{o}dan no kamae, met linkervoet naar achter.\\
Seigan no kamae.\\
\pA: Kleine maki uchi, maar slaat onder het zwaard, naar de pols langs onder, dan naar boven (pols over). Dan met de mune het zwaard weg zweepslagen. Opwapenen en maki uchi.
\pB: Rechterpols wegtrekken om zwaard te laten passeren. Mee achteruit gaan. Soort van roofblock langs links en maki uchi aanvallen.\\
\pA: Hiki achteruit, zwaard naar linkerheup trekken.\\
\pB: Inkomen en maki uchi naar hoofd aanvallen.\\
\pA: Rooftop block langs links, rechts inkomen en maki uchi naar hoofd aanvallen.\\
\pB: Te ure gasumi opvangen.\\
\pA: Zwaard terugtrekken en steken naar de buik/lies.\\
\pB: Rechtervoet naar achter en lichaam achteruit, in jodan no kamae. Zo ga je ver genoeg weg om de steek niet in je lichaam te krijgen. Onmiddellijk zwaardslag op zwaard (do), om de aanval op te vangen en je te beschermen.\\
\pA: Maki uchi naar het hoofd.\\
\pB: Tori opvangen.\\
\pA: Naar do aanval rechterheup tegenstander overgaan.\\
\pB: Tori volgt de aanval, ondertussen je rechterheup achteruit smijten, ter ontwijking. Nu staat je punt in zijn centrum en je neemt een dreigende houding aan.\\
Allebei naar inno kamae.\\
\pB: Do aanval naar linkerheup tegenstander.\\
\pA: Opvangen en kleine men.\\
\pB: Do aanval naar linkerheup tegenstander.\\
\pA: Opvangen, \-{o}-kachi onderdoor en dan maki uchi erna.\\
\pB: Linkerheup achteruit smijten en zwaard naar linkerheup trekken om te ontwijken. Hiki achteruit tegelijkertijd, zal ook nodig zijn.\\
\pA: Men aanval.\\
\pB: Insteken langs linkerkant verdediging.\\
\pA: \-{o}-gasumi aanval. Is meer een steek naar het linkeroog van de tegenstander.\\
\pB: Zwaard als verdediging langs onder, terwijl je naar rechts onder een hoek van 45 graden uitstapt. Je staat nu in te ure gasumi.\\
\pB: Inschuiven, maar contact houden met het zwaard van de tegenstander langs onder. Als je dicht genoeg bent, kan je do aanvallen van de rechterkant uit, door je zwaard eronder uit te trekken, op te wapenen en aan te vallen.\\
\pA: Linkerheup achteruit smijten, hiki achteruit en tegelijk de slag opvangen. Afwerken met tak-tak-tak afwerking: slag erboven, naar links do en dan recht erop en op de knie zitten.\\
\pB: 2de aanval tak-tak-tak afwerking opvangen met je katana punt naar beneden, de kromming naar boven en ver genoeg van je lichaam. Je draait er ook een beetje naartoe, zodat je rechterheup ver genoeg naar achter is. Als je dan niet genoeg opvangt, is je heup tenminste ver genoeg naar achter en kan het zwaard misschien voor je buik passeren. Het nadeel is wel, dat je bij de 3de aanval van tak-tak-tak, niet meer snel genoeg bent om de finale slag op te vangen, dus je bent dood.

\subsubsection{Hakka no tachi}

Genoeg afstand (4 matten).\\
Groeten, met katana links.\\
Dichterbij komen tot op zwaardlengte afstand.\\
Zwaard trekken als je dicht genoeg bij elkaar bent, zodat het 1 vloeiend samenkomen is naar kamae houding voor beide personen.\\
Allebei achteruit. Dan terug naar mekaar lopen. Zwaarden maken contact in seigan.\\
\pA: Opdrijvend zwaard wegduwen met kromming en dan do aanval.\\
\pB: Hiki achteruit en afweren. Dan op zwaard slagen en inkomen.\\
\pA: Gaat achteruit. Heeft net slag op zwaard gekregen en zijn hoofd-regio is open.\\
\pB: Inkomen en grote men aanval.\\
\pA: Afweren links, met gekruiste voorarmen. Zwaard gaat naar beneden. Volgen en dan steken naar de lies.\\
\pB: Trekt ondertussen het zwaard even in en steekt ook.\\
\pA: Draait verder om terug in seigan te komen.\\
\pB: Staat ook in seigan.\\
Allebei een beetje afstand nemen en naar shin no kamae gaan.\\
\pA: Yokomen aanval links naar het hoofd.\\
\pB: Langs onder de pols oversnijden.\\
\pA: Hand lossen, zwaard laten passeren.\\
\pB: Slaat zwaard weg met een zweepslag.\\
\pA: Laat zwaard naar beneden zakken.\\
\pB: Maki uchi aanval. Naar het hoofd dat nu open is.\\
\pA: Afweren links, met gekruiste voorarmen. Zwaard gaat naar beneden. Volgen en dan steken naar de lies.\\
\pB: Trekt ondertussen het zwaard even in en steekt ook.\\
\pA: Draait verder om terug in seigan te komen.\\
\pB: Staat ook in seigan, maar maakt er een rooftop block beweging van, terwijl hij rechts instapt en een grote men aanval doet.\\
\pA: Tori verdedigen.\\
\pB: Yokomen rechts.\\
\pA: Tori-achtig verdedigen.\\
\pB: Do rechts.\\
\pA: Lage tori verdediging.\\
\pB: \-{O}-gasumi.\\
\pA: Achteruit, naar sh\-{o}dan no kamae. Slag recht op het puntje van het zwaard.\\
\pB: Trekt zwaard naar linkerheup, terwijl hij hiki naar achter gaat. Je handen zijn nog steeds gekruist, maar nu voor je gordel.\\
\pA: Inkomen en grote men aanval.\\
\pB: Uitstappen links en maki uchi aanval.\\
\pA: Afweren links, door in te steken met gekruiste voorarmen. Dan steek naar vooren.\\
\pB: Zwaard langs onder ter bescherming en uitstappen rechts. En yokomen rechts aanvallen.\\
\pA: Afweren langs binnenkant. En do rechts aanvallen.\\
\pB: Achteruit en afweren do aanval. Gevolgd door kleine men.\\
\pA: Achteruit gaan en nog eens kleine men. Dan terug inkomen en do links aanvallen.\\
\pB: Achteruit gaan en afweren.\\
Allebei achteruit gaan en 2 maki uchis doen.\\
\pA: Zwaard naar de linkerheup trekken en zo blijven wachten.\\
\pB: Eerst naar sh\-{o}dan no kamae. En dan inkomen om naar het hoofd aan te vallen.\\
\pA: Rooftop block uitstappen naar rechts en dan naar de nek aanvallen, terwijl je gaat zitten op je knie. ``Toooooo''

\subsection{Bo jutsu}
(In de 2 rollen.)

\subsubsection{Algemene informatie}

Examens hebben elke keer 2 nieuwe kata's. D.w.z.\ dat de tweede kata bij elk examen langs links eindigt, zodat je daarna gemakkelijk kan groeten.\\
Dus de 2de, de 4de en de 6de eindigen links.

\subsubsection{Seri ai no bo}

houding rechts /\\
aanval, uitstappen naar links (praktisch ni, ne voet is al te ver) + bo erop\\
wapenen + uistappen naar rechts + yokomen van rechts\\
zitten + opening\\
aanval komt + uitspringen naar links en opvangen\\
wapenen + yokomen van links\\
wapenen + yokomen van rechts, maar is schijnbeweging om dan te zitten en een slag naar de enkel te doen\\
ze vangen op, schuiven naar voor om je pols aan te vallen\\
snel springen naar rechts en bo op het wapen leggen\\
achteruit smijten\\
klaar

\subsubsection{Sune hishigi no bo}

houding rechts /\\
aanval, uitstappen naar links (praktisch ni, ne voet is al te ver) + bo erop\\
zakken + mune links\\
wapenen + uistappen naar rechts + yokomen van rechts\\
zakken + tsune rechts\\
wapenen + shomen (recht naar het hoofd)\\
vangt speciaal op\\
steken, door de armen\\
achteruit smijten\\
klaar

\subsection{Naginata jutsu}
(In de 2 rollen.)

\subsubsection{Itsutsu no naginata}

TBD

\subsection{Iai jutsu Katori}

\subsubsection{Algemene informatie}

Ceremonie?\\

Zwaard wegsteken:
\begin{itemize}
\item[--] over de linker-schouder
\item[--] naar beneden gericht, totdat hij in de saya kan
\item[--] dan naar boven richten (verticaal)
\item[--] in de saya steken, de zwaartekracht helpt door het verticale aspect
\end{itemize}


\subsubsection{Kusa nagi no ken}

\suwaristart
Zwaard trekken, horizontaal. Je kapt naar de tegenstander zijn enkel of onderbeen.\\
Je hebt ondertussen je rechtervoet plat op de grond gezet, je knie net rechts van je centrum.\\
Een aanval kan terugkomen naar je rechterbeen, omdat dit van voor staat. Je reageert daarop, door je rechterbeen naar links te brengen om een zwaardslag te ontwijken. Ondertussen zal je ook opwapenen over je linkerschouder. Daarna zet je het rechterbeen terug, terwijl je een aanval naar beneden doet. Je zakt hierbij lichtjes door je benen, om mooi te flow van het zwaard te volgen.\\
Tori verdedigen.\\
Linkerhandpalm naar boven gericht, ter hoogte van het voorhoofd. Je linkerduim is lichtjes gebogen. Dit vormt een plooi in je duim, waar je de mune van het zwaard in kan leggen, zodat het niet naar opzij kan schuiven.\\
Je rechterhand heeft het handvat vast en is naar voren gericht.\\
Zwaard in de plooi van je duim naar boven schuiven om op te wapenen (en ondertussen ``yeeeep'' roepen).\\
\suwaristop

\subsubsection{Nuki tsuke no ken}

\suwaristart
Zwaard trekken, horizontaal. Je kapt naar de tegenstander zijn enkel of onderbeen.\\
Linkerknie naar voren, handpalm onder mune van het zwaard en je brengt het geheel horizontaal op schouderhoogte, met de scherpe kant naar boven en de punt naar de tegenstander gericht.\\
De mune ligt op je hand. Die hand vormt een soort van kuiltje in de lengte van je handpalm. Hierin kan het zwaard naar voren schuiven, zonder dat het weg kan schuiven naar de zijkanten.\\
Naar voren schuiven en ondertussen prikken. Het zwaard schuift naar voren door het kuiltje, je hand blijft op dezelfde plaats. Arm is gestrekt en je prikt naar de tegenstander zijn keel.\\
Dan onmiddelijk terugtrekken van het zwaard.\\
Sprong en wisselen benen.\\
Wapenen (ondertussen ``yeeeep'' roepen).\\
\suwaristop

\subsubsection{Nuki uchi no ken}

\suwaristart
Springen, zo hoog als je kan.\\
Benen optrekken. Je moet een slag naar je benen ontwijken, door erover te spingen.\\
Ondertussen je zwaard trekken en wapenen. Dit gebeurt allemaal in de lucht.\\
Neerkomen door je benen terug te strekken, zodat je eerst met je voeten op de grond land. En dan zakken en op je linkerknie gaan zitten.\\
Dit gaat gepaard met een finale slag. ``Yeeeep'' roepen.\\
\suwaristop

\subsubsection{Uken}

\suwaristart
TBD

\subsubsection{Saken}

\suwaristart
TBD

\subsubsection{Happoken}

TBD

\subsection{Extra}

\subsubsection{Nomenclatuur Katana onderdelen}

TBD

\subsubsection{Geschiedenis van de school}

TBD

\subsubsection{Noties over de geschiedenis van Japan en krijgskunst}

TBD

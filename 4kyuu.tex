\subsubsection{・Tai sabaki}
\begin{table}[H]
\begin{center}
\begin{tabular}{cc}
    ? & \ruby{大}{おお}? \\
    irimi & \={o} irimi\\
    ? & ? 
\end{tabular}
\end{center}
\label{kyuu_1_taisabaki}
\end{table}

\subsubsection{・Ukemi waza}
\subsubsection{・Tsuki waza}
\subsubsection{・Keri waza}
\subsubsection{・Hojo undo}
\subsubsection{・Tsukami kata \& te hodoki}
\subsubsection{・Aanvullende technieken}
\subsubsection{基本投げ技・Kihon nage waza}
\subsubsection{基本押え技・Kihon osae waza}
\subsubsection{・Historische technieken}
\subsubsection{型・Kata}
\subsubsection{乱取り・Randori}

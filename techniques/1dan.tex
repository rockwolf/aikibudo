\subsection{体捌き・Tai sabaki/Verplaatsingen}
\subsubsection{\ruby{入}{い}り\ruby{身}{み}・irimi}
TBD

\subsubsection{\ruby{開}{ひら}き・hiraki}
TBD opening?

\subsubsection{\ruby{大}{おお}\ruby{入}{い}り\ruby{身}{み}・\={o}-irimi}
TBD

\subsubsection{\ruby{引}{ひ}き・hiki}
TBD

\subsection{受身技・Ukemi waza/Valtechnieken}
\subsubsection{\ruby{前}{まえ}\ruby{受身}{うけみ}・mae ukemi}
Voorwaartse val, vooruit kijken in de richting waarin je wil rollen, hand plaatsen en over de schouder aan dezelfde kant rollen. Bij het neerkomen, moet het been tegenover die schouder gestrekt zijn en het andere ingetrokken. Voet plat op de grond, voet van het gestrekte been zodanig draaien + tenen oprollen naar boven, dat je die voet op de grond kan zetten met de bal van de voet. Tegelijkertijd sla je op de grond om je val te breken, met de arm aan de kant van het gestrekte been. Die arm moet naast je zijn, onder een hoek van 45 graden. Bekken tegelijk ook van de gond heffen.

\subsubsection{\ruby{後}{うしろ}\ruby{受身}{うけみ}・ushiro ukemi}
Achterwaartse val vertrekt vanuit kamae, gaat naar achter, tijdens het rollen de voet naar achter steken. Niet omhoog, want bij zwaardaanvallen kan je been/voet dan afgehakt worden! Direct rechtkomen en zien dat je terug in dezelfde kamae staat. Wordt o.a.\ gebruikt als je omver geduwd wordt. Dan moet je een volgende duw kunnen tegenhouden door je kamae houding.

\subsubsection{\ruby{横}{よこ}\ruby{受身}{うけみ}・yoko ukemi}
Zijwaarte val is als een rad maken. Je staat dwars op de valrichting en je kijkt ook nog steeds voor je (dus niet naar de valrichting). Over de beide schouders rollen en neerkomen zoals bij de voorwaartse val.

\subsection{蹴り技・Trap technieken}
\subsubsection{\ruby{前}{まえ}\ruby{蹴}{け}り・mae keri}
voorwaartse trap

\subsubsection{\ruby{回}{まわ}し\ruby{蹴}{け}り・mawashi keri}
cirkeltrap

\subsubsection{\ruby{横}{よこ}\ruby{蹴}{け}り・yoko keri} 
zijwaartse trap

\subsubsection{\ruby{後}{うし}ろ\ruby{蹴}{け}り・ushiro keri}
achterwaarte trap   

\subsubsection{\ruby{裏}{うら}\ruby{蹴}{け}り・ura keri}
achterzijde trap (hiel trap)

\subsection{突きと打ち技・Tsuki to uchi waza/Stoot- en slagtechnieken ({\bfseries\ruby{突}{つ}き・Stoot})}
\subsubsection{\ruby{ちょく}{直}\ruby{突}{つ}き・choku tsuki}
Rechte stoot, cfr. de eerste stoot bij tsuki uchi no kata. Linkerhand blijft open en is ingetrokken, de rechterhand doet de slag met gesloten vuist. Vuist wordt schuin gehouden op het einde, om de ellepijp niet te verdraaien.

\subsubsection{\ruby{腰}{こし}\ruby{突}{つ}き・koshi tsuki}
Heup stoot, cfr. de tweede stoot bij tsuki uchi no kata.

\subsubsection{\ruby{引}{ひ}き\ruby{突}{つ}き・hiki tsuki}
Terugtrek-stoot, heup achteruit smijten (hiki) om een lage aanvalsimpact te ontlopen. Je rechtervoet gaat daarbij in cat-stand (rechterhiel van de grond, zodat het rechterbeen los is en vlot kan bewegen). Ondertussen breng je de linkerhand onder je rechterarm, net voor de elleboog. Met je rechtervuist, ben je ondertussen een lage slag naar de lage aanval van de tegenstander aan het uitvoeren. Dit zal in veel gevallen een knietrap zijn, dus richt die slag naar een knie van een denkbeeldige knie-aanval. De rechterarm door deze slag gestrekt, schuin naar beneden. Het is ook de derde stoot bij tsuki uchi no kata.

\subsection{突きと打ち技・Tsuki to uchi waza/Stoot- en slagtechnieken ({\bfseries\ruby{打}{う}ち・Slag})}
\subsubsection{\ruby{順}{じゅん}\ruby{打}{う}ち・jyun uchi}
onderdanige slag

\subsubsection{\ruby{捻}{ひね}り\ruby{打}{う}ち・hineri uchi}
verdraaide inworp slag

\subsubsection{\ruby{逆}{ぎゃく}\ruby{打}{う}ち・gyaku uchi}
omgekeerde slag

\subsubsection{\ruby{表}{おもて}\ruby{横}{よこ}\ruby{面}{めん}\ruby{打}{う}ち・omote yoko men uchi}
buitenkant zijkant gezicht slag

\subsubsection{\ruby{裏}{うら}\ruby{打}{う}ち・ura yoko men uchi}
achterkant zijkant gezicht slag

\subsection{補助運動・Hojo und\={o}/Ondersteunende beweging}
\subsubsection{\ruby{握}{にぎ}り\ruby{返}{かえ}し・nigiri kaeshi}
greep omkering

\subsubsection{\ruby{捻}{ねじ}\ruby{返}{かえ}し・neji kaeshi}
draaien omkering

\subsubsection{\ruby{押}{お}し\ruby{返}{かえ}し・oshi kaeshi}
duw omkering

\subsubsection{\ruby{突}{つ}っ\ruby{張}{ぱ}り・tsuppari}
stuwkracht (omkering)

\subsubsection{\ruby{鎬}{しのぎ}・shinogi}
het overbruggen

\subsection{掴み型と手ほどき・Tsukami kata to te hodoki/Greep kata en hand bevrijding (Bevrijdingen op greep)}
\subsubsection{\ruby{前}{まえ}\ruby{純}{じゅん}\ruby{手}{て}\ruby{取}{ど}り・mae jyun te dori}
Langs voor, onschuldige hand vastpakken.

\subsubsection{\ruby{前}{まえ}\ruby{純}{ぎゃく}\ruby{手}{て}\ruby{取}{ど}り・mae gyaku te dori}
Langs voor, tegenovergestelde hand vastpakken.

\subsubsection{\ruby{前}{まえ}\ruby{度}{ど}\ruby{即}{そく}\ruby{手}{て}\ruby{取}{ど}り・mae do soku te dori}
Langs voor, met precisie, onmiddellijk hand vastpakken.

\subsubsection{\ruby{前}{まえ}\ruby{両}{りょう}\ruby{手}{て}\ruby{取}{ど}り・mae ry\={o} te dori}
Langs voor, beide handen vastpakken.

\subsubsection{\ruby{前}{まえ}\ruby{両}{りょう}\ruby{手}{て}\ruby{一方}{いっぽう}\ruby{取}{ど}り・mae ry\={o} te ipp\={o} dori}
Langs voor, met beide handen 1 kant vastpakken.

\subsubsection{\ruby{前}{まえ}\ruby{胸}{むな}\ruby{取}{ど}り・mae muna dori}
Langs voor, ter hoogte van de borst vastpakken.

\subsubsection{\ruby{後}{うし}ろ\ruby{襟}{えり}\ruby{取}{ど}り・ushiro eri dori}
Langs achter, kraag vastpakken.

\subsubsection{\ruby{後}{うし}ろ\ruby{両}{りょう}\ruby{手}{て}\ruby{取}{ど}り・ushiro ry\={o} te dori}
Langs achter, beide handen vastpakken.

\subsubsection{\ruby{後}{うし}ろ\ruby{下}{した}\ruby{手}{て}\ruby{取}{ど}り・ushiro shitate dori}
Langs achter, nederige positie vastpakken (onderarm greep op riem tegenstander).

\subsubsection{\ruby{後}{うし}ろ\ruby{上}{うわ}\ruby{手}{て}\ruby{取}{ど}り・ushiro uwate dori}
Langs achter, bovenste deel vastpakken (over-arm greep).

\subsubsection{\ruby{後}{うし}ろ\ruby{片}{かた}\ruby{手}{て}\ruby{取}{ど}り\ruby{襟}{えり}\ruby{締}{し}め・ushiro katate dori eri shime}
Langs achter, 1 hand vastpakken kraag wurging.

\subsection{掴み型と手ほどき・Tsukami kata to te hodoki/Greep kata en hand bevrijding (Bijkomende grepen)}
\subsubsection{\ruby{前}{まえ}\ruby{袖}{そで}\ruby{取}{ど}り・mae sode dori}
Langs voor, mouw vastpakken.

\subsubsection{\ruby{前}{まえ}\ruby{両}{りょう}\ruby{袖}{そで}\ruby{取}{ど}り・mae ry\={o} sode dori}
Langs voor, beide mouwen vastpakken.

\subsubsection{\ruby{後}{うし}ろ\ruby{両}{りょう}\ruby{袖}{そで}\ruby{取}{ど}り・ushiro ry\={o} sode dori}
Langs achter, beide mouwen vastpakken.

\subsubsection{\ruby{前}{まえ}\ruby{組}{くみ}\ruby{突}{つ}き・mae kumi tsuki}
Langs voor, worstelen (pakken). Dit is het omgekeerde van ushiro uwate???

\subsection{基本投げ技・Kihon nage waza}
\subsubsection{\ruby{向}{むか}え\ruby{倒}{だお}し・mukae daoshi}
Naartoe gaan en neerhalen.

\subsubsection{\ruby{四方投}{しほうな}げ・shih\={o} nage}
Worp in elke richting.

\subsubsection{\ruby{行}{ゆ}き\ruby{違}{ちが}え・yuki chigae}
Elkaar kruisen.

\subsubsection{\ruby{捻}{ねじ}\ruby{小手}{こて}\ruby{返}{がえ}し・neji kote kaeshi}
Draaien onderarm omkering.

\subsubsection{\ruby{小手}{こて}\ruby{返}{がえ}し・kote kaeshi}
Onderarm omkering.

\subsubsection{\ruby{天秤}{てんびん}\ruby{投}{な}げ・tenbin nage}
Weegschaal worp.

\subsubsection{\ruby{鉢}{はち}\ruby{廻}{まわ}し\ruby{廻}{まわ}し・hachi mawashi}
Hersenpan roteren.

\subsection{基本押さえ技・Kihon osae waza}
Zie~\ref{kihonosaewaza}.

\subsection{1段の技・Technieken 1ste dan}
\begin{table}[H]
\begin{center}
\begin{tabular}{B|s|s|B|s|s}
    {\bfseries 技・waza} & {\bfseries nage} & {\bfseries osae} & {\bfseries 技・waza} & {\bfseries nage} & {\bfseries osae}\\
    \hline
     &  &  & \ruby{腰}{こし}\ruby{投}{な}げ &  & \\
     & $\star$ & $\star$ & koshi nage & $\star$ & \\
     &  &  & \tran{heup worp} &  & \\
    \hline
     &  &  & \ruby{裏}{うら}\ruby{腕}{うで}\ruby{投}{な}げ &  & \\
     & $\star$ & $\star$ & ura ude nage & $\star$ & \\
     &  &  & \tran{achterkant arm worp} &  & \\
    \hline
     &  &  & \ruby{後}{うしろ}\ruby{肩}{かた}\ruby{落}{おと}し &  & \\
     & $\star$ & $\star$ & ushiro kata otoshi & $\star$ & \\
     &  &  & \tran{achterwaarts schouder laten vallen} &  & \\
    \hline
     &  &  & \ruby{呂}{ろ}\ruby{伏}{ふ}せ\ruby{入}{い}り\ruby{身}{み} &  & \\
     & $\star$ & $\star$ & rofuse irimi &  & $\star$\\
     &  &  & \tran{(rug)wervels naar beneden buigen met inkomen van het lichaam} &  & \\
    \hline
     &  &  & \ruby{呂}{ろ}\ruby{伏}{ふ}せ\ruby{転換}{てんかん} &  & \\
     & $\star$ & $\star$ & rofuse tenkan &  & $\star$\\
     &  &  & \tran{(rug)wervels naar beneden buigen, met omleiden} &  & \\
    \hline
     &  &  & \ruby{後}{うしろ}\ruby{捻}{ねじ}\ruby{砕}{くだ}き &  & \\
     & $\star$ &  & ushiro neji kudaki &  & $\star$\\
     &  &  & \tran{achterwaarts draaien breken} &  & \\
    \hline
     &  &  & \ruby{小手}{こて}\ruby{砕}{くだ}き &  & \\
     & $\star$ &  & kote kudaki &  & $\star$\\
     &  &  & \tran{onderarm breken} & 
\end{tabular}
\end{center}
\label{dan_1_kihonnagewaza}
\end{table}

\subsection{和の精神・Geest van harmonie}

\subsection{型・Kata (zonder wapen)}
\subsubsection{\ruby{八歩}{はっぽ}\ruby{拳}{けん}\ruby{型}{かた}・happoken kata}
8-stappen vuist kata

\subsubsection{\ruby{突}{つ}き\ruby{打}{う}ちの\ruby{型}{かた}・tsuki uchi no kata}
De {\textit stoot/slag kata}, maakt gebruik van alle slagen die beschreven zijn bij het onderdeel tsuki en uchi waza. Je begint met klaar te staan, je vuisten naast je lichaam. Zeg luid en duidelijk "tsuki uchi no kata", waarna je 1s pauzeert. Dan val je, met een rechte rug, zachtjes naar voren. Wanneer je niet anders kan als je rechtervoet naar voren plaatsen, geef je ineens een rechte stoot (choku tsuki). Linker hand dichtbij trekken, cfr.\ karate stoten. Verschil met karate: de linkerhand blijft open, zoals bij de gewone kamae houding. Je eindigt die stoot door stevig op je benen te staan, zwaartepunt verlagen en rechte rug houden. Daarna, je rechtervoet een beetje naar rechts schuiven, om dan met je linkerhand een heup-stoot (koshi tsuki) te geven. Daarna kom je terug recht, je trekt je terug in hiki (rechter been geplooid naar achter, de hiel van je rechtervoet is lichtjes van de grond). Tegelijkertijd, geef je ter verdediging een terugtrek-stoot (hiki tsuki). Je linkerhand breng je daarbij onder je rechterarm (net achter de elleboog).

\subsubsection{\ruby{座}{すわ}り\ruby{技}{わざ}の\ruby{型}{かた}・suwari waza no kata}
zittende technieken kata

\subsubsection{\ruby{蹴}{け}り\ruby{五歩}{ごほ}の\ruby{型}{かた}・keri goho no kata}
trap 5 stappen kata

\subsection{型・Kata (met wapen)}

%    \hline
%    \ruby{四方}{しほう}\ruby{斬}{ぎ}り & \ruby{四方}{しほう}\ruby{投}{な}げ & \ruby{剣}{けん}の\ruby{型}{かた} & \ruby{短}{たん}\ruby{棒}{ぼう}の\ruby{型}{かた}\\
%    shih\={o} giri & shih\={o} nage & ken no kata & tanbo no kata\\
%    \tran{in elke richting, iemand afmaken met een zwaard} & \tran{worp in elke richting} & \tran{zwaard kata} & \tran{korte stok kata}

\subsection{Extra informatie}

\subsection{体捌き・Tai sabaki/Verplaatsingen}
\subsubsection{\ruby{入}{い}り\ruby{身}{み}・irimi}
TBD

\subsubsection{\ruby{開}{ひら}き・hiraki}
TBD opening?

\subsubsection{\ruby{大}{おお}\ruby{入}{い}り\ruby{身}{み}・\={o}-irimi}
TBD

\subsubsection{\ruby{引}{ひ}き・hiki}
Achteruit springen, vanuit kamae. Voorste voet vertrekt eerst. Gaat naar achteren. Achterste voet vertrekt dan ook, om plaats te maken voor de voorste voet. Eindigen in kamae.

\subsubsection{\ruby{}{}・enka}
In kamae staan. Draaien over je hielen, zodat je kamae naar de andere kant gericht is.

\subsubsection{\ruby{}{}・mawate}
Wisselen van voet. Als de rechtervoet van voor staat in je kamae, dan breng je de linkervoet naar voor en de rechter naar achter.

\subsection{受身技・Ukemi waza/Valtechnieken}
\subsubsection{\ruby{前}{まえ}\ruby{受身}{うけみ}・mae ukemi}
Voorwaartse val, vooruit kijken in de richting waarin je wil rollen, hand plaatsen en over de schouder aan dezelfde kant rollen. Bij het neerkomen, moet het been tegenover die schouder gestrekt zijn en het andere ingetrokken. Voet plat op de grond, voet van het gestrekte been zodanig draaien + tenen oprollen naar boven, dat je die voet op de grond kan zetten met de bal van de voet. Tegelijkertijd sla je op de grond om je val te breken, met de arm aan de kant van het gestrekte been. Die arm moet naast je zijn, onder een hoek van 45 graden. Bekken tegelijk ook van de gond heffen.

\subsubsection{\ruby{後}{うしろ}\ruby{受身}{うけみ}・ushiro ukemi}
Achterwaartse val vertrekt vanuit kamae, gaat naar achter, tijdens het rollen de voet naar achter steken. Niet omhoog, want bij zwaardaanvallen kan je been/voet dan afgehakt worden! Direct rechtkomen en zien dat je terug in dezelfde kamae staat. Wordt o.a.\ gebruikt als je omver geduwd wordt. Dan moet je een volgende duw kunnen tegenhouden door je kamae houding.

\subsubsection{\ruby{横}{よこ}\ruby{受身}{うけみ}・yoko ukemi}
Zijwaarte val is als een rad maken. Je staat dwars op de valrichting en je kijkt ook nog steeds voor je (dus niet naar de valrichting). Over de beide schouders rollen en neerkomen zoals bij de voorwaartse val.

\subsection{蹴り技・Trap technieken}
\subsubsection{\ruby{前}{まえ}\ruby{蹴}{け}り・mae keri}
voorwaartse trap

\subsubsection{\ruby{回}{まわ}し\ruby{蹴}{け}り・mawashi keri}
cirkeltrap

\subsubsection{\ruby{横}{よこ}\ruby{蹴}{け}り・yoko keri} 
zijwaartse trap

\subsubsection{\ruby{後}{うし}ろ\ruby{蹴}{け}り・ushiro keri}
achterwaarte trap   

\subsubsection{\ruby{裏}{うら}\ruby{蹴}{け}り・ura keri}
achterzijde trap (hiel trap)

\subsection{突きと打ち技・Tsuki to uchi waza/Stoot- en slagtechnieken ({\bfseries\ruby{突}{つ}き・Stoot})}
\subsubsection{\ruby{ちょく}{直}\ruby{突}{つ}き・choku tsuki}
Rechte stoot, cfr.\ de eerste stoot bij tsuki uchi no kata. Linkerhand blijft open en is ingetrokken, de rechterhand doet de slag met gesloten vuist. Vuist wordt schuin gehouden op het einde, om de ellepijp niet te verdraaien.

\subsubsection{\ruby{腰}{こし}\ruby{突}{つ}き・koshi tsuki}
Heup stoot, cfr.\ de tweede stoot bij tsuki uchi no kata.

\subsubsection{\ruby{引}{ひ}き\ruby{突}{つ}き・hiki tsuki}
Terugtrek-stoot, heup achteruit smijten (hiki) om een lage aanvalsimpact te ontlopen. Je rechtervoet gaat daarbij in cat-stand (rechterhiel van de grond, zodat het rechterbeen los is en vlot kan bewegen). Ondertussen breng je de linkerhand onder je rechterarm, net voor de elleboog. Met je rechtervuist, ben je ondertussen een lage slag naar de lage aanval van de tegenstander aan het uitvoeren. Dit zal in veel gevallen een knietrap zijn, dus richt die slag naar een knie van een denkbeeldige knie-aanval. De rechterarm door deze slag gestrekt, schuin naar beneden. Het is ook de derde stoot bij tsuki uchi no kata.

\subsection{突きと打ち技・Tsuki to uchi waza/Stoot- en slagtechnieken ({\bfseries\ruby{打}{う}ち・Slag})}
\subsubsection{\ruby{順}{じゅん}\ruby{打}{う}ち・jyun uchi}
onderdanige slag

\subsubsection{\ruby{捻}{ひね}り\ruby{打}{う}ち・hineri uchi}
verdraaide inworp slag

\subsubsection{\ruby{逆}{ぎゃく}\ruby{打}{う}ち・gyaku uchi}
omgekeerde slag

\subsubsection{\ruby{表}{おもて}\ruby{横}{よこ}\ruby{面}{めん}\ruby{打}{う}ち・omote yoko men uchi}
buitenkant zijkant gezicht slag

\subsubsection{\ruby{裏}{うら}\ruby{打}{う}ち・ura yoko men uchi}
achterkant zijkant gezicht slag

\subsection{補助運動・Hojo und\={o}/Ondersteunende beweging}
\subsubsection{\ruby{握}{にぎ}り\ruby{返}{かえ}し・nigiri kaeshi}
Greep omkering.\\
\noindent \textbf{Standaard vorm:}\\
\noindent Door de knie\:{e}n gaan. Schouders recht zetten. Hand naar beneden strekken in centrum tegenstander. Naar links draaien en je arm in je centrum trekken, dus je elleboog komt naar binnen. Rechtkomen, door je benen te strekken. Die doen al het werk. Je houdt ondertussen je hand voor je, in je centrum. Als je recht bent gekomen, met je bovenlichaam krachtig naar rechts draaien, om in  kamai te komen. Hierdoor breekt de kracht van de tegenstander. Je linkerhand moet steeds mee volgen, ter bescherming. Op het einde, als je recht bent gekomen en je in kamae hebt gezet, dan pak je met beide handen de voorarm van de tegenstander over. Daarna is het zijn beurt.
\textbf{Tweede vorm:}\\
\noindent Wordt gebruikt als de tegenstander te sterk is, want de standaard vorm kan je blokkeren als je wil. Irimi inkomen en je arm uit de greep plooien, tot die voor je centrum komt. Tai sabaki afwerken en op het einde in een vlotten beweging in kamae komen. Dit laatste moet ook weer met een krachtige beweging van het bovenlichaam. Voorarm tegenstander overpakken en dan is het zijn beurt.

\subsubsection{\ruby{捻}{ねじ}\ruby{返}{かえ}し・neji kaeshi}
draaien omkering

\subsubsection{\ruby{押}{お}し\ruby{返}{かえ}し・oshi kaeshi}
duw omkering

\subsubsection{\ruby{突}{つ}っ\ruby{張}{ぱ}り・tsuppari}
stuwkracht (omkering)

\subsubsection{\ruby{鎬}{しのぎ}・shinogi}
het overbruggen

\subsection{掴み型と手ほどき・Tsukami kata to te hodoki/Greep kata en hand bevrijding (Bevrijdingen op greep)}
\subsubsection{\ruby{前}{まえ}\ruby{純}{じゅん}\ruby{手}{て}\ruby{取}{ど}り・mae jyun te dori}
Langs voor, onschuldige hand vastpakken. Denk aan de cirkel van je centrum. Bekken rechtzetten en de cirkelvorm houden.
Je draait zijwaarts naar links, zo ver, dat je linkervoet recht gezet wordt in de richting waarin je draait. Je rechterelleboog komt onder de arm van de tegenstander. Ondertussen pak je over met je linkerhand en geef je gas. Atami. Arm naar beneden duwen, voor zijn centrum, om hem uit evenwicht te brengen. Atami. Hand plaatsen ter controle. Andere hand lossen. Achteruit gaan in kamae.

\subsubsection{\ruby{前}{まえ}\ruby{純}{ぎゃく}\ruby{手}{て}\ruby{取}{ど}り・mae gyaku te dori}
Langs voor, tegenovergestelde hand vastpakken. Arm omhoog duwen, meer onder de elleboog (aan de triceps, bovenarm) en naar voor gaan met heel je lichaamsgewicht. Timing is hier belangrijk, je moet vertrekken als de aanval begint. Voor de hand-beweging om los te komen, cfr.\ een deurklink gebruiken.\\
\noindent Opmerking: Als je er technieken op moet doen, zie dat je de arm ver genoeg laat komen. Vooral bij mukae daoshi is dat belangrijk. Anders blijft de tegenstander te recht staan en krijg je de kop niet achteruit geklinkt. Als hij uit evenwicht komt, leunt hij naar voor en komt zijn hoofd lager.

\subsubsection{\ruby{前}{まえ}\ruby{度}{ど}\ruby{即}{そく}\ruby{手}{て}\ruby{取}{ど}り・mae do soku te dori}
Langs voor, met precisie, onmiddellijk hand vastpakken.

\subsubsection{\ruby{前}{まえ}\ruby{両}{りょう}\ruby{手}{て}\ruby{取}{ど}り・mae ry\={o} te dori}
Langs voor, beide handen vastpakken.

\subsubsection{\ruby{前}{まえ}\ruby{両}{りょう}\ruby{手}{て}\ruby{一方}{いっぽう}\ruby{取}{ど}り・mae ry\={o} te ipp\={o} dori}
Langs voor, met beide handen 1 kant vastpakken.

\subsubsection{\ruby{前}{まえ}\ruby{胸}{むな}\ruby{取}{ど}り・mae muna dori}
Langs voor, ter hoogte van de borst vastpakken.

\subsubsection{\ruby{後}{うし}ろ\ruby{襟}{えり}\ruby{取}{ど}り・ushiro eri dori}
Langs achter, kraag vastpakken.

\subsubsection{\ruby{後}{うし}ろ\ruby{両}{りょう}\ruby{手}{て}\ruby{取}{ど}り・ushiro ry\={o} te dori}
Langs achter, beide handen vastpakken.

\subsubsection{\ruby{後}{うし}ろ\ruby{下}{した}\ruby{手}{て}\ruby{取}{ど}り・ushiro shitate dori}
Langs achter, nederige positie vastpakken (onderarm greep op riem tegenstander).

\subsubsection{\ruby{後}{うし}ろ\ruby{上}{うわ}\ruby{手}{て}\ruby{取}{ど}り・ushiro uwate dori}
Langs achter, bovenste deel vastpakken (over-arm greep). Eerst de 2 armen naar voor (richting de horizon), om plaats te maken. Dan pas beide handen in de opening steken. Snel alles omhoog duwen en draaien. Draai zo ver, dat hij uit evenwicht blijft. Je moet met je bovenlichaam zijn centrum innemen. Afwerken door de arm die niet over het centrum draaide, te gebruiken om zijn arm ter hoogte van de elleboog (binnenkant), uit zijn centrum te duwen (cfr.\ 3 hoeken van een driehoek, waarbij zij 2 voeten 1 zijde vormen en het denkbeeldige punt voor hem, de top van de driehoek).

\subsubsection{\ruby{後}{うし}ろ\ruby{片}{かた}\ruby{手}{て}\ruby{取}{ど}り\ruby{襟}{えり}\ruby{締}{し}め・ushiro katate dori eri shime}
Langs achter, 1 hand vastpakken kraag wurging. Niet te ver uitstappen (20cm is genoeg). Naar opzij, zelfs een beetje naar achter (en NIET naar voor!). Inkomen met het bovenlichaam eerst, cfr.\ in tekenfilms, wanneer ze vertrekkeen om te lopen en 1 been is nog niet op de grond.\\
Vb.\ links: Stap links uit naar achteren (niet te ver), rechter voet vertrekt ook en komt van de grond. Dan ineens terugkomen, dus die voet terugzetten. Maar als ge dat doet, eerst uw bovelichaam laten komen. Zo komt heel uw gewicht eerst. Eindigen in een kamai, die de tegenstander met je volledige lichaamsgewicht uit evenwicht brengt en bijna volledig omver duwt. Afwerken door de arm te pakken en hem uit evenwicht te duwen (cfr.\ driehoek).

\subsection{掴み型と手ほどき・Tsukami kata to te hodoki/Greep kata en hand bevrijding (Bijkomende grepen)}
\subsubsection{\ruby{前}{まえ}\ruby{袖}{そで}\ruby{取}{ど}り・mae sode dori}
Langs voor, mouw vastpakken.

\subsubsection{\ruby{前}{まえ}\ruby{両}{りょう}\ruby{袖}{そで}\ruby{取}{ど}り・mae ry\={o} sode dori}
Langs voor, beide mouwen vastpakken.

\subsubsection{\ruby{後}{うし}ろ\ruby{両}{りょう}\ruby{袖}{そで}\ruby{取}{ど}り・ushiro ry\={o} sode dori}
Langs achter, beide mouwen vastpakken.

\subsubsection{\ruby{前}{まえ}\ruby{組}{くみ}\ruby{突}{つ}き・mae kumi tsuki}
Langs voor, worstelen (pakken). Dit is het omgekeerde van ushiro uwate???

\subsection{基本投げ技・Kihon nage waza}
\subsubsection{\ruby{向}{むか}え\ruby{倒}{だお}し・mukae daoshi}
Naartoe gaan en neerhalen. Ze pakken je bij je kraag. Naar voor wandelen, niet te snel, niet te traag. De aanvaller mag niet voelen dat er tegenstand is. Er zijn 2 mogelijke acties (voor aanval met rechter arm):\\
\begin{enumerate}
    \item{Niet ver uitstappen, maar omdraaien naar rechts.}
    \item{Uitstappen naar rechts.}
\end{enumerate}
Bij de \textit{eerste} actie, blijf je dichtbij zijn centrum. Je doet een slag met je rechter arm, alsof een zwaard van boven naar beneden geslagen wordt. Tegelijkertijd moet je goed inkomen en de tegenstander zijn plaats innemen. Door jin het centrum te blijven, komt hij nu uit evenwicht. Je doet met de slag zijn rechterarm naar beneden en dan moet je zijn onderarm/pols tegen je rechter-heup klemmen. Tai sabaki afweken (\={o}-irimi met je linkerbeen). Breng je rechterarm onder zijn kin, zodat hij in je arm loopt door de beweging die ontstaat als je hem meeneemt in de tai sabaki beweging. Zie dat zijn kin op je biceps ligt. Steek nu je arm omhoog. Hierdoor draait zijn hoofd weg en kun je het daarna gemakkelijker klemmen. Arm naar je linkerborst brengen en zijn hoofd klemmen. Loslaten en op tijd je knie weghalen (tenzij je de tegenstander zijn rug wil breken).\\
Bij de \textit{tweede} actie, ga je uitstappen naar rechts. Hoe moet het dan verder?


\subsubsection{\ruby{四方投}{しほうな}げ・shih\={o} nage}
Worp in elke richting.

\subsubsection{\ruby{行}{ゆ}き\ruby{違}{ちが}え・yuki chigae}
Elkaar kruisen.

\subsubsection{\ruby{捻}{ねじ}\ruby{小手}{こて}\ruby{返}{がえ}し・neji kote kaeshi}
Draaien onderarm omkering.
Arm laag tegen het dijbeen houden!
Moet op 2 manieren: standaard en tenkan.
Standaard, wil zeggen dat de \={o}-irimi horizontaal eindigt. Dus je staat recht tegenover de tegenstander. De neji kote kaeshi doe je dan naar je rechter-knie (bij starten langs links inkomen irimi). 
Tenkan, wil zeggen dat de \={o}-irimi wordt doorgezet. Je zet de neji kote kaeshi aan, maar bent dan gericht naar de richting waarin de tegenstander gaat vliegen.

\subsubsection{\ruby{天秤}{てんびん}\ruby{投}{な}げ・tenbin nage}
Weegschaal worp.

\subsubsection{\ruby{鉢}{はち}\ruby{廻}{まわ}し\ruby{廻}{まわ}し・hachi mawashi}
Hersenpan roteren.

\subsubsection{\ruby{腰}{こし}\ruby{投}{な}げ・koshi nage}
Heup worp.

\subsection{基本押さえ技・Kihon osae waza}
\subsubsection{\ruby{呂}{ろ}\ruby{伏}{ふ}せ\ruby{入}{い}り\ruby{身}{み}・rofuse irimi}
(rug)wervels naar beneden buigen met inkomen van het lichaam.

\subsubsection{\ruby{呂}{ろ}\ruby{伏}{ふ}せ\ruby{転換}{てんかん}・rofuse tenkan}
(rug)wervels naar beneden buigen, met omleiden.

\subsubsection{\ruby{後}{うしろ}\ruby{捻}{ねじ}\ruby{砕}{くだ}き・ushiro neji kudaki}
Achterwaarts draaien breken.

\subsubsection{\ruby{小手}{こて}\ruby{砕}{くだ}き・kote kudaki}
Onderarm breken.

\section{Bijkomende technieken}
\subsubsection{\ruby{裏}{うら}\ruby{腕}{うで}\ruby{投}{な}げ・ura ude nage}
Achterkant arm worp.

\subsubsection{\ruby{小手}{こて}\ruby{返}{がえ}し・kote kaeshi}
Onderarm omkering.

\subsubsection{\ruby{後}{うしろ}\ruby{肩}{かた}\ruby{落}{おと}し・ushiro kata otoshi}
Achterwaarts schouder laten vallen.

\subsection{和の精神・Geest van harmonie}

\subsection{型・Kata (zonder wapen)}
\subsubsection{\ruby{八歩}{はっぽ}\ruby{拳}{けん}\ruby{型}{かた}・happoken kata}
8-stappen vuist kata

\subsubsection{\ruby{突}{つ}き\ruby{打}{う}ちの\ruby{型}{かた}・tsuki uchi no kata}
De \textit{stoot/slag kata}, maakt gebruik van alle slagen die beschreven zijn bij het onderdeel tsuki en uchi waza. Je begint met klaar te staan, je vuisten naast je lichaam. Zeg luid en duidelijk ``tsuki uchi no kata'', waarna je 1s pauzeert. Dan val je, met een rechte rug, zachtjes naar voren. Wanneer je niet anders kan als je rechtervoet naar voren plaatsen, geef je ineens een rechte stoot (choku tsuki). Linker hand dichtbij trekken, cfr.\ karate stoten. Verschil met karate: de linkerhand blijft open, zoals bij de gewone kamae houding. Je eindigt die stoot door stevig op je benen te staan, zwaartepunt verlagen en rechte rug houden. Daarna, je rechtervoet een beetje naar rechts schuiven, om dan met je linkerhand een heup-stoot (koshi tsuki) te geven. Daarna kom je terug recht, je trekt je terug in hiki (rechter been geplooid naar achter, de hiel van je rechtervoet is lichtjes van de grond). Tegelijkertijd, geef je ter verdediging een terugtrek-stoot (hiki tsuki). Je linkerhand breng je daarbij onder je rechterarm (net achter de elleboog).\\
\\
Timing:\\
1001 1002 1003 iaaaa 1001 1002 1003 ieeep\\
1001 1002 1003 iaaaa 1001 1002 1003 toooo\\
\\
Extra details:
\begin{itemize}
    \begin{item}
    Starten met je vuisten naast je dijen te houden, met je voeten een beetje uit elkaar, en naar voren te vallen. Let erop dat je voet schuift over de grond en dat je niet op de grond stampt.
    \end{item}
    \begin{item}
    Bij de zweep-slag, moet je goed je bovenlichaam gebruiken on kracht te genereren. Je slaagt dan goed door, waardoor je een beetje scheef eindigt. Daarna zet je je eerst terug in een rechte kamae.
    \end{item}
    \begin{item}
    Eindigen door je recht te zetten. Je vuist wordt een open hand, die komt naast je lichaam, terwijl je naar achter gaat. Twee passen naar achter en je moet terug staan waar je begonnen was. Dan groeten.
    \end{item}
    \begin{item}
    Groeten = handen op je quadriceps (dus effecties van voor en niet langs de zijkant) en buigen, tot je vingertoppen do bovenkant van je knie\:{e}n raken.
    Je ogen volgen mee de buiging, dus naar beneden kijken tijdens het buigen.
    \end{item}
\end{itemize}
\\
Als oefening tussen persoon A en B:\\
Beginnen met rechte stoot langs rechts.\\
A: rechte stoot\\
B: afweren met de bovenkant van de rechter-voorarm langs de binnenkant naar links bewegend (gedan barai of zoiets)\\
A: heup-stoot\\
B: linkerarm naar de andere kant bewegen om af te weren met de buitenkant van dezelfde voorarm.\\
B: met de linkerarm een stoot naar voor geven... rechte stoot die vanuit de hara komt.\\
A: Hiki tsuki... je doet de stoot hierbij over de arm van de tegenstander, tegen zijn borst. Dicht genoeg bij blijven, zijn stoot kan je net niet raken, omdat jou stoot tegen zijn borstkas, je lichaam net iets langer maakt.\\
A: onderdanige slag\\
B: opzij gaan en afweren\\
A: inkomen met elleboog\\
B: nogmaals opzij gaan en afweren\\
B: inkomen om de tegenstander vast te pakken\\
A: Zweep-slag (ura) in de zijkant (ribben) van de tegenstander, terwijl je achteruit gaat.

\subsubsection{\ruby{座}{すわ}り\ruby{技}{わざ}の\ruby{型}{かた}・suwari waza no kata}
zittende technieken kata

\subsubsection{?? katori - ceremonie}
start: 6 matten afstand
Voeten uit mekaar (want ze hadden een harnas aan).
Groeten naar mekaar, met de handen open, van voor op de dijbenen.
Draaien naar de meester.
Voeten in mekaar (rechter in de linker) en groeten, ook met de handen open, van voor op de dijbenen.
Draaien naar mekaar.
Zitten in seiza, passief.
Groeten: linker hand op de grond, rechterhand volgt.
Naar mekaar blijven kijken en buigen.
Rechterhand maar even op de grond en dan terug naar boven.
Terug de rug recht brengen.
Tenen actief.
Naar mekaar komen, met de handen in de lies. Deze keer, vuisten maken rond de duim. Hierdoor kan de tegenstander moeilijker je arm vastpakken, want dan schuift hij er af.
Een dikke 2 vuisten afstand houden. Dat mag 30cm zijn, dat werkt gemakkelijker.
\\
einde: Tenen actief en in chico terug achteruit. Weer 6 matten afstand.
Tenen passief. Op dezelfde manier groeten.
Rechtstaan.
Op dezelfde manier groeten.
Weer draaien naar de meester en groeten.

\subsubsection{?? katori - 1ste}
aanvaller:
Rechte slag naar beneden met rechterhand.
Rechterknie stapt rechts uit.
verdediger:
Vertrekken wanneer de aanval nog moet beginnen.
Dubbele hand (op mekaar) schiet naar voor om op de achterkant van de bovenarm (boven de oksel) te duwen.
Die dubbele hand schuift tegelijk een beetje naar boven richting de elleboogholte.
Hierdoor wordt de aanvaller uit evenwicht geduwd.
Over arm schuiven om de hand vast te pakken en die op je rechterknie te leggen (je rechterknie was rechts uitgestapt).
Linkerhand ter hoogte van de elleboog.
2 details:
* de vingertoppen van de linkerhand moeten naar boven gericht zijn.
* duwen \textit{op} de elleboog, het is de bedoeling om dat gewricht kapot te doen.
Je duwt de elleboog richting je rechtervoet.
Rechterhand blijft vasthouden. Rechterknie gaat naar voren. Hier kan je op elke moment de arm nog breken.
Hand plat op de grond leggen, dus niet getorst.
Linkerhand duwt op de bovenarm, achter de elleboog, ter controle.
Linkerknie op de schouder leggen, ter controle.
Nu kan je de linkerhand gebruiken om een "tooo" te doen.
Met linkerhand terug achter de elleboog vastpinnen.
Linkerknie achteruit.
Linkerhand laat los.
Rechterknie achteruit.
Rechterhand houd nog 1 seconde vast, om controle te laten zien, voor het geval hij nog leeft.
Rechterhand laat los.
In kamae.
Andere laten rechtkomen.
Vuisten rond duimen maken in de lies.
Aanvaller gaat eerst naar zijn plaats.
Dan verdediger.

\subsubsection{?? katori - 2de}
Aanvaller:
Vastpakken vest met linkerhand en roepen.
Zijn rechterknie stapt rechts uit.
Verdediger:
2 stoten naar boven:
* linkervuist raakt zijn elleboog
* rechtervuist op zijn kin
Detail: Kom een beetje naar voor, om de overhand te krijgen in het evenwicht. Dus overtuigend inkomen.
Dan komen de 2 vuisten naar beneden in nog een dubbele slag:
* linkervuist slaat in een drukpunt schep beweging (van biceps aanvaller naar voorarm), om hem naar voor te laten buigen
* rechtervuist slaat op zijn neusbrug
Tegelijkertijd mae keri met je linkerbeen.
Linkerbeen ineens naar achter zetten, ver genoeg.
Ondertussen draait je lichaam mee en dat zal zijn arm in een klem brengen.
Je linkerhand mag helpen, weer met de vingertoppen naar boven gericht, maar in eendebakkes C vastpakken.
Afwerking, zoals bij de eerste.

\subsubsection{?? katori - 3de}
Aanvaller: Rechte stoot. Rechterknie stapt rechts uit.
Verdediger: Tegelijkertijd als beide handen in kamae naar voor en een shinogi doen.
Let er op dat die shinogi van ver genoeg vertrekt en naar je lichaam komt.
Arm aanvaller dus niet wegduwen.
Vuistslag laten doorgaan en weer laten neerkomen en afwerken.
Aanvaller moet normaal gezien recht liggen, in de richting van de vuistslag.

\subsubsection{?? katori - 4de}
Kuruma daoshi, de wiel neerhaling.
Aanvaller: Rechterhand slaat met open hand, van achter beginnen in een cirkelvormige beweging.
Verdediger: Voor de aanval start, met open hand de aanval blokkeren op de schouder. Paf!
Eigenlijk niet helemaal blokkeren. Je zal met de linkerhand ineens de rechterarm van de tegenstander \textit{begeleiden},
om die cirkelbeweging verder te zetten. Daardoor komt de tegenstander ten val en land hij op zijn rug.
Je linkerknie is intussen naar achter gedraait en omhoog gekomen.
Met rechterhand een "tooo" doen.
De aanvaller moet zijn gezicht beschermen en ook zijn hoofd van de grond houden.
Afwerken door je rechterhand onder de rechterarm van de tegenstander te steken. Dan die arm over hem duwen, ter hoogte van de bovenarm langs de buitenkant.
Even houden, ter controle.
In kamae achteruit.
Aanvaller moet wegrollen van de verdediger.
Komt dan recht.
Kamae verdediger worden 2 vuisten rond de duimen.
Aanvaller gaat terug op zijn plaats.
Verdediger ook.

\subsubsection{?? katori - 5de}
Gekruist wurgen.
2 atamis in de ribben.
Niet uitstappen!
1 arm naar beneden, 1 naar boven.
De nadruk op de arm naar beneden, dat is de belangrijkste.
Neerleggen en mee indraaien.
Je doet eigenlijk een \={o}-irimi.
Je eindigt dan langs de zijkant, terwijl zijn arm vertikaal is.
Hij heeft je immers nog vast.
Je linkerhand doet de afwerking.
Je rechterhand schuift er dan onder, duwt de arm over de aanvallen en houdt even controle.
Dan achteruit gaan in kamae en wachten.
De aanvaller draait weg van de verdediger en zet zich recht.

\subsubsection{suwari waza no kata}
Voor mekaar zitten.
1.
Je wordt vastgepakt. Rechterhand gebruiken om naast hem te bewegen en hem in een cirkelvormige beweging uit evenwicht te brengen. Als hij valt, bijschuiven en zijn linkerarm over je 2 knie\:{e}n leggen, pols vasthouden ter controle. Slag naar het gezicht (hij moet het gezicht beschermen). Die linkerarm tussen jou en de tegenstander steken. Tegenstander strekt 1 been, andere knie geplooid over dat gestrekt been. Zo moet hij ook wegrollen, van de andere weg. Terug naar je plaats gaan in chico. Andere zal ondertussen rechtkomen en ook terug naar zijn plaats gaan.
2.
Idem, maar nu zal de linkerhand ook voor een balansverstoring zorgen, naar boven gericht. Hand in kamae naar binnen gericht (dus niet de arm vastpakken). De rest is hetzelfde.

\subsubsection{\ruby{蹴}{け}り\ruby{五歩}{ごほ}の\ruby{型}{かた}・keri goho no kata}
trap 5 stappen kata

\subsection{型・Kata (met wapen)}

%    \hline
%    \ruby{四方}{しほう}\ruby{斬}{ぎ}り & \ruby{四方}{しほう}\ruby{投}{な}げ & \ruby{剣}{けん}の\ruby{型}{かた} & \ruby{短}{たん}\ruby{棒}{ぼう}の\ruby{型}{かた}\\
%    shih\={o} giri & shih\={o} nage & ken no kata & tanbo no kata\\
%    \tran{in elke richting, iemand afmaken met een zwaard} & \tran{worp in elke richting} & \tran{zwaard kata} & \tran{korte stok kata}
\subsubsection{tanto no kata}
Groeten + Tanto no kata zeggen.
Begin zoals bij katori: chidori vertrekken en tanto trekken.
In een grote beweging naar het hoofd slagen. Let er op dat de punt goed naar voor is,
want daar snij je het gezicht mee.
Die beweging stopt niet in kamae houding, maar gaat verder om een 8-snijbeweging te maken, zoals in eskrima.
Beginnen langs rechts en er op letten dat je niet te ver naar achter gaat. Je tanto moet steeds voor je blijven.
Snijden, weer met de punt goed naar voor, in de richting van de gi-lijnen.
Na de 8-beweging, doe je onmiddellijk een steek. Je linkervoet verplaatst \textit{niet}. Je leunt gewoon naar voor om te steken.
Dit was de eerste blok. Kiai mag hier.
Dan de tanto overpakken naar earth grip. Je houdt het boven je hoofd, met de snijkant naar links.
Even wachten en dan haken naar rechts, zogezegd in het linkeroog om dat neusbeen open te rijten. Terugkomen met de punt goed naar voor om de hals open te snijden. Linkervoet gaat achterom je rechterbeen, terwijl je naar voor blijft kijken met getorst lichaam. Ondertussen heb je je linkerhand onder de tanto geplaatst. Steken naar voren, in de buik (onder de borstplaat). En dan de tanto klinken, zodat je het geheel naar boven kunt duwen, onder de borst, tot in de longen.
Dit was de tweede blok. Kiai mag hier.
Linkerhand lost de ondersteuning en pakt de tanto over, net onder de tsuba. Uit de torsing draaien en blok 1 begint terug, maar nu langs de andere kant.
Dat wordt de derde blok. Kiai op het einde.
Dan weer earth grip en neusbeen openrijten. Dat is de vierde blok. Kiai.
Laatste keer uit de torsing draaien en snijden in het gezicht, met de punt goed naar voren. Laatste kiai.
2 passen achteruit en notou.

\subsubsection{ken no kata}
Zwaard trekken. In kamae voor mekaar staan. Allebei 1 maki uchi doen.
Ene geeft opening (te ura gasumi). De andere valt men aan.
Rechts/links werken.
Men R, men L.
Do R, do L.
\={O}-gasumi. De andere gaat naar \textit{in no kamae}. Dan valt hij aan. Achteruit gaan en zwaard bijtrekken.
De andere geeft nog een maki uchi. Je zet je linkerbeen naar achter en gaat in gedan no kamae staan.
Nu sta je terug zoals in't begin en is het jouw beurt om opening te laten.
Ieder doet zo 1 kant.

\subsubsection{bo no kata}
Is bijna hetzelfde als ken no kata, met enkele verschillen.
Bo in de helft vastpakken. Rechterhandpalm naar boven gericht, daar ligt de bo in.
Groeten, zonder dat de bo beweegt.
Bo naar voren schuiven en tegoei vastpakken voor aan te vallen. Linkerhand pakt over, langs boven. Achteruit zwieren en een maki uchi doen. Nu sta je in begin-positie en doe je iets gelijkaardig aan de bewegingen van ken no kata.
Men R, men L.
Do R, do L.
Een bo is lang, dus kan je ook laag gaan.
Sune R, sune L. Goed door de knie\:{e}n zakken.
I.p.v.\={o}-gasumi, doen we hier mateage. Dat kunnen we zowel R als L doen.
Een slag komt, naar rechts springen en opvangen. Opzij duwen en dreigen naar de keel.
Allebei een maki uchi. En dan sta je weer op dezelfde manier.
En die blok nog eens herhalen, maar met de rollen gewisseld.

\subsection{Extra informatie}

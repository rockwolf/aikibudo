\subsection{体捌き・Tai sabaki/Verplaatsingen}
\subsubsection{\ruby{入}{い}り\ruby{身}{み}・irimi}
Schuin naar voren, uit de aanvalslijn. Heup iets naar achter. Cfr.\ de stage, waar we het tegen de muur gedaan hebben.\\
De voorste voet vertrekt eerst. Rug recht houden. Soepel bewegen.\\
De armen moeten in kamae houding staan. Bij irimi links, armen hoog houden en zo doen alsof je een aanval afweert langs rechts.\\
Terug achteruit gaan op dezelde manier en eindigen in de startpositie, in kamae houding.\\
Er zijn duidelijk 3 stappen:
\begin{itemize}
\item Naar voor schuiven.
\item Heup draaien en wat naar achter smijten, terwijl je je voet intrekt. Geen gewicht op de rechtervoet.
\item Rechtervoet naar achter strekken, om terug in een kamae houding te eindigen.
\end{itemize}

\subsubsection{\ruby{開}{ひら}き・hiraki}
Starten in kamae houding.\\
Schuin naar achter gaan, en doen alsof je een aanval al afwerend laat binnenkomen.\\
Eindigen, door in kamae-houding terug naar de beginpositie schuiven.

\subsubsection{\ruby{大}{おお}\ruby{入}{い}り\ruby{身}{み}・\={o}-irimi}
De armen moeten in kamae houding staan. Bij \={o}-irimi links, rechterarm laten zakken (en je linker gaat ter compensatie omhoog), alsof je een aanval afweert en de controle houdt tijdens het draaien.\\
Eindigen in omgekeerde richting, terug in kamae houding. Ook de kamae houding is tegengesteld.

\subsubsection{\ruby{引}{ひ}き・hiki}
Achteruit springen, vanuit kamae. Voorste voet vertrekt eerst. Gaat naar achteren. Achterste voet vertrekt dan ook, om plaats te maken voor de voorste voet. Eindigen in kamae.

\subsubsection{\ruby{}{}・nagashi}
\textbf{Nagashi voorste voet:}\\
Voorste voet draait naar binnen. Dat is de nieuwe richting waarin we gaan staan.\\Achterste voet beweegt naar voor, in de lijn van je lichaam. Tijdens het kruisen van je ander been, draai je je om en schuif je de voet verder naar achter (zodat je weer in een kamae eindigt).\\
Zo eindig je dan in de tegenovergestelde richting. Altijd kamae houding aanhouden.\\
\\
\textbf{Nagashi achterste voet:}\\
Je achterste voet draait eerst naar binnen. Diezelfde voet dient nu als draaipunt, waar je voorste voet rond draait.\\
Je maakt een volledige toer van 360 graden met je voorste voet (die nu je achterste geworden is), maar dicht bij de as van je lichaam. Die toer mag niet te breed, want daar verlies je stabiliteit en snelheid mee.\\
Het geheel is meer een voze beweging, waar je een langere tijd op 1 been draait. Nagashi voorste voet is handiger, maar deze kan ook zijn nut hebben in bepaalde situaties.

\subsubsection{\ruby{}{}・enka}
In kamae staan. Draaien over je hielen, zodat je kamae naar de andere kant gericht is.

\subsubsection{\ruby{}{}・mawate}
Wisselen van voet. Als de rechtervoet van voor staat in je kamae, dan breng je de linkervoet naar voor en de rechter naar achter.

\subsection{受身技・Ukemi waza/Valtechnieken}
\subsubsection{\ruby{前}{まえ}\ruby{受身}{うけみ}・mae ukemi}
Voorwaartse val, vooruit kijken in de richting waarin je wil rollen, hand plaatsen en over de schouder aan dezelfde kant rollen. Bij het neerkomen, moet het been tegenover die schouder gestrekt zijn en het andere ingetrokken. Voet plat op de grond, voet van het gestrekte been zodanig draaien + tenen oprollen naar boven, dat je die voet op de grond kan zetten met de bal van de voet. Tegelijkertijd sla je op de grond om je val te breken, met de arm aan de kant van het gestrekte been. Die arm moet naast je zijn, onder een hoek van 45 graden. Bekken tegelijk ook van de gond heffen.

\subsubsection{\ruby{後}{うしろ}\ruby{受身}{うけみ}・ushiro ukemi}
Achterwaartse val vertrekt vanuit kamae, gaat naar achter, tijdens het rollen de voet naar achter steken. Niet omhoog, want bij zwaardaanvallen kan je been/voet dan afgehakt worden! Direct rechtkomen en zien dat je terug in dezelfde kamae staat. Wordt o.a.\ gebruikt als je omver geduwd wordt. Dan moet je een volgende duw kunnen tegenhouden door je kamae houding.

\subsubsection{\ruby{横}{よこ}\ruby{受身}{うけみ}・yoko ukemi}
Zijwaarte val is als een rad maken. Je staat dwars op de valrichting en je kijkt ook nog steeds voor je (dus niet naar de valrichting). Over de beide schouders rollen en neerkomen zoals bij de voorwaartse val.

\subsection{蹴り技・Keri waza/Trap technieken}
\subsubsection{\ruby{前}{まえ}\ruby{蹴}{け}り・mae keri}
Voorwaartse trap.

\subsubsection{\ruby{回}{まわ}し\ruby{蹴}{け}り・mawashi keri}
Cirkeltrap.

\subsubsection{\ruby{横}{よこ}\ruby{蹴}{け}り・yoko keri}
zijwaartse trap

\subsubsection{\ruby{後}{うし}ろ\ruby{蹴}{け}り・ushiro keri}
Achterwaarte trap.

\subsubsection{\ruby{裏}{うら}\ruby{蹴}{け}り・ura keri}
Achterzijde trap (hiel trap).

\subsection{突きと打ち技・Tsuki to uchi waza/Stoot- en slagtechnieken ({\bfseries\ruby{突}{つ}き・Stoot})}
\subsubsection{\ruby{直}{ちょく}\ruby{突}{つ}き・choku tsuki}
Rechte stoot, cfr.\ de eerste stoot bij tsuki uchi no kata. Linkerhand blijft open en is ingetrokken, de rechterhand doet de slag met gesloten vuist. Vuist wordt schuin gehouden op het einde, om de ellepijp niet te verdraaien.

\subsubsection{\ruby{腰}{こし}\ruby{突}{つ}き・koshi tsuki}
Heup stoot, cfr.\ de tweede stoot bij tsuki uchi no kata.

\subsubsection{\ruby{引}{ひ}き\ruby{突}{つ}き・hiki tsuki}
Terugtrek-stoot, heup achteruit smijten (hiki) om een lage aanvalsimpact te ontlopen. Je rechtervoet gaat daarbij in cat-stand (rechterhiel van de grond, zodat het rechterbeen los is en vlot kan bewegen). Ondertussen breng je de linkerhand onder je rechterarm, net voor de elleboog. Met je rechtervuist, ben je ondertussen een lage slag naar de \textit{buik} van de tegenstander aan het uitvoeren. De rechterarm door deze slag gestrekt, schuin naar beneden. Het is ook de derde stoot bij tsuki uchi no kata.

\subsection{突きと打ち技・Tsuki to uchi waza/Stoot- en slagtechnieken ({\bfseries\ruby{打}{う}ち・Slag})}
\subsubsection{\ruby{順}{じゅん}\ruby{打}{う}ち・jyun uchi}
Onderdanige slag. Let erop dat je bovenarm horizontaal is, parallel met de grond. Je voorarm moet onder een hoek van 45 graden naar voor gericht zijn. Met de kneukels op de neusbrug slaan, zodat de tegenstander op zijn knieën zakt. Vandaar ook de naam \textit{onderdanige slag}.

\subsubsection{\ruby{捻}{ひね}り\ruby{打}{う}ち・hineri uchi}
Verdraaide inworp slag. Je hebt net de onderdanige slag gedaan en draait nu de heup in de andere richting.\\
Je linkervuist komt naar beneden, dicht bij je lichaam.\\
Je rechtervuist breng je naar je linkerschouder, terwijl je een elleboog-stoot geeft.\\
Bij die elleboog-stoot, gebruik je de beweging van je lichaam om kracht bij te zetten. En daarbij beweeg je ook nog eens heel je lichaam naar voren, om dit momentum ook te gebruiken om extra kracht bij te zetten.

\subsubsection{\ruby{逆}{ぎゃく}\ruby{打}{う}ち・gyaku uchi}
Omgekeerde slag. Je trekt hier je lichaam terug open, na de hineri uchi.\\
Je rechter-vuist geeft een slag naar de zijkant van de tegenstander (vb.\ de zijkant van zijn gezicht). Cfr.\ een zweepslag.\\
Belangrijk is dat je je goed open trekt: rechterarm beweegt naar rechts, linker-schouder gaat een beetje naar links-achter.

\subsubsection{\ruby{表}{おもて}\ruby{横}{よこ}\ruby{面}{めん}\ruby{打}{う}ち・omote yoko men uchi}
Buitenkant zijkant gezicht slag.

\subsubsection{\ruby{裏}{うら}\ruby{打}{う}ち・ura yoko men uchi}
Achterkant zijkant gezicht slag.

\subsection{補助運動・Hojo und\={o}/Ondersteunende beweging}
\subsubsection{\ruby{握}{にぎ}り\ruby{返}{かえ}し・nigiri kaeshi}
Greep omkering.\\
\textbf{Standaard vorm:}\\
Door de knie\"{e}n gaan. Schouders recht zetten. Hand naar beneden strekken in centrum tegenstander. Naar links draaien en je arm in je centrum trekken, dus je elleboog komt naar binnen. Rechtkomen, door je benen te strekken. Die doen al het werk. Je houdt ondertussen je hand voor je, in je centrum. Als je recht bent gekomen, met je bovenlichaam krachtig naar rechts draaien, om in  kamai te komen. Hierdoor breekt de kracht van de tegenstander. Je linkerhand moet steeds mee volgen, ter bescherming. Op het einde, als je recht bent gekomen en je in kamae hebt gezet, dan pak je met beide handen de voorarm van de tegenstander over. Daarna is het zijn beurt.\\
\textbf{Tweede vorm:}\\
Wordt gebruikt als de tegenstander te sterk is, want de standaard vorm kan je blokkeren als je wil. Irimi inkomen en je arm uit de greep plooien, tot die voor je centrum komt. Tai sabaki afwerken en op het einde in een vlotten beweging in kamae komen. Dit laatste moet ook weer met een krachtige beweging van het bovenlichaam. Voorarm tegenstander overpakken en dan is het zijn beurt.

\subsubsection{\ruby{捻}{ねじ}\ruby{返}{かえ}し・neji kaeshi}
Draaien omkering.

\subsubsection{\ruby{押}{お}し\ruby{返}{かえ}し・oshi kaeshi}
Duw omkering.

\subsubsection{\ruby{突}{つ}っ\ruby{張}{ぱ}り・tsuppari}
Stuwkracht (omkering).

\subsubsection{\ruby{鎬}{しのぎ}・shinogi}
Het overbruggen.\\
Tegenover mekaar staan, rechtervoet naar voor. De voorste voeten staan op gelijke hoogte, ongeveer 1 voet-breedte ertussen.\\
Je gaat de slag halen, dus op tijd vertrekken. Handpalm naar je gezicht gericht, zodat je duim zijn arm controleert. Dit moet achter zijn elleboog, zodat je controle over zijn arm hebt. Niet ervoor, want dan kan hij een elleboogstoot geven.\\
Slag laten komen, dicht naast je gezicht. Steeds de arm begeleiden.\\
Je draait ook met je heup, zodat je parallel met de arm eindigt. Je benen zijn getorst en staan in een soort van chidori-positie.\\
Van daaruit, begin je met een vuistslag die de overhand neemt. Je probeert naar zijn gezicht te slagen. Goed indraaien met de heup. Je eindigt dan weer in de eerste houding.

\subsection{掴み型と手ほどき・Tsukami kata to te hodoki/Greep kata en hand bevrijding (Bevrijdingen op greep)}
\subsubsection{\ruby{前}{まえ}\ruby{純}{じゅん}\ruby{手}{て}\ruby{取}{ど}り・mae jyun te dori}
Langs voor, onschuldige hand vastpakken. Denk aan de cirkel van je centrum. Bekken rechtzetten en de cirkelvorm houden.
Je draait zijwaarts naar links, zo ver, dat je linkervoet recht gezet wordt in de richting waarin je draait. Je rechterelleboog komt onder de arm van de tegenstander. Ondertussen pak je over met je linkerhand en geef je gas. Atami. Arm naar beneden duwen, voor zijn centrum, om hem uit evenwicht te brengen. Atami. Hand plaatsen ter controle. Andere hand lossen. Achteruit gaan in kamae.

\subsubsection{\ruby{前}{まえ}\ruby{純}{ぎゃく}\ruby{手}{て}\ruby{取}{ど}り・mae gyaku te dori}
Langs voor, tegenovergestelde hand vastpakken. Arm omhoog duwen, meer onder de elleboog (aan de triceps, bovenarm) en naar voor gaan met heel je lichaamsgewicht. Timing is hier belangrijk, je moet vertrekken als de aanval begint. Voor de hand-beweging om los te komen, cfr.\ een deurklink gebruiken.\\
Opmerking: Als je er technieken op moet doen, zie dat je de arm ver genoeg laat komen. Vooral bij mukae daoshi is dat belangrijk. Anders blijft de tegenstander te recht staan en krijg je de kop niet achteruit geklinkt. Als hij uit evenwicht komt, leunt hij naar voor en komt zijn hoofd lager.

\subsubsection{\ruby{前}{まえ}\ruby{度}{ど}\ruby{即}{そく}\ruby{手}{て}\ruby{取}{ど}り・mae do soku te dori}
Langs voor, met precisie, onmiddellijk hand vastpakken.

\subsubsection{\ruby{前}{まえ}\ruby{両}{りょう}\ruby{手}{て}\ruby{取}{ど}り・mae ry\={o} te dori}
Langs voor, beide handen vastpakken.

\subsubsection{\ruby{前}{まえ}\ruby{両}{りょう}\ruby{手}{て}\ruby{一方}{いっぽう}\ruby{取}{ど}り・mae ry\={o} te ipp\={o} dori}
Langs voor, met beide handen 1 kant vastpakken.

\subsubsection{\ruby{前}{まえ}\ruby{胸}{むな}\ruby{取}{ど}り・mae muna dori}
Langs voor, ter hoogte van de borst vastpakken.

\subsubsection{\ruby{後}{うし}ろ\ruby{襟}{えり}\ruby{取}{ど}り・ushiro eri dori}
Langs achter, kraag vastpakken.

\subsubsection{\ruby{後}{うし}ろ\ruby{両}{りょう}\ruby{手}{て}\ruby{取}{ど}り・ushiro ry\={o} te dori}
Langs achter, beide handen vastpakken.

\subsubsection{\ruby{後}{うし}ろ\ruby{下}{した}\ruby{手}{て}\ruby{取}{ど}り・ushiro shitate dori}
Langs achter, nederige positie vastpakken (onderarm greep op riem tegenstander).

\subsubsection{\ruby{後}{うし}ろ\ruby{上}{うわ}\ruby{手}{て}\ruby{取}{ど}り・ushiro uwate dori}
Langs achter, bovenste deel vastpakken (over-arm greep). Eerst de 2 armen naar voor (richting de horizon), om plaats te maken. Dan pas beide handen in de opening steken. Snel alles omhoog duwen en draaien. Draai zo ver, dat hij uit evenwicht blijft. Je moet met je bovenlichaam zijn centrum innemen. Afwerken door de arm die niet over het centrum draaide, te gebruiken om zijn arm ter hoogte van de elleboog (binnenkant), uit zijn centrum te duwen (cfr.\ 3 hoeken van een driehoek, waarbij zij 2 voeten 1 zijde vormen en het denkbeeldige punt voor hem, de top van de driehoek).

\subsubsection{\ruby{後}{うし}ろ\ruby{片}{かた}\ruby{手}{て}\ruby{取}{ど}り\ruby{襟}{えり}\ruby{締}{し}め・ushiro katate dori eri shime}
Langs achter, 1 hand vastpakken kraag wurging. Niet te ver uitstappen (20cm is genoeg). Naar opzij, zelfs een beetje naar achter (en NIET naar voor!). Inkomen met het bovenlichaam eerst, cfr.\ in tekenfilms, wanneer ze vertrekkeen om te lopen en 1 been is nog niet op de grond.\\
Vb.\ links: Stap links uit naar achteren (niet te ver), rechter voet vertrekt ook en komt van de grond. Dan ineens terugkomen, dus die voet terugzetten. Maar als ge dat doet, eerst uw bovelichaam laten komen. Zo komt heel uw gewicht eerst. Eindigen in een kamai, die de tegenstander met je volledige lichaamsgewicht uit evenwicht brengt en bijna volledig omver duwt. Afwerken door de arm te pakken en hem uit evenwicht te duwen (cfr.\ driehoek).

\subsection{掴み型と手ほどき・Tsukami kata to te hodoki/Greep kata en hand bevrijding (Bijkomende grepen)}
\subsubsection{\ruby{前}{まえ}\ruby{袖}{そで}\ruby{取}{ど}り・mae sode dori}
Langs voor, mouw vastpakken.

\subsubsection{\ruby{前}{まえ}\ruby{両}{りょう}\ruby{袖}{そで}\ruby{取}{ど}り・mae ry\={o} sode dori}
Langs voor, beide mouwen vastpakken.

\subsubsection{\ruby{後}{うし}ろ\ruby{両}{りょう}\ruby{袖}{そで}\ruby{取}{ど}り・ushiro ry\={o} sode dori}
Langs achter, beide mouwen vastpakken.

\subsubsection{\ruby{前}{まえ}\ruby{組}{くみ}\ruby{突}{つ}き・mae kumi tsuki}
Langs voor, armen kruisen. Dit is een begeleiding om naar neji kote kaeshi over te gaan.\\
De aanvaller komt naar voren met 1 schouder, en wil je omver kieperen of je linkerhand tegenhouden, zodat je je zwaard niet kan trekken. De aanvaller kiest dus de kant waarop de techniek moet uitgevoerd worden, want je moet werken op de kant waar de schouder naar voren gericht is. Langs die kant is de arm immers verder.\\
Hiki doen en tegelijkertijd je handen kruisen, rond zijn aanvallende pols. Je begeleid dan met je rechterhand, zodat je naar neji kote kaeshi kunt overgaan. De linkerhand houdt de arm tegen, dat hij niet weg kan.\\
Let er op, dat je niet te hoog gaat. In je centrum werken. Als de arm omhoog staat, vastpakken met je linkerhand en met de rechterhand (die ondertussen klaar staat aan de pols), doe je een neji kote kaeshi. Je doet die tenkan, zodat de aanvaller wordt gegooid in de richting waarin hij aanvalt.\\
Extra opmerking: \textit{In het begin bij hiki, niet uit de aanvalslijn gaan naar zijn buitenkant. Anders lukt die tenkan beweging niet meer, omdat je in de weg staat. Ook niet naar de binnenkant, want dan kan je de begeleiding niet goed meer doen. Recht naar achter dus.}

\subsection{基本投げ技・Kihon nage waza/Basis worp technieken}
\subsubsection{\ruby{向}{むか}え\ruby{倒}{だお}し・mukae daoshi}
Naartoe gaan en neerhalen. Ze pakken je bij je kraag. Naar voor wandelen, niet te snel, niet te traag. De aanvaller mag niet voelen dat er tegenstand is. Er zijn 2 mogelijke acties (voor aanval met rechter arm):\\
\begin{enumerate}
    \item{Niet ver uitstappen, maar omdraaien naar rechts.}
    \item{Uitstappen naar rechts.}
\end{enumerate}
Bij de \textit{eerste} actie, blijf je dichtbij zijn centrum. Je doet een slag met je rechter arm, alsof een zwaard van boven naar beneden geslagen wordt. Tegelijkertijd moet je goed inkomen en de tegenstander zijn plaats innemen. Door jin het centrum te blijven, komt hij nu uit evenwicht. Je doet met de slag zijn rechterarm naar beneden en dan moet je zijn onderarm/pols tegen je rechter-heup klemmen. Tai sabaki afweken (\={o}-irimi met je linkerbeen). Breng je rechterarm onder zijn kin, zodat hij in je arm loopt door de beweging die ontstaat als je hem meeneemt in de tai sabaki beweging. Zie dat zijn kin op je biceps ligt. Steek nu je arm omhoog. Hierdoor draait zijn hoofd weg en kun je het daarna gemakkelijker klemmen. Arm naar je linkerborst brengen en zijn hoofd klemmen. Loslaten en op tijd je knie weghalen (tenzij je de tegenstander zijn rug wil breken).\\
Bij de \textit{tweede} actie, ga je uitstappen naar rechts. Hoe moet het dan verder?


\subsubsection{\ruby{四方投}{しほうな}げ・shih\={o} nage}
Worp in elke richting.

\subsubsection{\ruby{行}{ゆ}き\ruby{違}{ちが}え・yuki chigae}
Elkaar kruisen:\\
Tsuki komt naar je gezicht.\\
Shinogi doen. Timing is belangrijk, je gaat zijn arm eigenlijk halen, op het moment dat hij begint met zijn aanval.\\
Ondertussen \={o}-irimi inkomen.\\
Een klem zetten op zijn elleboog. Je laat hem rond je draaien, met jezelf als pivotatiepunt.\\
Nog voor zijn linkervoet terug op de grond komt, leg je de linkerzijde van je hoofd op de binnenkant van zijn bovenarm en duw je hem daarmee opzij. Hierdoor blijft hij uit evenwicht. Je stapt dan ineens onderdoor, waardoor de binnenkant van zijn elleboog op je schouder komt te liggen. Blijf wandelen in de richting die ervoor zorgt dat hij uit evenwicht blijft.\\
Dan draai je verder en trek je je linkerbeen in en zet je het naar achter (tai sabaki van het onder de arm draaien, afwerken). Je linkerhand heeft zijn pols vast. Zijn pols moet dicht tegen je lichaam zijn, zodat je gemakkelijk kracht kan uitoefenen om te torsen.\\
Opmerkingen:
\begin{itemize}
    \item Yuki chigae wordt met 1 hand uitgevoerd. De tweede hand mag erbij komen als extra controle, maar die zet geen kracht.
    \item NIET eindigen na draaien, zodat je voor de tegenstander eindigt. Je moet er achter staan! (anders 0 punten)
\end{itemize}
De eindpositie torst de pols, als een schroef, richting zijn oksel. Pols dicht bij je centrum houden als je dat doet, waar je sterk bent. Nu staat hij normaal gezien op zijn tippen.\\
Daarna de pols naar beneden wringen, zodat hij naar de grond gaat. Maar start in het begin met de focus op het naar de oksel richten. Anders lukt dat naar beneden brengen niet goed.

\subsubsection{\ruby{捻}{ねじ}\ruby{小手}{こて}\ruby{返}{がえ}し・neji kote kaeshi}
Draaien onderarm omkering:\\
Arm laag tegen het dijbeen houden!\\
Moet op 2 manieren: standaard en tenkan.\\
Standaard, wil zeggen dat de \={o}-irimi horizontaal eindigt. Dus je staat recht tegenover de tegenstander. De neji kote kaeshi doe je dan naar je rechter-knie (bij starten langs links inkomen irimi).
Tenkan, wil zeggen dat de \={o}-irimi wordt doorgezet. Je zet de neji kote kaeshi aan, maar bent dan gericht naar de richting waarin de tegenstander gaat vliegen.

\subsubsection{\ruby{天秤}{てんびん}\ruby{投}{な}げ・tenbin nage}
Weegschaal worp:\\
De tegenstander komt naar je toe gelopen en pakt je vest vast, langs beide kanten, ter hoogtje van je bovenarmen.\\
Je stopt zijn inkomen, door hem een uppercut te geven, die op training voor de veiligheid een open hand wordt (met de handpalm naar hem gericht).\\
Tegelijkertijd doe je een tai sabaki \={o}-irimi, die je niet volledig afwerkt. Als die afgewerkt is, sta je bijna terug recht voor de tegenstander.\\
Op het einde, je elleboog op zijn binnenarm plaatsen en met de snijkant van je hand, hem naar beneden brengen.\\
Dan laat je hem rechtkomen, terwijl hij rechtkomt plaats je je bovenarm onder zijn bovenarm. Je hand wijst naar boven en je handpalm is naar achteren gericht. Zijn bovenarm moet op jouw bovenarm, in het putteke tussen je biceps en je schouder. Dan trek je zijn arm naar beneden, met je eigen arm als hefboom, zodat een klem op zijn arm wordt gezet. Hierdoor komt hij op zijn tippen te staan, wat hem onstabiel maakt.\\
Dan doe je een roei-beweging, vergelijkbaar met een slag met de bo. Tegelijkertijd schuif je naar voren en neem je zijn plaats in.\\

Houdt dus rekening met hetvolgende:
\begin{itemize}
    \item niet te ver gaan
    \item steeds beide handen gebruiken (kamae houding)
    \item tai sabaki \={o}-irimi niet volledig afwerken = je staat bijna terug recht voor de tegenstander
    \item niet met je heup schudden, met je volledig gewicht inkomen met een rechte rug
\end{itemize}

\subsubsection{\ruby{鉢}{はち}\ruby{廻}{まわ}し\ruby{廻}{まわ}し・hachi mawashi}
Hersenpan roteren.\\
Goed inkomen, met je heup naar voren gericht. Dus borstkas eerst. Tegenstander volledig omver lopen, zodat hij volledig uit evenwicht is.

\subsubsection{\ruby{腰}{こし}\ruby{投}{な}げ・koshi nage}
Heup worp.

\subsection{基本押さえ技・Kihon osae waza/Basis controle technieken・\textbf{ma}}
\subsubsection{\ruby{呂}{ろ}\ruby{伏}{ふ}せ\ruby{入}{い}り\ruby{身}{み}・rofuse irimi}
Rugwervels naar beneden buigen met inkomen van het lichaam.

\subsubsection{\ruby{呂}{ろ}\ruby{伏}{ふ}せ\ruby{転換}{てんかん}・rofuse tenkan}
Rugwervels naar beneden buigen, met omleiden.

\subsubsection{\ruby{後}{うしろ}\ruby{捻}{ねじ}\ruby{砕}{くだ}き・ushiro neji kudaki}
Achterwaarts draaien breken.

\subsubsection{\ruby{小手}{こて}\ruby{砕}{くだ}き・kote kudaki}
Onderarm breken.

\subsection{基本押さえ技・Kihon osae waza/Basis controle technieken・\textbf{chika ma}}
\subsubsection{\ruby{}{}\ruby{}{}・yuki chigae}
TBD

\subsubsection{\ruby{}{}\ruby{}{}・shih\={o} nage}
\textbf{chika ma}\\
TBD

\subsubsection{\ruby{}{}\ruby{}{}・mukae daoshi}
\textbf{chika ma}\\
TBD

\section{Bijkomende technieken}
\subsubsection{\ruby{裏}{うら}\ruby{腕}{うで}\ruby{投}{な}げ・ura ude nage}
Achterkant arm worp.

\subsubsection{\ruby{小手}{こて}\ruby{返}{がえ}し・kote kaeshi}
Onderarm omkering.

\subsubsection{\ruby{後}{うしろ}\ruby{肩}{かた}\ruby{落}{おと}し・ushiro kata otoshi}
Achterwaarts schouder laten vallen.

\subsection{和の精神・Wa no seishin/Geest van harmonie}
Mae, ushiro en met realistische uitdrukkingen: kiai, kime, zanshin en shisei.\\
\begin{description}
    \item[Kiai] Bij het werpen, roepen.
    \item[Kime] Met karakter, determinatie.
    \item[Zanshin] Tot rust komen, net na de worp. Uitademen.
    \item[Shisei] Houding. Rechte rug, kamae, enz.
\end{description}

Je moet er doen op volgende grepen:\\
\begin{description}
    \item[recht op recht] Irimi inkomen, enka en gaan zitten. Hij moet volgen, het gaat te snel voor hem en op het moment dat je gaat zitten, breng je de arm naar beneden en schep je hem omver.
    \item[gekruist] TBD
    \item[mae ry\={o} te dori] Het begin van koshi nage. Begeleiden, maar let er op dat je rechterhand langs de rechterkant van je gezicht blijft, tijdens de \={o}-irimi.
    \item[mae ry\={o} te dori] I.p.v.\ het begin van koshi nage te doen, kan je ook door de knieën gaan en je armen spreiden. Je laat je armen even diep zakken en dan komen ze terug omhoog om achter je te gaan. Daardoor moet de tegenstander over je springen en vallen. Je hoofd moet daarom iets naar opzij, wanneer je gaat zitten.
    \item[recht op recht, langs achter] Als je er 1 langs achter moet doen, dan pakken ze je van voor vast, bvb.\ recht op recht. Je begint hetzelfde, maar nu draai je met je rug naar de tegenstander en presenteer je de andere hand. Hij pakt die dan vast en je vertrekt van daaruit verder om de worp met die hand te doen.
    \item[schouders vastpakken langs voor (ry\={o} sode dori)] Inschuiven, dat je naast hem komt staan. Ondertussen de aanvallende arm begeleiden en vastklemmen onder je arm. Je rechterarm ondersteunt zijn oksel. \={O}-irimi afwerken, door je voet achteruit te zetten, dan enka en gaan zitten. Het is een vrij uitgebreide beweging, t.o.v.\ de andere wa no seishin technieken.
\end{description}

\subsection{型・Kata (zonder wapen)}
\subsubsection{\ruby{八歩}{はっぽ}\ruby{拳}{けん}\ruby{型}{かた}・happoken kata}
8-stappen vuist kata.\\
Groeten naar de sensei.\\
Groeten naar de technische commissie.\\
Bij het groeten, de handen op de dijen, van voor.\\
Rechtervoet naar rechts. Tegelijk de rechterhand omhoog brengen, met de snijkant van de hand naar voren.\\
Linkervoet naar links. Tegelijk de linkerhand omhoog brengen, naar de rechterhand. Een driehoekje vormen, ter hoogte van je solar plexus.\\
Lichtjes door je knie\"{e}n zakken. En uitademen.\\
Driehoek naar boven brengen, tot net boven je hoofd. Dan begint de daling naar beneden.\\
Bij de daling, plooi je de driehoek open, in een grote cirkelvormige beweging. Je eindigt met de ellebogen tegen je lichaam en je 2 handen, met open hand naar boven gericht, recht voor je.\\
Dieper zakken, linkerhand omsluit de rechtervuist. Zie dat je bovenarmen mooi 1 lijn vormen, parallel met de grond en dit ter hoogte van je solar plexus. Je kijkt naar links.\\
Linkervoet iets naar links en een gedan barae afweer doen. Let erop dat je bovenarm parallel is met de grond en dat je onderarm onder een hoek van 45 graden staat.\\
Rechtervoet ook naar voor, zodat hij parallel staat met de linkervoet. Je voeten staat uit mekaar. Tegelijkertijd trek je je linkerarm naar je linkerzij (heup) en maak je een vuist. Ook tegelijkertijd, moet je met je rechterhand een vuist maken en die naar je linkerschouder brengen, terwijl je een elleboog-stoot geeft.\\
Rechtervoet terug naar achter brengen, maar dubbel zo ver. Ondertussen doe je een afweer van een lage trap, met je voorarm (vuist maken, soort van lage gedan barae).\\
Onmiddellijk daarna, de linkervoet bijtrekken, zodat die ook parallel met de rechter komt te staan. Ook weer benen uiteen.\\
Terwijl je dat doet, trek je je rechtervuist naar je heup en geef je een stoot met je linkerhand.\\
Dan met je linkervoet, een kwartslag naar links een ook weer zo een gedan barae doen. Deze wordt nu wel onmiddellijk gevolgd door het vastgrijpen van de denkbeeldige tegenstander zijn vest.\\
Rechtervuist, tsuki ch\={u}dan.
Linkervuist, lage afweerslag naar een aanvallend been.\\
Rechtervuist, tsuki j\={o}dan.
Bij de vuistslagen, elke keer de beide armen laten werken. Als de ene vuist slaat, komt de andere terug naar de heup met snelheid, zodat echtra kracht kan gegenereerd worden.\\
Vanuit die positie, vertrekt de rechtervoet naar rechts, om weer een gedan barae te doen. De kata herhaald zich dan volledig langs de andere kant.\\
Na weer te eindigen met midden, lage en dan hoge slag, is het tijd voor de eindceremonie:\\
Even pauze en diep uitademen.\\
Linkervuist gaat open.\\
Linkervoet achteruit, zodat je beide voeten parallel staan.\\
Linkerhand naar beneden bijtrekken, terwijl je je linkervoet naar het midden verplaatst.\\
Rechtervoet bijtrekken.\\
Groeten, met open handen op de voorkant van het dijbeen.

\subsubsection{\ruby{突}{つ}き\ruby{打}{う}ちの\ruby{型}{かた}・tsuki uchi no kata}
De \textit{stoot/slag kata}, maakt gebruik van alle slagen die beschreven zijn bij het onderdeel tsuki en uchi waza. Je begint met klaar te staan, je vuisten naast je lichaam. Zeg luid en duidelijk ``tsuki uchi no kata'', waarna je 1s pauzeert. Dan val je, met een rechte rug, zachtjes naar voren. Wanneer je niet anders kan als je rechtervoet naar voren plaatsen, geef je ineens een rechte stoot (choku tsuki). Linker hand dichtbij trekken, cfr.\ karate stoten. Verschil met karate: de linkerhand blijft open, zoals bij de gewone kamae houding. Je eindigt die stoot door stevig op je benen te staan, zwaartepunt verlagen en rechte rug houden. Daarna, je rechtervoet een beetje naar rechts schuiven, om dan met je linkerhand een heup-stoot (koshi tsuki) te geven. Daarna kom je terug recht, je trekt je terug in hiki (rechter been geplooid naar achter, de hiel van je rechtervoet is lichtjes van de grond). Tegelijkertijd, geef je ter verdediging een terugtrek-stoot (hiki tsuki). Je linkerhand breng je daarbij onder je rechterarm (net achter de elleboog).\\
\\
Timing:\\
1 2 3 iaaaa 1 2 3 ieeep\\
1 2 3 iaaaa 1 2 3 toooo\\
Dat mogen snelle seconden zijn. Het moet vrij vlot gaan.\\
\\
Extra details:
\begin{itemize}
    \begin{item}
    Starten met je vuisten naast je dijen te houden, met je voeten een beetje uit elkaar, en naar voren te vallen. Let erop dat je voet schuift over de grond en dat je niet op de grond stampt.
    \end{item}
    \begin{item}
    Bij de zweep-slag, moet je goed je bovenlichaam gebruiken on kracht te genereren. Je slaagt dan goed door, waardoor je een beetje scheef eindigt. Daarna zet je je eerst terug in een rechte kamae.
    \end{item}
    \begin{item}
    Eindigen door je recht te zetten. Je vuist wordt een open hand, die komt naast je lichaam, terwijl je naar achter gaat. Twee passen naar achter en je moet terug staan waar je begonnen was. Dan groeten.
    \end{item}
    \begin{item}
    Groeten = handen op je quadriceps (dus effecties van voor en niet langs de zijkant) en buigen, tot je vingertoppen do bovenkant van je knie\"{e}n raken.
    Je ogen volgen mee de buiging, dus naar beneden kijken tijdens het buigen.
    \end{item}
\end{itemize}
Als oefening tussen persoon A en B:\\
Beginnen met rechte stoot langs rechts.\\
\begin{description}
\item [A] Rechte stoot.
\item [B] Afweren met de bovenkant van de rechter-voorarm langs de binnenkant naar links bewegend (gedan barai of zoiets).
\item [A] Heup-stoot.
\item [B] Linkerarm naar de andere kant bewegen om af te weren met de buitenkant van dezelfde voorarm.
\item [B] Met de linkerarm een stoot naar voor geven\ldots rechte stoot die vanuit de hara komt.
\item [A] Hiki tsuki\ldots je doet de stoot hierbij over de arm van de tegenstander, tegen zijn borst. Dicht genoeg bij blijven, zijn stoot kan je net niet raken, omdat jou stoot tegen zijn borstkas, je lichaam net iets langer maakt.
\item [A] Onderdanige slag.
\item [B] Opzij gaan en afweren.
\item [A] Inkomen met elleboog.
\item [B] Nogmaals opzij gaan en afweren.
\item [B] Inkomen om de tegenstander vast te pakken.
\item [A] Zweep-slag (ura) in de zijkant (ribben) van de tegenstander, terwijl je achteruit gaat.
\end{description}

\subsubsection{\ruby{座}{すわ}り\ruby{技}{わざ}の\ruby{型}{かた}・suwari waza no kata}
Zittende technieken kata.\\
Voor mekaar zitten.\\
1.\\
Je wordt vastgepakt. Rechterhand gebruiken om naast hem te bewegen en hem in een cirkelvormige beweging uit evenwicht te brengen. Als hij valt, bijschuiven en zijn linkerarm over je 2 knie\"{e}n leggen, pols vasthouden ter controle. Slag naar het gezicht (hij moet het gezicht beschermen). Die linkerarm tussen jou en de tegenstander steken. Tegenstander strekt 1 been, andere knie geplooid over dat gestrekt been. Zo moet hij ook wegrollen, van de andere weg. Terug naar je plaats gaan in chico. Andere zal ondertussen rechtkomen en ook terug naar zijn plaats gaan.
2.\\
Idem, maar nu zal de linkerhand ook voor een balansverstoring zorgen, naar boven gericht. Hand in kamae naar binnen gericht (dus niet de arm vastpakken). De rest is hetzelfde.

\subsubsection{?? daito ryuu, ceremonie}
Start: 6 matten afstand\\
Voeten uit mekaar (want ze hadden een harnas aan).\\
Groeten naar mekaar, met de handen open, van voor op de dijbenen.\\
Draaien naar de meester.\\
Voeten in mekaar (rechter in de linker) en groeten, ook met de handen open, van voor op de dijbenen.\\
Draaien naar mekaar.\\
Zitten in seiza, passief.\\
Groeten: linker hand op de grond, rechterhand volgt.\\
Naar mekaar blijven kijken en buigen.\\
Rechterhand maar even op de grond en dan terug naar boven.\\
Terug de rug recht brengen.\\
Tenen actief.\\
Naar mekaar komen, met de handen in de lies. Deze keer, vuisten maken rond de duim. Hierdoor kan de tegenstander moeilijker je arm vastpakken, want dan schuift hij er af.\\
Een dikke 2 vuisten afstand houden. Dat mag 30cm zijn, dat werkt gemakkelijker.\\
Einde: Tenen actief en in chico terug achteruit. Weer 6 matten afstand.\\
Tenen passief. Op dezelfde manier groeten.\\
Rechtstaan.\\
Op dezelfde manier groeten.\\
Weer draaien naar de meester en groeten.

\subsubsection{?? daito ryuu, 1ste}
\textbf{aanvaller:}\\
Rechte slag naar beneden met rechterhand.\\
Rechterknie stapt rechts uit.\\
\textbf{verdediger:}\\
Vertrekken wanneer de aanval nog moet beginnen.\\
Dubbele hand (op mekaar) schiet naar voor om op de achterkant van de bovenarm (boven de oksel) te duwen.\\
Die dubbele hand schuift tegelijk een beetje naar boven richting de elleboogholte.\\
Hierdoor wordt de aanvaller uit evenwicht geduwd.\\
Over arm schuiven om de hand vast te pakken en die op je rechterknie te leggen (je rechterknie was rechts uitgestapt).\\
Linkerhand ter hoogte van de elleboog.\\
2 details:
\begin{itemize}
    \item de vingertoppen van de linkerhand moeten naar boven gericht zijn.
    \item duwen \textit{op} de elleboog, het is de bedoeling om dat gewricht kapot te doen.
\end{itemize}
Je duwt de elleboog richting je rechtervoet.\\
Rechterhand blijft vasthouden. Rechterknie gaat naar voren. Hier kan je op elke moment de arm nog breken.\\
Hand plat op de grond leggen, dus niet getorst.\\
Linkerhand duwt op de bovenarm, achter de elleboog, ter controle.\\
Linkerknie op de schouder leggen, ter controle.\\
Nu kan je de linkerhand gebruiken om een ``tooo'' te doen.\\
Met linkerhand terug achter de elleboog vastpinnen.\\
Linkerknie achteruit.\\
Linkerhand laat los.\\
Rechterknie achteruit.\\
Rechterhand houd nog 1 seconde vast, om controle te laten zien, voor het geval hij nog leeft.\\
Rechterhand laat los.\\
In kamae.\\
Andere laten rechtkomen.\\
Vuisten rond duimen maken in de lies.\\
Aanvaller gaat eerst naar zijn plaats.\\
Dan verdediger.

\subsubsection{?? daito ryuu, 2de}
Aanvaller:\\
Vastpakken vest met linkerhand en roepen.\\
Zijn rechterknie stapt rechts uit.\\
Verdediger:\\
2 stoten naar boven:
\begin{itemize}
    \item linkervuist raakt zijn elleboog
    \item rechtervuist op zijn kin
\end{itemize}
Detail: Kom een beetje naar voor, om de overhand te krijgen in het evenwicht. Dus overtuigend inkomen.\\
Dan komen de 2 vuisten naar beneden in nog een dubbele slag:
\begin{itemize}
    \item linkervuist slaat in een drukpunt schep beweging (van biceps aanvaller naar voorarm), om hem naar voor te laten buigen
    \item rechtervuist slaat op zijn neusbrug
\end{itemize}
Tegelijkertijd mae keri met je linkerbeen.\\
Linkerbeen ineens naar achter zetten, ver genoeg.\\
Ondertussen draait je lichaam mee en dat zal zijn arm in een klem brengen.\\
Je linkerhand mag helpen, weer met de vingertoppen naar boven gericht, maar in eendebakkes C vastpakken.\\
Afwerking, zoals bij de eerste.

\subsubsection{?? daito ryuu, 3de}
\textbf{aanvaller:}\\
Rechte stoot. Rechterknie stapt rechts uit.\\
\textbf{verdediger:}\\
Tegelijkertijd als beide handen in kamae naar voor en een shinogi doen.\\
Let er op dat die shinogi van ver genoeg vertrekt en naar je lichaam komt.\\
Arm aanvaller dus niet wegduwen.\\
Vuistslag laten doorgaan en weer laten neerkomen en afwerken.\\
Aanvaller moet normaal gezien recht liggen, in de richting van de vuistslag.

\subsubsection{?? daito ryuu, 4de}
Kuruma daoshi, de wiel neerhaling.\\
\textbf{aanvaller:}\\
Rechterhand slaat met gesloten vuist (zijdelingse hamerslag), van achter beginnen in een cirkelvormige beweging.\\
(Zelfde slag als bij suwari waza no kata.)\\
\textbf{verdediger:}\\
Voor de aanval start, met open hand de aanval blokkeren op de schouder. Paf!\\
Eigenlijk niet helemaal blokkeren. Je zal met de linkerhand ineens de rechterarm van de tegenstander \textit{begeleiden}, om die cirkelbeweging verder te zetten. Daardoor komt de tegenstander ten val en land hij op zijn rug.\\
Je linkerknie is intussen naar achter gedraait en omhoog gekomen.\\
Met rechterhand een ``tooo'' doen.\\
De aanvaller moet zijn gezicht beschermen en ook zijn hoofd van de grond houden.\\
Afwerken door je rechterhand onder de rechterarm van de tegenstander te steken. Dan die arm over hem duwen, ter hoogte van de bovenarm langs de buitenkant.\\
Even houden, ter controle.\\
In kamae achteruit.\\
Aanvaller moet wegrollen van de verdediger.\\
Komt dan recht.\\
Kamae verdediger worden 2 vuisten rond de duimen.\\
Aanvaller gaat terug op zijn plaats.\\
Verdediger ook.

\subsubsection{?? daito ryuu, 5de}
Gekruist wurgen.\\
2 atamis in de ribben.\\
Niet uitstappen!\\
1 arm naar beneden, 1 naar boven.\\
De nadruk op de arm naar beneden, dat is de belangrijkste.\\
Neerleggen en mee indraaien.\\
Je doet eigenlijk een \={o}-irimi.\\
Je eindigt dan langs de zijkant, terwijl zijn arm vertikaal is.\\
Hij heeft je immers nog vast.\\
Je linkerhand doet de afwerking.\\
Je rechterhand schuift er dan onder, duwt de arm over de aanvallen en houdt even controle.\\
Dan achteruit gaan in kamae en wachten.\\
De aanvaller draait weg van de verdediger en zet zich recht.

\subsubsection{\ruby{蹴}{け}り\ruby{五歩}{ごほ}の\ruby{型}{かた}・keri goho no kata}
Trap 5 stappen kata.

\subsection{型・Kata (met wapen)}
\subsubsection{tanto no kata}
Groeten + Tanto no kata zeggen.\\
Begin zoals bij katori: chidori vertrekken en tanto trekken.\\
In een grote beweging naar het hoofd slagen. Let er op dat de punt goed naar voor is, want daar snij je het gezicht mee.\\
Die beweging stopt niet in kamae houding, maar gaat verder om een 8-snijbeweging te maken, zoals in eskrima.\\
Beginnen langs rechts en er op letten dat je niet te ver naar achter gaat. Je tanto moet steeds voor je blijven.\\
Snijden, weer met de punt goed naar voor, in de richting van de gi-lijnen.\\
Na de 8-beweging, doe je onmiddellijk een steek. Je linkervoet verplaatst \textit{niet}. Je leunt gewoon naar voor om te steken.\\
Dit was het \textit{eerste} blok. Kiai mag hier.\\
Dan de tanto overpakken naar earth grip. Je houdt het boven je hoofd, met de snijkant naar links.\\
Even wachten en dan haken naar rechts, zogezegd in het linkeroog om dat neusbeen open te rijten. Terugkomen met de punt goed naar voor om de hals open te snijden. Linkervoet gaat achterom je rechterbeen, terwijl je naar voor blijft kijken met getorst lichaam. Ondertussen heb je je linkerhand onder de tanto geplaatst. Steken naar voren, in de buik (onder de borstplaat). En dan de tanto klinken, zodat je het geheel naar boven kunt duwen, onder de borst, tot in de longen.\\
Dit was de \textit{tweede} blok. Kiai mag hier.\\
Linkerhand lost de ondersteuning en pakt de tanto over, net onder de tsuba. Uit de torsing draaien en blok 1 begint terug, maar nu langs de andere kant.\\
Dat wordt de derde blok. Kiai op het einde.\\
Dan weer earth grip en neusbeen openrijten. Dat is de vierde blok. Kiai.\\
Laatste keer uit de torsing draaien en snijden in het gezicht, met de punt goed naar voren. Laatste kiai.\\
2 passen achteruit en notou.

\subsubsection{\ruby{短}{たん}\ruby{棒}{ぼう}の\ruby{型}{かた}・tanbo no kata}
Groeten + Tanbo no kata zeggen.\\
Begin zoals bij katori: chidori vertrekken en tanbo trekken.\\
In een grote beweging naar het hoofd slagen. Het gaat hier om een slag, recht naar beneden.\\
Die beweging wordt onmiddellijk gevolgt door een slag naar de linkerkant van het gezicht. Een men-slag, dus schuin naar beneden.\\
Daarna een gelijkaardige slag langs rechts.\\
Terug links, maar een lage slag.\\
Terug rechts, een lage slag.\\
Tanbo omdraaien in je rechterhand, zodat er maar een kleine punyo naar voor gericht is. Tegelijk inschuiven met heel je lichaam om een ferme stoot te geven op zijne solar plexus. Dit wordt gevolgt door een stevigere kiai en een kleine pauze.\\
Daarna pakt de linkerhand de bo over. Linkervoet komt naar voor en doet een slag naar het hoofd, weer recht naar beneden. Hetzelfde wordt nu herhaald, maar dan in spiegelbeeld. Na de finale solar plexus stoot, weer eindigen met een stevige kiai en een kleine pauze.\\
Dan pakt de rechterhand de tanbo weer over.\\
Diep uitademen.\\
2 passen achteruit en notou.\\
Voeten bij mekaar en terug groeten.

\subsubsection{\ruby{剣}{けん}の\ruby{型}{かた}・ken no kata}
Zwaard trekken. In kamae voor mekaar staan. Allebei 1 maki uchi doen.\\
Ene geeft opening (te ura gasumi). De andere valt men aan.\\
Rechts/links werken.\\
Men R, men L.\\
Do R, do L.\\
\={O}-gasumi. De andere gaat naar \textit{in no kamae}. Dan valt hij aan. Achteruit gaan en zwaard bijtrekken.\\
De andere geeft nog een maki uchi. Je zet je linkerbeen naar achter en gaat in gedan no kamae staan.\\
Nu sta je terug zoals in't begin en is het jouw beurt om opening te laten.\\
Ieder doet zo 1 kant.

\subsubsection{bo no kata}
Is bijna hetzelfde als ken no kata, met enkele verschillen.\\
Bo in de helft vastpakken. Rechterhandpalm naar boven gericht, daar ligt de bo in.\\
Groeten, zonder dat de bo beweegt.\\
Bo naar voren schuiven en tegoei vastpakken voor aan te vallen. Linkerhand pakt over, langs boven. Achteruit zwieren en een maki uchi doen. Nu sta je in begin-positie en doe je iets gelijkaardig aan de bewegingen van ken no kata.\\
Men R, men L.\\
Do R, do L.\\
Een bo is lang, dus kan je ook laag gaan.\\
Sune R, sune L. Goed door de knie\"{e}n zakken.\\
I.p.v.\={o}-gasumi, doen we hier mateage. Dat kunnen we zowel R als L doen.\\
Een slag komt, naar rechts springen en opvangen. Opzij duwen en dreigen naar de keel.\\
Allebei een maki uchi. En dan sta je weer op dezelfde manier.\\
En die blok nog eens herhalen, maar met de rollen gewisseld.

\subsection{Randori}
\subsubsection{1 tegen 1}
Diverse technieken, ma en chika ma. Soepel werken.

\subsubsection{Canalisations}
Je moet er doen op volgende aanvallen:\\
\begin{description}
    \item[ura yoko men uchi] Beginnen zoals bij mukae daoshi, maar dan arm onder je arm klemmen en in enka gaan zitten.
    \item[omote yoko men uchi] Slag laten komen, begeleiden met rechterarm en naar voren smijten, terwijl je op zijn plaats gaat zitten. De aanvallende beweging moet zich voortzetten in een cirkel, waardoor hij valt.
    \item[tsuki ch\={u}dan] Irimi inkomen, vuist laten passeren. Vastklemmen, enka en gaan zitten.
    \item[tsuki j\={o}dan] Vuist laten komen, begeleiden met rechterarm en naar voren smijten, terwijl je op zijn plaats gaat zitten.
\end{description}

\subsubsection{Wa no seishin}
Zie hoger, dit is reeds in detail uitgelegd.

\subsection{Extra informatie}
Kihon nage waza en kihon osae waza zijn het belangrijkste.

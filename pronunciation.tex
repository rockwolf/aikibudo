\noindent Een aantal letters worden iets anders uitgesproken in de romaji uitspraak, t.o.v. het Nederlands.
Het is interessant om dit even te lezen, om zo geen verkeerde uitspraak te leren voor de techniek.\\

\noindent j: $|zj|$ zoals in strand\textbf{j}anet, \textbf{J}efke, ... (en NIET zoals in \textbf{J}ommeke)\\
ch: $|tsj|$ zoals in \textbf{tsj}oeke \textbf{tsj}oeke tuut tuut\\
sh: $|sj|$ zoals in \textbf{ch}oco, \textbf{sh}ampoo, ...\\
u: korte $|oe|$ zoals in nen t\textbf{oe}k \textbf{oe}p aa bakkes, vake en m\textbf{oe}ke, ...\\
\={o} of ou: lange $|o|$ zoals in b\textbf{oo}t\\
\={u} of uu: lange $|u|$ zoals in v\textbf{uu}r
